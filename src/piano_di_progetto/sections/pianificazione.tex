\section{Pianificazione} \label{sec:pianificazione}

Le due scadenze previste, la RTB e la PB, verranno suddivise in Sprint. 

\noindent
Ogni Sprint è associato ad una milestone, così facendo si mantiene un sistema di riferimento temporale e una struttura che permette di monitorare il progresso nel corso del progetto.

\vspace{0.3cm}
\noindent
Durante ciascuno Sprint sarà settato un obiettivo primario, ma si includeranno anche altre attività presenti nel Backlog. Questo modello Sprint consente al team di concentrarsi su compiti specifici e, allo stesso tempo, offre la possibilità di adattare attività che emergono durante lo sviluppo, garantendo un approccio più agile e adatto al cambiamento.

\noindent
La RTB verrà suddivisa nei seguenti Sprint con i relativi obiettivi primari:
\begin{itemize}
    \item Sprint-1 (dal 13/11/2023 al 03/12/2023): Analisi preliminare;
    \item Sprint-2 (dal 04/12/2023 al 17/12/2023): Progettazione Technology Baseline;
    \item Sprint-3 (dal 18/12/2023 al 31/12/2023): PoC - server;
    \item Sprint-4 (dal 01/01/2024 al 07/01/2024): PoC - email.
\end{itemize}

\noindent
La PB verrà suddivisa nelle seguenti fasi:
    \begin{itemize}
    \item Sprint-5 (dal 29/01/2024 al 04/02/2024): Architettura;
    \item Sprint-6 (dal 05/02/2024 al 18/02/2024): Progettazione;
    \item Sprint-7 (dal 19/02/2024 al 03/03/2024): Codifica;
    \item Sprint-8 (dal 04/03/2024 al 17/03/2024): Stress test;
    \item Sprint-9 (dal 18/03/2024 al 24/03/2024): Revisione generale.
\end{itemize}

La seguente pianificazione non riporterà azioni ritenute ovvie come lo sviluppo dei verbali o la verifica dei documenti. Queste attività sono descritte dettagliatamente nelle Norme di Progetto sezione xxxxxxx. %TODO




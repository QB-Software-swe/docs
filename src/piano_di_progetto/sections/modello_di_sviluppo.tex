\section{Modello di sviluppo} \label{sec:modello-sviluppo}
Il gruppo ha optato per adottare un approccio di sviluppo Agile basato sul modello Scrum per gestire il progetto. 
Questo implica una metodologia iterativa e incrementale, focalizzata sull'ottenimento di risultati rapidi e sull'adattamento alle eventuali evoluzioni del contesto.
 
Nella pratica, il processo di lavoro segue le seguenti fasi:
\begin{itemize}
    \item \textbf{backlog requisiti}: ogni requisito è inserito in un backlog con un grado di importanza associato, determinato in base alla sua criticità. Questo consente al team di sapere quali sono i requisiti più significativi e quindi di concentrarsi prima su quelli;
    \item \textbf{pianificazione Sprint}: per ogni Sprint vengono selezionati i requisiti dalla backlog tenendo conto della loro priorità e della disponibilità delle risorse del team;
    \item \textbf{definizione obiettivi}: all'inizio di ogni Sprint, si stabiliscono gli obiettivi da raggiungere entro una data specifica. Questo processo prevede anche la previsione anticipata dei potenziali rischi;
    \item \textbf{assegnazione compiti}: il Responsabile distribuisce gli obiettivi identificati al team, assegnando compiti specifici e indicando le relative scadenze;
    \item \textbf{verifica}: i verificatori analizzano il lavoro svolto dal team per individuare eventuali errori, permettendo così un controllo della qualità costante;
    \item \textbf{retrospettiva}: al termine di ciascuno Sprint, il gruppo si riunisce per valutare il lavoro svolto e l'avanzamento del progetto. Vengono discusse eventuali modifiche necessarie per affrontare le sfide incontrate durante lo Sprint.
\end{itemize}

Questo approccio consente al team di fornire risultati iterativi e incrementali e di risolvere velocemente eventuali problematiche che emergono nel corso del lavoro.

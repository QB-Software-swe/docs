\subsection{Capitolato C5 - WMS3: warehouse management 3D} \label{sec:c5}
	\subsubsection{Descrizione}
	
		\begin{table}[H]
			\begin{tabularx}{\textwidth}{p{3cm}|X}
				\toprule
				\textbf{Proponente} & Sanmarco Informatica SPA
				\\
				\textbf{Committenti} & Proff. Tullio Vardenega e Riccardo Cardin
				\\
				\textbf{Obiettivo} & Warehouse Management 3D
				\\\toprule
			\end{tabularx}
			\caption{informazioni di carattere generale sul capitolato.}
		\end{table}
		
		Il capitolato descrive l’ idea di un magazzino 3D: un ambiente che permetta di rappresentare oggetti e ambienti in modo tridimensionale, molto utile durante la	progettazione di un magazzino.
		
	\subsubsection{Dominio applicativo}
		Si richiede un applicativo o una web-app che possa consentire di progettare e simulare le scaffalature all’interno di spazi fisici valutando anche i flussi di movimentazione dei materiali.
		%
		Il prodotto finale dovrà avere le seguenti caratteristiche:
		\begin{itemize}
			\item l’ambiente 3D dovrà essere navigabile tramite tastiera/mouse e preferibilmente;
			\item il layout dovrà essere caricato da un database;
			\item possibilità di aggiungere scaffalature e ridimensionarle a piacimento;
			\item permettere lo spostamento di un prodotto da una scaffalatura ad un’altra;
			\item possibilità di identificare facilmente le aree disponibili in cui posare gli oggetti.
		\end{itemize}
		%
		Non viene richiesto un sistema di login, l’unico utilizzatore sarà un amministratore.
		Inoltre non sarà necessario realizzare alcun modulo che permetta la persistenza dei
		dati o l’integrazione di sistemi di realtà aumentata con VR (Virtual Realtiy).

	\subsubsection{Dominio tecnologico}
		Per questo progetto l’azienda consiglia:
		\begin{itemize}
			\item Three.js, framework JavaScript che permette di creare ambienti 3D in un browser web;
			\item API di gestione e posizionamento dei prodotti (se ritenute necessarie).
		\end{itemize}
		%
		In alternativa suggerisce altre 2 tecnologie:
		\begin{itemize}
			\item Unity, motore grafico sviluppato da Unity Technologies (C\#);
			\item Unreal Engine, motore grafico sviluppato da Epic Games (C++).
		\end{itemize}

	\subsubsection{Aspetti positivi}
		\begin{itemize}
			\item Questo progetto ha riscontrato l’interesse di molti componenti del gruppo;
			\item l’ azienda si rende disponibile con contatti frequenti tra gruppo di lavoro e proponente.
		\end{itemize}		

	\subsubsection{Fattori critici}
		\begin{itemize}
			\item Problemi riguardanti il carico computazionale richiesto dai software per sviluppo 3D;
			\item prevista una complessità importante per quanto riguarda la permanenza dei dati e la mobilità degli oggetti.
		\end{itemize}
	
	\subsubsection{Conclusioni}
		Per concludere, nonostante il progetto abbia ricevuto l’approvazione da molti componenti del gruppo, abbiamo scelto di non cimentarci in questo progetto. Il motivo primario è che i software di sviluppo valutati dal gruppo, Unity ed Unreal, richiedono computer con specifiche tecniche superiori alla maggior parte dei componenti del gruppo, rendendo così impossibile per tutti gli studenti partecipare attivamente allo sviluppo.
\subsection{Capitolato C8 - JMAP, il nuovo protocollo standard per la comunicazione via email} \label{sec:c8}
	\subsubsection{Descrizione}
	
		\begin{table}[H]
			\begin{tabularx}{\textwidth}{p{3cm}|X}
				\toprule
				\textbf{Proponente} & Zextras
				\\
				\textbf{Committenti} & Proff. Tullio Vardenega e Riccardo Cardin
				\\
				\textbf{Obiettivo} & Sviluppare un servizio di posta elettronica con JMAP
				\\\toprule
			\end{tabularx}
			\caption{informazioni di carattere generale sul capitolato.}
		\end{table}
		
		Il capitolato ha per oggetto lo sviluppo di un servizio di posta elettronica che utilizza il protocollo JMAP.
		
		Il servizio deve essere testabile per permettere all'azienda di valutare le prestazioni, la manutenibilità e la completezza del protocollo JMAP, paragonandolo agli attuali protocolli attualmente implementati in Carbonio, piattaforma gratuita e open source per la collaborazione e la gestione dell'e-mail, sviluppato da Zextras.
		
	\subsubsection{Dominio applicativo}
		Il capitolato si colloca nell'ambito delle telecomunicazioni, più precisamente nella posta elettronica on-premise e nella sincronizzazione dei dati correlati: contatti e calendari. I protocolli standard attualmente utilizzati, POP e IMAP, sono abbastanza datati e le necessità sorte in questi ultimi anni hanno portato l'adozione di soluzioni personalizzate per superare le loro limitazioni, determinando la necessità di sviluppare un protocollo su misura o proprietario.
		
		Per fare fronte alle nuove necessità nasce JMAP: un protocollo efficiente e moderno. Il prodotto da sviluppare dovrà essere in grado, ad esempio, di eseguire le seguenti operazioni:
		\begin{multicols}{2}
			\begin{itemize}
				\item l'invio e la ricezione di una e-mail;
				\item la gestione delle cartelle;
				\item la gestione dei contenuti di una cartella;
				\item l'eliminazione di un oggetto;
				\item l'eliminazione di una cartella;
				\item la condivisione di una cartella;
				\item l'eliminazione di una condivisione cartella.
			\end{itemize}
		\end{multicols}
		
	\subsubsection{Dominio tecnologico}
		Per lo svolgimento del capitolato, l'azienda richiede l'utilizzo delle seguenti tecnologie:
		\begin{itemize}
			\item protocollo JMAP: per la comunicazione tra server e client. In particolare, si deve usare una delle \href{https://jmap.io/software.html\#libraries}{librerie ufficialmente riconosciute da JMAP};
			\item un sistema di container: per poter lanciare in simultanea più istanze del servizio;
			\item uno dei \href{https://jmap.io/software.html\#clients}{client ufficalmente supportati}.
		\end{itemize}
		Sono state inoltre consigliate le seguenti tecnologie:
		\begin{itemize}
			\item Java: per lo sviluppo backend;
			\item Docker: come sistema di container.
		\end{itemize}
		
	\subsubsection{Aspetti positivi}
		\begin{itemize}
			\item Presentazione chiara, scopi e obbiettivi dello stakeholder ben definiti;
			\item interesse da parte del gruppo per le tecnologie proposte;
			\item opportunità di lavorare con un protocollo all'avanguardia che è:
%           \begin{multicols}{3}
                \begin{itemize}
    				\item moderno;
                    \item espandibile;
                    \item flessibile;
    				\item sicuro;
    				\item veloce;
    				\item open source;
    				\item pubblicato ufficialmente come standard dall'IETF (Internet Engineering Task Force).
    			\end{itemize}
%           \end{multicols}
			\item l'opportunità di sviluppare un servizio basato su contenitori rappresenta un'occasione rilevante, in quanto si tratta di una tecnologia in constante crescita e ampiamente adottata dalle aziende;
			\item possibilità di utilizzare Java come linguaggio di programmazione, che gode delle seguenti peculiarità:
			\begin{itemize}
				\item implementazioni di feature moderne dei linguaggi di programmazione;
				\item implementazione delle ultime tecnologie sviluppate per JVM, che solitamente sono facilmente portate in Java.
			\end{itemize}
		\end{itemize}
		
	\subsubsection{Fattori critici}
		\begin{itemize}
			\item Il protocollo JMAP è completo solo per il supporto all'e-mail e la parte core, mentre le altre funzionalità sono ancora in sviluppo;
			\item implementazione limitata per alcuni linguaggi delle librerie proposte, con rischio di cambiamenti molto repentini nella struttura dell'interfaccia della librerie.
		\end{itemize}
		
	\subsubsection{Conclusione}
		In seguito all'incontro, molto positivo, con Zextras, i dubbi riguardanti le criticità sono stati chiariti. Il team QB Software, preso atto delle criticità e dell'opportunità (fattori positivi), e in vista del crescente interesse del gruppo, ha deciso di impegnarsi nello sviluppo del capitolato C8.
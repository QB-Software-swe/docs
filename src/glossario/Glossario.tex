% DOC TYPE E DEF QB SOFTWARE %%%%%%%%%%%%%%%%%%%%%%%%%
\documentclass[12pt]{article}
\renewcommand*\contentsname{Indice}
%%%%%%%%%%%%%%%%%%%% PACKAGES %%%%%%%%%%%%%%%%%%%%
\usepackage[english,italian]{babel}
\usepackage[a4paper, margin=3cm]{geometry}
\usepackage[T1]{fontenc}
\usepackage[utf8]{inputenc}
\usepackage[table]{xcolor}
\usepackage{amsmath}
\usepackage{graphicx}
\usepackage{float}
\usepackage{tabularx}
\usepackage{booktabs}
\usepackage{hyperref}
\usepackage{xcolor}
\usepackage{multicol}
\usepackage{multirow}
\usepackage{soul}
\usepackage{enumitem}
\usepackage{textcomp}
\usepackage{eurosym}
\usepackage{lastpage}
\usepackage{fancyhdr}
\usepackage{adjustbox}
\usepackage{subfiles}

%%%%%%%%%%%%%%%%%%%% COMMAND -> New commands %%%%%%%%%%%%%%%%%%%%
\newcommand{\mailtoQBS}
{
	\href{mailto:qbsoftware.swe@gmail.com}{qbsoftware.swe@gmail.com}
}

\let\oldpar\paragraph
\renewcommand{\paragraph}[1]{\oldpar{#1}\mbox{}\\}

%%%%%%%%%%%%%%%%%%%% COMMAND -> Redefine LaTeX Commands %%%%%%%%%%%%%%%%%%%%
\def\title#1{\gdef\THETITLE{#1}}
\def\date#1{\gdef\THEDATE{#1}}
\def\footname#1{\gdef\THEFOOTNAME{#1}}

\let\oldtexteuro\texteuro
\renewcommand{\texteuro}{\euro}

%%%%%%%%%%%%%%%%%%%% STYLES -> Commands %%%%%%%%%%%%%%%%%%%%
\newcommand{\makefirstpage}
{
	\begin{titlepage}
		
		% Defines a new command for the horizontal lines, change thickness here
		\newcommand{\HRule}{\rule{\linewidth}{0.2mm}} 
		
		\center % Center everything on the page
		
		
		%	Heading Sections
		
		\textsc{\LARGE QB Software}\\[.1cm] 
		\includegraphics[scale=.15]{imgs/qb-software-logo.png}\\[-.1 cm] 
		\includegraphics[scale=.025]{imgs/x.png}\\[0.5cm]
		\includegraphics[scale=.3]{imgs/unipd_logo.png}\\[.5cm]
		\textsc{\Large Università degli studi di Padova}\\[0.5cm] 
		\textsc{\large corso di ingegneria del software }\\[0.5cm]
		\textsc{\large anno accademico 2023/2024 }\\[0.5cm]
		
		
		%	Title section and date section
		
		\ifdefined\THEDATE
		\HRule \\[0.4cm]
		{ \huge{ \bfseries {\THETITLE}} \\ [.5cm]
			\THEDATE}\\[0.4cm] 
		\HRule \\[0.4cm]
		\else
		\HRule \\[0.4cm]
		{ \huge{ \bfseries {\THETITLE}} } \\
		\HRule \\[0.4cm]
		\fi
		%
		\textsc{Contatti:} \mailtoQBS\\[0.3cm]
		
		\vfill 
		
	\end{titlepage}
}

%%%%%%%%%%%%%%%%%%%% ACCESSIBILITY %%%%%%%%%%%%%%%%%%%%
\let\oldhref\href
\renewcommand{\href}[2]{\oldhref{#1}{\textcolor{blue}{\ul{#2}}}}

\hypersetup{
	colorlinks = true,
	linkcolor = cyan,
}

%%%%%%%%%%%%%%%%%%%% STYLES -> Settings %%%%%%%%%%%%%%%%%%%%
% Enable header and footer style
\pagestyle{fancy}
\thispagestyle{empty}
\thispagestyle{fancy}
\pagestyle{fancy}

% Header
\setlength{\topmargin}{-40pt}
\setlength{\headsep}{60pt}
\fancyhf{}
\lhead{QB Software}
\setlength{\headheight}{15pt}
\rhead{\includegraphics[width=1cm]{imgs/qb-software-logo.png}}

% Footer 
\fancyfoot{}
\fancyfoot[L]{\THEFOOTNAME}
\fancyfoot[R]{Pagina \thepage~di~\pageref{LastPage}}
\futurelet\TMPfootrule\def\footrule{\TMPfootrule}
\setcounter{page}{0}
\pagenumbering{arabic}
\renewcommand{\footrulewidth}{0.3pt}

%%%%%%%%%%%%%%%%%%%% ENVIRONMENT %%%%%%%%%%%%%%%%%%%%
% Remove colorless padding in booktabs
\setlength{\aboverulesep}{0cm}
\setlength{\belowrulesep}{0cm}
\setlength{\extrarowheight}{.75ex}

\newenvironment{todo}{
	\rowcolors{2}{cyan!80!black!30!}{cyan!80!black!20!}
	\begin{tabular}{p{3.48cm}>{\raggedright\arraybackslash}p{4cm}>{\raggedright\arraybackslash}p{6.5cm}}
		\toprule
		\rowcolor{gray!20} \textbf{ID}	& \textbf{Interessato} & \textbf{Task} 
		\\\midrule
	}{
		\bottomrule
	\end{tabular}
}

\newenvironment{changelog}{
	\noindent
	{\Large \textbf{Registro delle modifiche}}
	\noindent
	\begin{table}[h]
		\rowcolors{2}{cyan!80!black!30!}{cyan!80!black!20!}
		\begin{adjustbox}{width=\textwidth}
			\begin{tabular}{|c|c|p{2.35cm}|c|p{3.3cm}|}
				\hline
				\rowcolor{gray!20}
				\textbf{V.} & \textbf{Data} & \textbf{Membro} & \textbf{Ruolo} & \textbf{Descrizione} \\
				
				\hline
			}{
				\hline
			\end{tabular}
		\end{adjustbox}
	\end{table}
	
	\clearpage
}

% 1   2    3        4             5            6            7          
% Ver Data Relatore RuoloRelatore DataVerifica Verificatore Descrizione
\newcommand{\newlog}[7]{
	#1 & #5 & #6 & Verificatore & Controllo qualità \\
	   & #2 & #3 & #4           & #7 \\\hline
}

\newcommand{\milestone}[3]{
	#1 & #2 & #3 & Responsabile & Approvazione \par documento
	\\\hline
}

% INFORMAZIONI DOCUMENTO %%%%%%%%%%%%%%%%%%%%%%%%%
\title{Glossario}
%\date{EMPTY}
\footname{Glossario}

\begin{document}
	% PRIME PAGINE %%%%%%%%%%%%%%%%%%%%%%%%%
	\makefirstpage
	
	% 1   2    3        4             5            6            7          
% Ver Data Relatore RuoloRelatore DataVerifica Verificatore Descrizione
\begin{changelog}
	\newlog{0.1.0}{12/12/2023}{A. Feltrin}{Responsabile}{16/12/2023}{A. Domuta}{Stesura verbale}
\end{changelog}
	\clearpage
	
	\tableofcontents
	\clearpage
	
	% CONTENUTO QUI %%%%%%%%%%%%%%%%%%%%%%%%%
	\section{A}
		\subsection{API}
			Sta per Application Programming Interface. \'E un insieme di regole e strumenti che permette a diversi software di comunicare tra loro. Questo avviene definendo dati e metodi che le applicazioni possono usare per comunicare tra loro.
		\subsection{Attivit\'a}
			Secondo lo standard ISO 12207:1997, un'attivit\'a, nel contesto di ingegneria del software, \'e un'operazione o un insieme di operazioni che son necessarie per il raggiungimento di un obiettivo. 
	\clearpage	
	\section{B}
		\subsection{Baseline}
			\'E un punto raggiunto e consolidato che funge da base. Si usa per riferirsi a una versione specifica di un prodotto software. Viene usata come punto di riferimento per lo sviluppo e la gestione futura del software.
		\subsection{Branch}
			Linea di sviluppo parallela e indipendete all'interno di una repository. Rende possibile agli sviluppatori di creare feature contemporaneamente sul medesimo progetto evitando interferenze.
	\clearpage	
	\section{C}
		\subsection{Capitolato}
			Documento redatto dal proponente e specifica i vincoli contrattuali, tra cui scadenze e budget per il progetto stesso. Viene presentato in una gara d'appalto per cercare un fornitore in grado di occuparsi del progetto.
		\subsection{Caso d'uso}
			Scenari di utilizzo che descrivono le funzionalit\'a di un sistema dal punto di vista dell'utente, vengono specificati: input, attivit\'a e risposte del sistema. Non fornisce dettagli implementativi. Forniscono una panoramica completa delle funzionalit\'a di sistema e dei flussi di lavoro. Definire i casi d'uso \'e fondamentale per comprendere i bisogni dell'utente.
		\subsection{Client} % Da definire
			\subsection{Ciclo di vita del software} % Da definire
		\subsection{Code coverage}
			Si usa per misurare la porzione di codice che \'e soggetta all'esecuzione dei test. Il codice che non viene coperto dai test \'e in potenza difettoso.
		\subsection{Committente}
			\'E l'entit\'a che fornisce vincoli contrattuali e requisiti, commissiona la realizzazione di un prodotto o servizio.
		\subsection{Compito}
			Prevede attivit\'a e responsabilit\'a, viene assegnato a una persona o pi\'u in base agli obiettivi che si vogliono raggiungere. Ha delle scadenze e pu\'o essere suddiviso in compiti pi\'u piccoli o attivit\'a per semplificarne la comprensione ed esecuzione.
	\clearpage
	\section{D}
		\subsection{Deployment}
		Fase del ciclo di vita del software, che conclude lo sviluppo e il relativo collaudo e dà inizio alla manutenzione.
		\subsection{Design Pattern}
		Soluzione progettuale generale ad un problema ricorrente. Non è un codice finito che può essere trasformato direttamente in un programma, ma un modello che può essere applicato per risolvere un problema in diverse situazioni.
		\subsection{Diagramma di Gantt}
		Grafico disposto su barre orizzontali che mostra l'andamento e le tempistiche di un progetto in base alle sue attività.
		\subsection{Diagramma UML}
		Unified Modeling Language (UML) è uno standard industriale basato sul paradigma orientato agli oggetti, allo scopo di agevolare la modellazione e progettazione di sistemi informatici.
		\subsection{Discord}
		Piattaforma digitale di VoIP (Voice Over Internet Protocol), dedicata alla messaggistica istantanea e alla distribuzione digitale di contenuti.
		\subsection{Docker}
		Piattaforma open source che consente di creare, implementare, eseguire, aggiornare e gestire componenti eseguibili standardizzati, noti come container.

	\clearpage
	\section{E}
		\subsection{Economicit\'a}
		\subsection{Efficacia}
		Capacità di raggiungere gli obiettivi prefissati.
		\subsection{Efficienza}
		Capacità di raggiungere gli obiettivi prefissati nel modo più rapido e con il minor utilizzo di risorse.
	\clearpage
	\section{F}
		\subsection{Fase}

		\subsection{Fornitore}
		Entità responsabile della fornitura di un software esistente o dello sviluppo di un software personalizzato.
		\subsection{Framework} % Utile??
		Architettura logica di supporto che fornisce una struttura coerente ed efficace per lo sviluppo di applicazioni

	\clearpage
	\section{G}
		\subsection{Gantt Project}
		 Software open source per la creazione di diagrammi di Gantt.
		\subsection{Git}
		Software di controllo di versione (VCS, Version Control System) open-source.


		\subsection{Github}
		Hosting service per progetti software che usa Git come sistema di controllo di versione. Offre funzionalità di issue tracking.
	\clearpage
	\section{H}
		\subsection{Hosting}
		Allocare su un server web un'applicazione web, rendendola così accessibile dalla rete Internet e ai suoi utenti.
		
	\clearpage
	\section{I}
		\subsection{IMAP}
		Internet Message Access Protocol, protocollo di posta elettronica che permette ad un client di accedere alle proprie email presenti su un server di posta.
		\subsection{ISO/IEC 12207:1997}
		Standard internazionale per il ciclo di vita del software. L'obiettivo di questo standard è fornire un quadro comune per lo sviluppo e la gestione del software, con una terminologia ben definita, che può essere utilizzata come riferimento dall'industria del software.

		\subsection{Issue}
		Funzionalità di GitHub per il tracciamento di problemi da risolvere, richieste di funzionalità e altre attività di progetto.
		

		\subsection{ITS (Issue Tracking System)}
		Software il cui compito è la registrazione e presa in carico delle issue da parte degli utenti.
	\clearpage
	\section{J}
		\subsection{Java}
		Linguaggio di programmazione ad alto livello, orientato agli oggetti e a tipizzazione statica, specificatamente progettato per essere il più possibile indipendente dalla piattaforma di esecuzione.
		\subsection{Jetty}
		Web server open source e servlet container scritto in Java.

		\subsection{JMAP}
		JSON Meta Application Protocol, è il successore del protocollo IMAP per la gestione delle email. JMAP utilizza JSON.

		\subsection{JVM}
		Java Virtual Machine, macchina virtuale che esegue programmi Java compilati in bytecode.
	\clearpage
	\section{L}
		\subsection{Latex}
		Linguaggio di markup per la preparazione di testi.
		\subsection{Latex workshop}
		Estensione di Visual Studio Code che permette di scrivere documenti in Latex.
		\subsection{Linux}
		Famiglia di sistemi operativi di tipo Unix-like, rilasciati sotto varie possibili distribuzioni, aventi la caratteristica comune di utilizzare come nucleo il kernel Linux.

	\clearpage
	\section{M}
		\subsection{Mantenibile}
		Facilità con cui un prodotto software può essere modificato e mantenute nel tempo.
		\subsection{Merge}
		Unione di due o più branch di un VCS, ad esempio GitHub.
		\subsection{Metrica}
		Strumento di misura per quantificare un aspetto di un processo o di un prodotto.
		\subsection{Milestone}
		Obiettivo che deve essere raggiunto entro una data prefissata.
		\subsection{Modello agile}
		Approccio allo sviluppo software che enfatizza la distribuzione incrementale, la collaborazione e l'apprendimento continuo.

		\subsection{Modello incrementale}
		Modello di sviluppo software che prevede la consegna di funzionalità in maniera incrementale, in modo da avere un prodotto finale completo e funzionante.

	\clearpage
	\section{N}
		\subsection{Norme}
		\subsection{Notion}
	\clearpage
	\section{O}
	\clearpage
	\section{P}
		\subsection{Package}
		\subsection{PDCA} % Citato??
		\subsection{POP3}
		\subsection{Precondizione}
		\subsection{Processo}
		\subsection{Product baseline}
		\subsection{PoC (Proof of Concept)}
		\subsection{Proponente}
		\subsection{Postcondizione}
		\subsection{Pull request}
	\clearpage
	\section{R}
		\subsection{Repository}
		\subsection{RTB (Requirement and Technology Baseline)}
		\subsection{Requisito}
		\subsection{Rollback}
		\subsection{Ruolo}
	\clearpage
	\section{S}
		\subsection{Server}
		\subsection{SMTP}
		\subsection{Stakeholder}
		\subsection{Standard}
		\subsection{Stato dell'arte}
	\clearpage
	\section{T}
		\subsection{Test}
		\subsection{TexLive}
		\subsection{Ticketing}
		\subsection{Tracciamento}
	\clearpage
	\section{U}
		\subsection{UML}
	\clearpage
	\section{V}
		\subsection{Versionamento}
	\clearpage
	\section{Z}	

\end{document}
%LTeX: language=it
% DOC TYPE E DEF QB SOFTWARE %%%%%%%%%%%%%%%%%%%%%%%%%
\documentclass[12pt]{article}
%%%%%%%%%%%%%%%%%%%% PACKAGES %%%%%%%%%%%%%%%%%%%%
\usepackage[english,italian]{babel}
\usepackage[a4paper, margin=3cm]{geometry}
\usepackage[T1]{fontenc}
\usepackage[utf8]{inputenc}
\usepackage[table]{xcolor}
\usepackage{amsmath}
\usepackage{graphicx}
\usepackage{float}
\usepackage{tabularx}
\usepackage{booktabs}
\usepackage{hyperref}
\usepackage{xcolor}
\usepackage{multicol}
\usepackage{multirow}
\usepackage{soul}
\usepackage{enumitem}
\usepackage{textcomp}
\usepackage{eurosym}
\usepackage{lastpage}
\usepackage{fancyhdr}
\usepackage{adjustbox}
\usepackage{subfiles}

%%%%%%%%%%%%%%%%%%%% COMMAND -> New commands %%%%%%%%%%%%%%%%%%%%
\newcommand{\mailtoQBS}
{
	\href{mailto:qbsoftware.swe@gmail.com}{qbsoftware.swe@gmail.com}
}

\let\oldpar\paragraph
\renewcommand{\paragraph}[1]{\oldpar{#1}\mbox{}\\}

%%%%%%%%%%%%%%%%%%%% COMMAND -> Redefine LaTeX Commands %%%%%%%%%%%%%%%%%%%%
\def\title#1{\gdef\THETITLE{#1}}
\def\date#1{\gdef\THEDATE{#1}}
\def\footname#1{\gdef\THEFOOTNAME{#1}}

\let\oldtexteuro\texteuro
\renewcommand{\texteuro}{\euro}

%%%%%%%%%%%%%%%%%%%% STYLES -> Commands %%%%%%%%%%%%%%%%%%%%
\newcommand{\makefirstpage}
{
	\begin{titlepage}
		
		% Defines a new command for the horizontal lines, change thickness here
		\newcommand{\HRule}{\rule{\linewidth}{0.2mm}} 
		
		\center % Center everything on the page
		
		
		%	Heading Sections
		
		\textsc{\LARGE QB Software}\\[.1cm] 
		\includegraphics[scale=.15]{imgs/qb-software-logo.png}\\[-.1 cm] 
		\includegraphics[scale=.025]{imgs/x.png}\\[0.5cm]
		\includegraphics[scale=.3]{imgs/unipd_logo.png}\\[.5cm]
		\textsc{\Large Università degli studi di Padova}\\[0.5cm] 
		\textsc{\large corso di ingegneria del software }\\[0.5cm]
		\textsc{\large anno accademico 2023/2024 }\\[0.5cm]
		
		
		%	Title section and date section
		
		\ifdefined\THEDATE
		\HRule \\[0.4cm]
		{ \huge{ \bfseries {\THETITLE}} \\ [.5cm]
			\THEDATE}\\[0.4cm] 
		\HRule \\[0.4cm]
		\else
		\HRule \\[0.4cm]
		{ \huge{ \bfseries {\THETITLE}} } \\
		\HRule \\[0.4cm]
		\fi
		%
		\textsc{Contatti:} \mailtoQBS\\[0.3cm]
		
		\vfill 
		
	\end{titlepage}
}

%%%%%%%%%%%%%%%%%%%% ACCESSIBILITY %%%%%%%%%%%%%%%%%%%%
\let\oldhref\href
\renewcommand{\href}[2]{\oldhref{#1}{\textcolor{blue}{\ul{#2}}}}

\hypersetup{
	colorlinks = true,
	linkcolor = cyan,
}

%%%%%%%%%%%%%%%%%%%% STYLES -> Settings %%%%%%%%%%%%%%%%%%%%
% Enable header and footer style
\pagestyle{fancy}
\thispagestyle{empty}
\thispagestyle{fancy}
\pagestyle{fancy}

% Header
\setlength{\topmargin}{-40pt}
\setlength{\headsep}{60pt}
\fancyhf{}
\lhead{QB Software}
\setlength{\headheight}{15pt}
\rhead{\includegraphics[width=1cm]{imgs/qb-software-logo.png}}

% Footer 
\fancyfoot{}
\fancyfoot[L]{\THEFOOTNAME}
\fancyfoot[R]{Pagina \thepage~di~\pageref{LastPage}}
\futurelet\TMPfootrule\def\footrule{\TMPfootrule}
\setcounter{page}{0}
\pagenumbering{arabic}
\renewcommand{\footrulewidth}{0.3pt}

%%%%%%%%%%%%%%%%%%%% ENVIRONMENT %%%%%%%%%%%%%%%%%%%%
% Remove colorless padding in booktabs
\setlength{\aboverulesep}{0cm}
\setlength{\belowrulesep}{0cm}
\setlength{\extrarowheight}{.75ex}

\newenvironment{todo}{
	\rowcolors{2}{cyan!80!black!30!}{cyan!80!black!20!}
	\begin{tabular}{p{3.48cm}>{\raggedright\arraybackslash}p{4cm}>{\raggedright\arraybackslash}p{6.5cm}}
		\toprule
		\rowcolor{gray!20} \textbf{ID}	& \textbf{Interessato} & \textbf{Task} 
		\\\midrule
	}{
		\bottomrule
	\end{tabular}
}

\newenvironment{changelog}{
	\noindent
	{\Large \textbf{Registro delle modifiche}}
	\noindent
	\begin{table}[h]
		\rowcolors{2}{cyan!80!black!30!}{cyan!80!black!20!}
		\begin{adjustbox}{width=\textwidth}
			\begin{tabular}{|c|c|p{2.35cm}|c|p{3.3cm}|}
				\hline
				\rowcolor{gray!20}
				\textbf{V.} & \textbf{Data} & \textbf{Membro} & \textbf{Ruolo} & \textbf{Descrizione} \\
				
				\hline
			}{
				\hline
			\end{tabular}
		\end{adjustbox}
	\end{table}
	
	\clearpage
}

% 1   2    3        4             5            6            7          
% Ver Data Relatore RuoloRelatore DataVerifica Verificatore Descrizione
\newcommand{\newlog}[7]{
	#1 & #5 & #6 & Verificatore & Controllo qualità \\
	   & #2 & #3 & #4           & #7 \\\hline
}

\newcommand{\milestone}[3]{
	#1 & #2 & #3 & Responsabile & Approvazione \par documento
	\\\hline
}

% INFORMAZIONI DOCUMENTO %%%%%%%%%%%%%%%%%%%%%%%%%
\title{Verbale interno}
\date{del 04/12/2023}
\footname{Verbale interno del 04/12/2023}

\begin{document}
	% PRIME PAGINE %%%%%%%%%%%%%%%%%%%%%%%%%
	\makefirstpage
	
	% 1   2    3        4             5            6            7          
% Ver Data Relatore RuoloRelatore DataVerifica Verificatore Descrizione
\begin{changelog}
	\newlog{0.1.0}{12/12/2023}{A. Feltrin}{Responsabile}{16/12/2023}{A. Domuta}{Stesura verbale}
\end{changelog}
	\clearpage
	
	\tableofcontents
	\clearpage

    \section{Informazioni generali}
	
	\subsection{Luogo e data dell'incontro}
	
    	\begin{itemize}
    		\item \textbf{Luogo}: meeting su Discord
    		\item \textbf{Data}: 04/12/2023
    		\item \textbf{Ora di inizio}: 15:30
    		\item \textbf{Ora di fine}: 17:30
    	\end{itemize}
	
	\subsection{Presenze}
	
    	\begin{itemize}
    		\item \textbf{Totale presenze}:
    		\begin{itemize}
    			\item Bustreo Alessandro
    			\item Destro Stefano
    			\item Domuta Alessia 
    			\item Feltrin Alessandro 
    			\item Fontana Raffaele Paolo 
    			\item Giurisato Andrea 
    			\item Rovea Silvia
    		\end{itemize}
    		
    		\item \textbf{Totale assenze}: 0
    		
    		\item \textbf{Partecipanti esterni}:
    		\begin{itemize}
    			\item nessuno
    		\end{itemize}
    	\end{itemize}

    \section{Ordine del giorno}
        Questioni fissate nella riunione interna precedente:
    	\begin{itemize}
    		\item terminare la discussione della struttura e dei contenuti da inserire nel documento Piano di Qualifica;
			\item pianificare le singole attività per l'inizio del prossimo Sprint.
    	\end{itemize}
    	%
    	Nuove questioni:
    	\begin{itemize}
    		\item retrospettiva sullo Sprint appena concluso.
    	\end{itemize}
    
    \section{Verbale}

	In seguito a una discussione sul resoconto del periodo precedente, il gruppo ha sviluppato la seguente retrospettiva:
		\begin{table}[H]
			\begin{tabularx}{\textwidth}{X|X}
				\hline
				\multicolumn{1}{|>{\centering\arraybackslash}X|}{\textbf{Keep doing}}
				&
				\multicolumn{1}{>{\centering\arraybackslash}X|}{\textbf{Improvements}}
				\\\hline\hline
				Il gruppo ha raggiunto gli obiettivi del primo Sprint
				&
				Rendere più atomiche le issue su GitHub e assegnarle a una singola persona
				\\\arrayrulecolor{gray}\hline
				&
				Verificatori devono utilizzare gli strumenti forniti da GitHub per comunicare con chi stende un documento
				\\\arrayrulecolor{gray}\hline
				&
				Migliorare il sistema con cui vengono assegnati i documenti da verificare ai verificatori 
				\\\arrayrulecolor{gray}\hline
				&
				Il sistema di automazione può introdurre ulteriori funzionalità per rimuovere il carico di lavoro che può essere svolto automaticamente
				\\\arrayrulecolor{gray}\hline
				&
				L'Analisi dei Requisiti va sistemata basandosi sulla discussione avvenuta durante il \emph{pit stop: analisi dei requisiti "per davvero"}.
			\end{tabularx}
			\caption{retrospettiva del 04/12/2023.}
		\end{table}
		\noindent\\
		Il gruppo, dopo la retrospettiva, ha discusso degli obbiettivi da raggiungere per il prossimo Sprint. In particolare, il gruppo ha deciso di:
		\begin{itemize}
			\item proseguire con lo studio delle tecnologie da utilizzare per la realizzazione del PoC;
			\item modificare i casi d'uso;
			\item continuare con le NdP, in particolare espandere il processo di gestione organizzativa e stendere quello relativa alla fornitura;
			\item correggere e migliorare l'Analisi dei Requisiti;
			\item continuare con la stesura del Glossario.
		\end{itemize}
		\noindent\\
		Il team ha deciso di posticipare la stesura del Piano di Qualifica, in quanto non ritenuto prioritario per il momento.

		\noindent
		Il team ha deciso di richiedere una riunione con il proponente per la prossima settimana, in modo da chiarire alcuni dubbi riguardanti lo stack tecnologico.
    \section{Azioni da intraprendere}
    
        \begin{todo}
            \hline
			\href{https://github.com/QB-Software-swe/docs/issues/44}{VI-2023-12-04-\#44}
            &
            R. Fontana
            &
            Manutenzione script di automazione
            \\\hline
			\href{https://github.com/QB-Software-swe/docs/issues/45}{VI-2023-12-04-\#45}
            &
            S. Rovea
            &
            Espandere gestione organizzativa nel file NdP
            \\\hline
            \href{https://github.com/QB-Software-swe/docs/issues/46}{VI-2023-12-04-\#46}
            &
            A. Feltrin
            &
            Scrivere processo di fornitura nel file NdP
			\\\hline
			\href{https://github.com/QB-Software-swe/docs/issues/47}{VI-2023-12-04-\#47}
            &
            S. Rovea
            &
            Correzione e continuazione casi d'uso relative alle email
			\\\hline
			\href{https://github.com/QB-Software-swe/docs/issues/48}{VI-2023-12-04-\#48}
            &
            A. Giurisato
            &
            Correzione e continuazione casi d'uso relative alle cartelle
			\\\hline
			\href{https://github.com/QB-Software-swe/docs/issues/49}{VI-2023-12-04-\#49}
            &
            A. Bustreo
            &
            Correzione e continuazione casi d'uso facoltativi
			\\\hline
			\href{https://github.com/QB-Software-swe/docs/issues/50}{VI-2023-12-04-\#50}
            &
            A. Feltrin
            &
            Proseguire con stesura del Glossario
			\\\hline
			\href{https://github.com/QB-Software-swe/docs/issues/51}{VI-2023-12-04-\#51}
            &
            S. Destro
            &
            Inizio configurazione Maven per PoC
			\\\hline
			\href{https://github.com/QB-Software-swe/docs/issues/52}{VI-2023-12-04-\#52}
            &
            A. Domuta
            &
            Inizio configurazione Jetty per PoC
			\\\hline
			\href{https://github.com/QB-Software-swe/docs/issues/53}{VI-2023-12-04-\#53}
            &
            R. Fontana
            &
            Inizio configurazione RESTeasy per PoC
			\\
    	\end{todo}
    
    \section{Ordine del giorno per la prossima riunione interna}
        \begin{itemize}
			\item Pianificare le singole attività per l'inizio del prossimo Sprint.
    	\end{itemize}
\end{document}
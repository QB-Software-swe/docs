% DOC TYPE E DEF QB SOFTWARE %%%%%%%%%%%%%%%%%%%%%%%%%
\documentclass[12pt]{article}
%%%%%%%%%%%%%%%%%%%% PACKAGES %%%%%%%%%%%%%%%%%%%%
\usepackage[english,italian]{babel}
\usepackage[a4paper, margin=3cm]{geometry}
\usepackage[T1]{fontenc}
\usepackage[utf8]{inputenc}
\usepackage[table]{xcolor}
\usepackage{amsmath}
\usepackage{graphicx}
\usepackage{float}
\usepackage{tabularx}
\usepackage{booktabs}
\usepackage{hyperref}
\usepackage{xcolor}
\usepackage{multicol}
\usepackage{multirow}
\usepackage{soul}
\usepackage{enumitem}
\usepackage{textcomp}
\usepackage{eurosym}
\usepackage{lastpage}
\usepackage{fancyhdr}
\usepackage{adjustbox}
\usepackage{subfiles}

%%%%%%%%%%%%%%%%%%%% COMMAND -> New commands %%%%%%%%%%%%%%%%%%%%
\newcommand{\mailtoQBS}
{
	\href{mailto:qbsoftware.swe@gmail.com}{qbsoftware.swe@gmail.com}
}

\let\oldpar\paragraph
\renewcommand{\paragraph}[1]{\oldpar{#1}\mbox{}\\}

%%%%%%%%%%%%%%%%%%%% COMMAND -> Redefine LaTeX Commands %%%%%%%%%%%%%%%%%%%%
\def\title#1{\gdef\THETITLE{#1}}
\def\date#1{\gdef\THEDATE{#1}}
\def\footname#1{\gdef\THEFOOTNAME{#1}}

\let\oldtexteuro\texteuro
\renewcommand{\texteuro}{\euro}

%%%%%%%%%%%%%%%%%%%% STYLES -> Commands %%%%%%%%%%%%%%%%%%%%
\newcommand{\makefirstpage}
{
	\begin{titlepage}
		
		% Defines a new command for the horizontal lines, change thickness here
		\newcommand{\HRule}{\rule{\linewidth}{0.2mm}} 
		
		\center % Center everything on the page
		
		
		%	Heading Sections
		
		\textsc{\LARGE QB Software}\\[.1cm] 
		\includegraphics[scale=.15]{imgs/qb-software-logo.png}\\[-.1 cm] 
		\includegraphics[scale=.025]{imgs/x.png}\\[0.5cm]
		\includegraphics[scale=.3]{imgs/unipd_logo.png}\\[.5cm]
		\textsc{\Large Università degli studi di Padova}\\[0.5cm] 
		\textsc{\large corso di ingegneria del software }\\[0.5cm]
		\textsc{\large anno accademico 2023/2024 }\\[0.5cm]
		
		
		%	Title section and date section
		
		\ifdefined\THEDATE
		\HRule \\[0.4cm]
		{ \huge{ \bfseries {\THETITLE}} \\ [.5cm]
			\THEDATE}\\[0.4cm] 
		\HRule \\[0.4cm]
		\else
		\HRule \\[0.4cm]
		{ \huge{ \bfseries {\THETITLE}} } \\
		\HRule \\[0.4cm]
		\fi
		%
		\textsc{Contatti:} \mailtoQBS\\[0.3cm]
		
		\vfill 
		
	\end{titlepage}
}

%%%%%%%%%%%%%%%%%%%% ACCESSIBILITY %%%%%%%%%%%%%%%%%%%%
\let\oldhref\href
\renewcommand{\href}[2]{\oldhref{#1}{\textcolor{blue}{\ul{#2}}}}

\hypersetup{
	colorlinks = true,
	linkcolor = cyan,
}

%%%%%%%%%%%%%%%%%%%% STYLES -> Settings %%%%%%%%%%%%%%%%%%%%
% Enable header and footer style
\pagestyle{fancy}
\thispagestyle{empty}
\thispagestyle{fancy}
\pagestyle{fancy}

% Header
\setlength{\topmargin}{-40pt}
\setlength{\headsep}{60pt}
\fancyhf{}
\lhead{QB Software}
\setlength{\headheight}{15pt}
\rhead{\includegraphics[width=1cm]{imgs/qb-software-logo.png}}

% Footer 
\fancyfoot{}
\fancyfoot[L]{\THEFOOTNAME}
\fancyfoot[R]{Pagina \thepage~di~\pageref{LastPage}}
\futurelet\TMPfootrule\def\footrule{\TMPfootrule}
\setcounter{page}{0}
\pagenumbering{arabic}
\renewcommand{\footrulewidth}{0.3pt}

%%%%%%%%%%%%%%%%%%%% ENVIRONMENT %%%%%%%%%%%%%%%%%%%%
% Remove colorless padding in booktabs
\setlength{\aboverulesep}{0cm}
\setlength{\belowrulesep}{0cm}
\setlength{\extrarowheight}{.75ex}

\newenvironment{todo}{
	\rowcolors{2}{cyan!80!black!30!}{cyan!80!black!20!}
	\begin{tabular}{p{3.48cm}>{\raggedright\arraybackslash}p{4cm}>{\raggedright\arraybackslash}p{6.5cm}}
		\toprule
		\rowcolor{gray!20} \textbf{ID}	& \textbf{Interessato} & \textbf{Task} 
		\\\midrule
	}{
		\bottomrule
	\end{tabular}
}

\newenvironment{changelog}{
	\noindent
	{\Large \textbf{Registro delle modifiche}}
	\noindent
	\begin{table}[h]
		\rowcolors{2}{cyan!80!black!30!}{cyan!80!black!20!}
		\begin{adjustbox}{width=\textwidth}
			\begin{tabular}{|c|c|p{2.35cm}|c|p{3.3cm}|}
				\hline
				\rowcolor{gray!20}
				\textbf{V.} & \textbf{Data} & \textbf{Membro} & \textbf{Ruolo} & \textbf{Descrizione} \\
				
				\hline
			}{
				\hline
			\end{tabular}
		\end{adjustbox}
	\end{table}
	
	\clearpage
}

% 1   2    3        4             5            6            7          
% Ver Data Relatore RuoloRelatore DataVerifica Verificatore Descrizione
\newcommand{\newlog}[7]{
	#1 & #5 & #6 & Verificatore & Controllo qualità \\
	   & #2 & #3 & #4           & #7 \\\hline
}

\newcommand{\milestone}[3]{
	#1 & #2 & #3 & Responsabile & Approvazione \par documento
	\\\hline
}

% INFORMAZIONI DOCUMENTO %%%%%%%%%%%%%%%%%%%%%%%%%
\title{Verbale interno}
\date{del 19 Ottobre 2023}
\footname{Verbale interno del 19/10/2023}

\begin{document}
	% PRIME PAGINE %%%%%%%%%%%%%%%%%%%%%%%%%
	\makefirstpage
	
	% 1   2    3        4             5            6            7          
% Ver Data Relatore RuoloRelatore DataVerifica Verificatore Descrizione
\begin{changelog}
	\newlog{0.1.0}{12/12/2023}{A. Feltrin}{Responsabile}{16/12/2023}{A. Domuta}{Stesura verbale}
\end{changelog}
	\clearpage
	
	\tableofcontents
	\clearpage

    \section{Informazioni generali}
    
	    \subsection{Luogo e data dell'incontro}
	    
		    \begin{itemize}
		    	\item \textbf{Luogo}: meeting su Discord
		    	\item \textbf{Data}: 19/10/2023
		    	\item \textbf{Ora di inizio}: 17:00
		    	\item \textbf{Ora di fine}: 19:00
		    \end{itemize}
    
	    \subsection{Presenze}
	    
		    \begin{itemize}
		    	\item \textbf{Totale presenze}:
		    	\begin{itemize}
		    		\item Bustreo Alessandro
		    		\item Destro Stefano
		    		\item Domuta Alessia 
		    		\item Feltrin Alessandro 
		    		\item Fontana Raffaele Paolo 
		    		\item Giurisato Andrea 
		    		\item Rovea Silvia
		    	\end{itemize}
		    	
		    	\item \textbf{Totale assenze}: 0
		    	
		    	\item \textbf{Partecipanti esterni}:
		    	\begin{itemize}
		    		\item nessuno
		    	\end{itemize}
		    \end{itemize}
    
    \section{Ordine del giorno}
	    Questioni fissate nella riunione interna precedente:
	    \begin{itemize}
	    	\item discussione e scelta dei capitolati;
	    	\item continuare lo sviluppo del way of working:
	    	\begin{itemize}
	    		\item strumenti da utilizzare per redigere i documenti;
	    		\item software da usare per tenere traccia e valutare la qualità del lavoro.
	    	\end{itemize}
	    \end{itemize}
	    %
	    Nuove questioni:
	    \begin{itemize}
	    	\item nessuna.
	    \end{itemize}
    
    \section{Verbale}
	    Si è tenuta una discussione in merito ai capitolati, durante la quale è stato
	    predisposto un foglio di Google condiviso tra il team, dove ogni membro del gruppo era tenuto ad esprimere le proprie preferenze in merito agli appalti. Ogni capitolato è stato oggetto di una discussione dove sorgevano dei fattori positivi e delle criticità, nel mentre ogni componente del gruppo poteva scegliere se modificare la propria preferenza. Alla fine di questo processo sono stati selezionati i 3 appalti che avevano ottenuto un maggiore consenso da parte del gruppo.
	    Gli appalti scelti sono: C8, C5 e C2, in questo ordine di preferenza. Alcune
	    domande sono sorte durante la discussione del C8, per questo si è deciso di
	    contattare l'azienda proponente Zextras per dei chiarimenti.
	    
	    Nella seconda parte della riunione si è continuato con la strutturazione di
	    un way of working, dove sono state prese le seguenti decisioni:
	    \begin{itemize}
	    	\item provare ad utilizzare Notion come strumento di organizzazione interna;
	    	\item creare un account GitHub;
	    	\item utilizzare GitFlow come flusso di lavoro per git;
	    	\item usare \LaTeX\ come strumento per redigere i documenti e Overleaf come editor;
	    	\item implementare due template \LaTeX, uno per le riunioni interne, seguendo la struttura descritta nel verbale del 16/10/2023, e un altro per le riunioni esterne.
	    \end{itemize}
	    È stato proposto di utilizzare GitHub Projects come strumento per valutare la qualità
	    dell'operato, la scelta definitiva è stata rinviata alla prossima riunione per valutare
	    altre possibili alternative (esempio: Jira e Trello).
	    
	    \section{Azioni da intraprendere}
	    \begin{todo}
	    	\href{https://github.com/QB-Software-swe/docs/issues/11}{VI-2023-10-19-\#11} 
	    	&
	    	Silvia Rovea
	    	&
	    	Scrivere una e-mail per prendere appuntamento con l'azienda Zextras
	    	\\\midrule
	    	-
	    	&
	    	Alessandro Bustreo
	    	&
	    	Imparare ad utilizzare Notion e creare l'account di GitHub
	    	\\\midrule
	    	-
	    	&
	    	Alessandro Feltrin
	    	&
	    	Imparare le basi di: GitHub, Git e GitFlow
	    	\\\midrule
	    	\href{https://github.com/QB-Software-swe/docs/issues/9}{VI-2023-10-19-\#9} 
	    	&
	    	Fontana Raffaele Paolo 
	    	&
	    	Creare il template in \LaTeX\, per le riunioni
	    	\\\midrule
	    	-
	    	&
	    	\emph{Tutto il gruppo}
	    	&
	    	Rileggersi i capitolati scelti. Valutare la possibilità di utilizzare altri servizi per il controllo della qualità
	    	\\
	    \end{todo}
    
    \section{Ordine del giorno per la prossima riunione interna}
	    \begin{itemize}
	    	\item Discutere le domande da proporre a Zextras e preparasi per il meeting; 
	    	\item continuare lo sviluppo del \emph{way of working}:
	    	\begin{itemize}
	    		\item prendere la scelta definitiva sul software da utilizzare per tenere traccia e valutare la qualità del lavoro;
	    		\item decidere se Notion è adatto come strumento di organizzazione interna;
	    	\end{itemize}
	    	\item preparare la divisione dei ruoli per la scrittura dei documenti di candidatura;
	    	\item iniziare a strutturare la divisione dei ruoli e le ore assegnate.
	    \end{itemize}
\end{document}
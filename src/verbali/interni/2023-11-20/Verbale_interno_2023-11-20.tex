%LTeX: language=it
% DOC TYPE E DEF QB SOFTWARE %%%%%%%%%%%%%%%%%%%%%%%%%
\documentclass[12pt]{article}
%%%%%%%%%%%%%%%%%%%% PACKAGES %%%%%%%%%%%%%%%%%%%%
\usepackage[english,italian]{babel}
\usepackage[a4paper, margin=3cm]{geometry}
\usepackage[T1]{fontenc}
\usepackage[utf8]{inputenc}
\usepackage[table]{xcolor}
\usepackage{amsmath}
\usepackage{graphicx}
\usepackage{float}
\usepackage{tabularx}
\usepackage{booktabs}
\usepackage{hyperref}
\usepackage{xcolor}
\usepackage{multicol}
\usepackage{multirow}
\usepackage{soul}
\usepackage{enumitem}
\usepackage{textcomp}
\usepackage{eurosym}
\usepackage{lastpage}
\usepackage{fancyhdr}
\usepackage{adjustbox}
\usepackage{subfiles}

%%%%%%%%%%%%%%%%%%%% COMMAND -> New commands %%%%%%%%%%%%%%%%%%%%
\newcommand{\mailtoQBS}
{
	\href{mailto:qbsoftware.swe@gmail.com}{qbsoftware.swe@gmail.com}
}

\let\oldpar\paragraph
\renewcommand{\paragraph}[1]{\oldpar{#1}\mbox{}\\}

%%%%%%%%%%%%%%%%%%%% COMMAND -> Redefine LaTeX Commands %%%%%%%%%%%%%%%%%%%%
\def\title#1{\gdef\THETITLE{#1}}
\def\date#1{\gdef\THEDATE{#1}}
\def\footname#1{\gdef\THEFOOTNAME{#1}}

\let\oldtexteuro\texteuro
\renewcommand{\texteuro}{\euro}

%%%%%%%%%%%%%%%%%%%% STYLES -> Commands %%%%%%%%%%%%%%%%%%%%
\newcommand{\makefirstpage}
{
	\begin{titlepage}
		
		% Defines a new command for the horizontal lines, change thickness here
		\newcommand{\HRule}{\rule{\linewidth}{0.2mm}} 
		
		\center % Center everything on the page
		
		
		%	Heading Sections
		
		\textsc{\LARGE QB Software}\\[.1cm] 
		\includegraphics[scale=.15]{imgs/qb-software-logo.png}\\[-.1 cm] 
		\includegraphics[scale=.025]{imgs/x.png}\\[0.5cm]
		\includegraphics[scale=.3]{imgs/unipd_logo.png}\\[.5cm]
		\textsc{\Large Università degli studi di Padova}\\[0.5cm] 
		\textsc{\large corso di ingegneria del software }\\[0.5cm]
		\textsc{\large anno accademico 2023/2024 }\\[0.5cm]
		
		
		%	Title section and date section
		
		\ifdefined\THEDATE
		\HRule \\[0.4cm]
		{ \huge{ \bfseries {\THETITLE}} \\ [.5cm]
			\THEDATE}\\[0.4cm] 
		\HRule \\[0.4cm]
		\else
		\HRule \\[0.4cm]
		{ \huge{ \bfseries {\THETITLE}} } \\
		\HRule \\[0.4cm]
		\fi
		%
		\textsc{Contatti:} \mailtoQBS\\[0.3cm]
		
		\vfill 
		
	\end{titlepage}
}

%%%%%%%%%%%%%%%%%%%% ACCESSIBILITY %%%%%%%%%%%%%%%%%%%%
\let\oldhref\href
\renewcommand{\href}[2]{\oldhref{#1}{\textcolor{blue}{\ul{#2}}}}

\hypersetup{
	colorlinks = true,
	linkcolor = cyan,
}

%%%%%%%%%%%%%%%%%%%% STYLES -> Settings %%%%%%%%%%%%%%%%%%%%
% Enable header and footer style
\pagestyle{fancy}
\thispagestyle{empty}
\thispagestyle{fancy}
\pagestyle{fancy}

% Header
\setlength{\topmargin}{-40pt}
\setlength{\headsep}{60pt}
\fancyhf{}
\lhead{QB Software}
\setlength{\headheight}{15pt}
\rhead{\includegraphics[width=1cm]{imgs/qb-software-logo.png}}

% Footer 
\fancyfoot{}
\fancyfoot[L]{\THEFOOTNAME}
\fancyfoot[R]{Pagina \thepage~di~\pageref{LastPage}}
\futurelet\TMPfootrule\def\footrule{\TMPfootrule}
\setcounter{page}{0}
\pagenumbering{arabic}
\renewcommand{\footrulewidth}{0.3pt}

%%%%%%%%%%%%%%%%%%%% ENVIRONMENT %%%%%%%%%%%%%%%%%%%%
% Remove colorless padding in booktabs
\setlength{\aboverulesep}{0cm}
\setlength{\belowrulesep}{0cm}
\setlength{\extrarowheight}{.75ex}

\newenvironment{todo}{
	\rowcolors{2}{cyan!80!black!30!}{cyan!80!black!20!}
	\begin{tabular}{p{3.48cm}>{\raggedright\arraybackslash}p{4cm}>{\raggedright\arraybackslash}p{6.5cm}}
		\toprule
		\rowcolor{gray!20} \textbf{ID}	& \textbf{Interessato} & \textbf{Task} 
		\\\midrule
	}{
		\bottomrule
	\end{tabular}
}

\newenvironment{changelog}{
	\noindent
	{\Large \textbf{Registro delle modifiche}}
	\noindent
	\begin{table}[h]
		\rowcolors{2}{cyan!80!black!30!}{cyan!80!black!20!}
		\begin{adjustbox}{width=\textwidth}
			\begin{tabular}{|c|c|p{2.35cm}|c|p{3.3cm}|}
				\hline
				\rowcolor{gray!20}
				\textbf{V.} & \textbf{Data} & \textbf{Membro} & \textbf{Ruolo} & \textbf{Descrizione} \\
				
				\hline
			}{
				\hline
			\end{tabular}
		\end{adjustbox}
	\end{table}
	
	\clearpage
}

% 1   2    3        4             5            6            7          
% Ver Data Relatore RuoloRelatore DataVerifica Verificatore Descrizione
\newcommand{\newlog}[7]{
	#1 & #5 & #6 & Verificatore & Controllo qualità \\
	   & #2 & #3 & #4           & #7 \\\hline
}

\newcommand{\milestone}[3]{
	#1 & #2 & #3 & Responsabile & Approvazione \par documento
	\\\hline
}

% INFORMAZIONI DOCUMENTO %%%%%%%%%%%%%%%%%%%%%%%%%
\title{Verbale interno}
\date{del 20/11/2023}
\footname{Verbale interno del 20/11/2023}

\begin{document}
	% PRIME PAGINE %%%%%%%%%%%%%%%%%%%%%%%%%
	\makefirstpage
	
	% 1   2    3        4             5            6            7          
% Ver Data Relatore RuoloRelatore DataVerifica Verificatore Descrizione
\begin{changelog}
	\newlog{0.1.0}{12/12/2023}{A. Feltrin}{Responsabile}{16/12/2023}{A. Domuta}{Stesura verbale}
\end{changelog}
	\clearpage
	
	\tableofcontents
	\clearpage

    \section{Informazioni generali}
    
    \subsection{Luogo e data dell'incontro}
    
    \begin{itemize}
    	\item \textbf{Luogo}: meeting su Discord
    	\item \textbf{Data}: 20/11/2023
    	\item \textbf{Ora di inizio}: 16:00
    	\item \textbf{Ora di fine}: 17:00
    \end{itemize}
    
    \subsection{Presenze}
    
    \begin{itemize}
    	\item \textbf{Totale presenze}:
    	\begin{itemize}
    		\item Bustreo Alessandro
    		\item Destro Stefano
    		\item Domuta Alessia 
    		\item Feltrin Alessandro 
    		\item Fontana Raffaele Paolo 
    		\item Giurisato Andrea 
    		\item Rovea Silvia
    	\end{itemize}
    	
    	\item \textbf{Totale assenze}: 0
    	
    	\item \textbf{Partecipanti esterni}:
    	\begin{itemize}
    		\item nessuno.
    	\end{itemize}
    \end{itemize}
    
    \section{Ordine del giorno}
    Questioni fissate nella riunione interna precedente:
    \begin{itemize}
    	\item Sviluppare altri processi del way of working oltre a quelli normati nelle norme
    	di progetto;
    	\item aggiornare il gruppo sullo stato dell’analisi dei requisiti.
    \end{itemize}
    
    Nuove questioni:
    \begin{itemize}
    	\item sistema di automazione.
    \end{itemize}
    
    \section{Verbale}
    Il gruppo ha discusso dello stato di avanzamento dei lavori, sono stati mostrati all'intero gruppo i casi d'uso finora sviluppati e le motivazione di quelle decisioni. Il team di QB Software ha deciso la struttura e i contenuti del documento riguardante l'AdR:
    \begin{itemize}
    	\item introduzione,
    	\item scopo/descrizione del prodotto,
    	\item i casi d'uso individuati:
    	\begin{itemize}
    		\item introduzione e spiegazione della sintassi usata;
    		\item casi d'uso, ogni caso d'uso dovrà essere riportato seguendo il \href{https://github.com/rcardin/swe-imdb/blob/main/use-cases/template.md}{template proposto dal prof. Cardin} (Online, ultima visita 20/11/2023).
    	\end{itemize}
    \end{itemize}
    \noindent
    Il gruppo poi ha discusso il sistema di automazione per il way of working della documentazione. È stato proposto di creare uno script in Python che utilizza un file configurazione CSV: \verb|config.csv|, e le label attribuite alla issue per fare la build in automatico del documento aggiornando la versione in base alla tipologia di modifica fatta.
     
    \section{Azioni da intraprendere}
    \begin{todo}
    	\href{https://github.com/QB-Software-swe/docs/issues/20}{VI-2023-10-26-\#20} 
    	& A. Bustreo
    	& Implementazione e testing del sistema di automazione 
    	\\\hline
    \end{todo}
    
    \section{Ordine del giorno per la prossima riunione interna}
    \begin{itemize}
    	\item nessuno.
    \end{itemize}
\end{document}
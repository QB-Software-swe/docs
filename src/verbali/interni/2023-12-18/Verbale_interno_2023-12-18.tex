% DOC TYPE E DEF QB SOFTWARE %%%%%%%%%%%%%%%%%%%%%%%%%
\documentclass[12pt]{article}
%%%%%%%%%%%%%%%%%%%% PACKAGES %%%%%%%%%%%%%%%%%%%%
\usepackage[english,italian]{babel}
\usepackage[a4paper, margin=3cm]{geometry}
\usepackage[T1]{fontenc}
\usepackage[utf8]{inputenc}
\usepackage[table]{xcolor}
\usepackage{amsmath}
\usepackage{graphicx}
\usepackage{float}
\usepackage{tabularx}
\usepackage{booktabs}
\usepackage{hyperref}
\usepackage{xcolor}
\usepackage{multicol}
\usepackage{multirow}
\usepackage{soul}
\usepackage{enumitem}
\usepackage{textcomp}
\usepackage{eurosym}
\usepackage{lastpage}
\usepackage{fancyhdr}
\usepackage{adjustbox}
\usepackage{subfiles}

%%%%%%%%%%%%%%%%%%%% COMMAND -> New commands %%%%%%%%%%%%%%%%%%%%
\newcommand{\mailtoQBS}
{
	\href{mailto:qbsoftware.swe@gmail.com}{qbsoftware.swe@gmail.com}
}

\let\oldpar\paragraph
\renewcommand{\paragraph}[1]{\oldpar{#1}\mbox{}\\}

%%%%%%%%%%%%%%%%%%%% COMMAND -> Redefine LaTeX Commands %%%%%%%%%%%%%%%%%%%%
\def\title#1{\gdef\THETITLE{#1}}
\def\date#1{\gdef\THEDATE{#1}}
\def\footname#1{\gdef\THEFOOTNAME{#1}}

\let\oldtexteuro\texteuro
\renewcommand{\texteuro}{\euro}

%%%%%%%%%%%%%%%%%%%% STYLES -> Commands %%%%%%%%%%%%%%%%%%%%
\newcommand{\makefirstpage}
{
	\begin{titlepage}
		
		% Defines a new command for the horizontal lines, change thickness here
		\newcommand{\HRule}{\rule{\linewidth}{0.2mm}} 
		
		\center % Center everything on the page
		
		
		%	Heading Sections
		
		\textsc{\LARGE QB Software}\\[.1cm] 
		\includegraphics[scale=.15]{imgs/qb-software-logo.png}\\[-.1 cm] 
		\includegraphics[scale=.025]{imgs/x.png}\\[0.5cm]
		\includegraphics[scale=.3]{imgs/unipd_logo.png}\\[.5cm]
		\textsc{\Large Università degli studi di Padova}\\[0.5cm] 
		\textsc{\large corso di ingegneria del software }\\[0.5cm]
		\textsc{\large anno accademico 2023/2024 }\\[0.5cm]
		
		
		%	Title section and date section
		
		\ifdefined\THEDATE
		\HRule \\[0.4cm]
		{ \huge{ \bfseries {\THETITLE}} \\ [.5cm]
			\THEDATE}\\[0.4cm] 
		\HRule \\[0.4cm]
		\else
		\HRule \\[0.4cm]
		{ \huge{ \bfseries {\THETITLE}} } \\
		\HRule \\[0.4cm]
		\fi
		%
		\textsc{Contatti:} \mailtoQBS\\[0.3cm]
		
		\vfill 
		
	\end{titlepage}
}

%%%%%%%%%%%%%%%%%%%% ACCESSIBILITY %%%%%%%%%%%%%%%%%%%%
\let\oldhref\href
\renewcommand{\href}[2]{\oldhref{#1}{\textcolor{blue}{\ul{#2}}}}

\hypersetup{
	colorlinks = true,
	linkcolor = cyan,
}

%%%%%%%%%%%%%%%%%%%% STYLES -> Settings %%%%%%%%%%%%%%%%%%%%
% Enable header and footer style
\pagestyle{fancy}
\thispagestyle{empty}
\thispagestyle{fancy}
\pagestyle{fancy}

% Header
\setlength{\topmargin}{-40pt}
\setlength{\headsep}{60pt}
\fancyhf{}
\lhead{QB Software}
\setlength{\headheight}{15pt}
\rhead{\includegraphics[width=1cm]{imgs/qb-software-logo.png}}

% Footer 
\fancyfoot{}
\fancyfoot[L]{\THEFOOTNAME}
\fancyfoot[R]{Pagina \thepage~di~\pageref{LastPage}}
\futurelet\TMPfootrule\def\footrule{\TMPfootrule}
\setcounter{page}{0}
\pagenumbering{arabic}
\renewcommand{\footrulewidth}{0.3pt}

%%%%%%%%%%%%%%%%%%%% ENVIRONMENT %%%%%%%%%%%%%%%%%%%%
% Remove colorless padding in booktabs
\setlength{\aboverulesep}{0cm}
\setlength{\belowrulesep}{0cm}
\setlength{\extrarowheight}{.75ex}

\newenvironment{todo}{
	\rowcolors{2}{cyan!80!black!30!}{cyan!80!black!20!}
	\begin{tabular}{p{3.48cm}>{\raggedright\arraybackslash}p{4cm}>{\raggedright\arraybackslash}p{6.5cm}}
		\toprule
		\rowcolor{gray!20} \textbf{ID}	& \textbf{Interessato} & \textbf{Task} 
		\\\midrule
	}{
		\bottomrule
	\end{tabular}
}

\newenvironment{changelog}{
	\noindent
	{\Large \textbf{Registro delle modifiche}}
	\noindent
	\begin{table}[h]
		\rowcolors{2}{cyan!80!black!30!}{cyan!80!black!20!}
		\begin{adjustbox}{width=\textwidth}
			\begin{tabular}{|c|c|p{2.35cm}|c|p{3.3cm}|}
				\hline
				\rowcolor{gray!20}
				\textbf{V.} & \textbf{Data} & \textbf{Membro} & \textbf{Ruolo} & \textbf{Descrizione} \\
				
				\hline
			}{
				\hline
			\end{tabular}
		\end{adjustbox}
	\end{table}
	
	\clearpage
}

% 1   2    3        4             5            6            7          
% Ver Data Relatore RuoloRelatore DataVerifica Verificatore Descrizione
\newcommand{\newlog}[7]{
	#1 & #5 & #6 & Verificatore & Controllo qualità \\
	   & #2 & #3 & #4           & #7 \\\hline
}

\newcommand{\milestone}[3]{
	#1 & #2 & #3 & Responsabile & Approvazione \par documento
	\\\hline
}

% INFORMAZIONI DOCUMENTO %%%%%%%%%%%%%%%%%%%%%%%%%
\title{Verbale interno}
\date{del 18/12/2023}
\footname{Verbale interno del 18/12/2023}

\begin{document}
	% PRIME PAGINE %%%%%%%%%%%%%%%%%%%%%%%%%
	\makefirstpage
	
	% 1   2    3        4             5            6            7          
% Ver Data Relatore RuoloRelatore DataVerifica Verificatore Descrizione
\begin{changelog}
	\newlog{0.1.0}{12/12/2023}{A. Feltrin}{Responsabile}{16/12/2023}{A. Domuta}{Stesura verbale}
\end{changelog}
	\clearpage
	
	\tableofcontents
	\clearpage

    \section{Informazioni generali}
	
	\subsection{Luogo e data dell'incontro}
	
    	\begin{itemize}
    		\item \textbf{Luogo}: meeting su Discord
    		\item \textbf{Data}: 18/12/2023
    		\item \textbf{Ora di inizio}: 16:00
    		\item \textbf{Ora di fine}: 18:00
    	\end{itemize}
	
	\subsection{Presenze}
	
    	\begin{itemize}
    		\item \textbf{Totale presenze}:
    		\begin{itemize}
    			\item Bustreo Alessandro
    			\item Destro Stefano
    			\item Domuta Alessia 
    			\item Feltrin Alessandro 
    			\item Fontana Raffaele Paolo 
    			\item Giurisato Andrea 
    			\item Rovea Silvia
    		\end{itemize}
    		
    		\item \textbf{Totale assenze}: 0
    		
    		\item \textbf{Partecipanti esterni}:
    		\begin{itemize}
    			\item nessuno
    		\end{itemize}
    	\end{itemize}

    \section{Ordine del giorno}
        Questioni fissate nella riunione interna precedente:
        \begin{itemize}
			\item pianificare le singole attività per l'inizio del prossimo Sprint.
    	\end{itemize}
    	%
    	Nuove questioni:
    	\begin{itemize}
    		\item retrospettiva dello Sprint passato (Sprint-2).
    	\end{itemize}
    
    \section{Verbale}

	\subsection{Retrospettiva}

	La riunione si è aperta con la retrospettiva relativa al secondo Sprint (dal 04/12/2023 al 17/12/2023).

	\subsubsection{Cosa ha funzionato}
	\begin{itemize}
		\item Gli obiettivi del secondo Sprint sono stati tutti raggiunti;
		\item sistema di automazione: il sistema di automazione, in seguito alle modifiche discusse durante la riunione del 12/04/2023, è stato migliorato con l'aggiunta della possibilità di controllare lo stato delle action prima di confermare una pull request. Questo si è rivelato molto utile e ci permette di evitare numerosi errori;
		\item quantità del carico di lavoro omogenea: la quantità di ore produttive di ogni membro del team è stata adeguata;
		\item 
	\end{itemize}
	
	\subsubsection{Cosa non ha funzionato}
	\begin{itemize}
		\item Distribuzione temporale del carico di lavoro dei Verificatori: il carico di lavoro dei Verificatori si è concentrato in particolare alla fine dello Sprint. Per il prossimo Sprint è necessario seguire meglio la pianificazione, così da terminare prima alcune issue e non concentrare il lavoro dei Verificatori a fine Sprint;
	\end{itemize}


	\subsubsection{Rischi}
	I rischi previsti durante la pianificazione del secondo sprint erano i seguenti: 
	\begin{itemize}
		\item 
	\end{itemize}
	I rischi effettivamente incontrati sono i seguenti:
	\begin{itemize}
		\item R: vista l'inesperienza con l'utilizzo delle tecnologie 
	\end{itemize}
	Il piano di mitigazioni si è rivelato efficace

	\subsubsection{Tabella riassuntiva}

	\begin{table}[H]
		\begin{tabularx}{\textwidth}{X|X}
			\hline
			\multicolumn{1}{|>{\centering\arraybackslash}X|}{\textbf{Keep doing}}
			&
			\multicolumn{1}{>{\centering\arraybackslash}X|}{\textbf{Improvements}}
			\\\hline\hline
			Obiettivi del secondo Sprint raggiunti
			&
			Distribuzione temporale del carico di lavoro dei Verificatori
			\\\arrayrulecolor{gray}\hline
			Quantità omogenea del carico di lavoro nel team
			&

			\\\arrayrulecolor{gray}\hline
			L'utilizzo degli strumenti per la review di GitHub
			&

			\\\arrayrulecolor{gray}\hline
			Sistema di automazione
			&

		\end{tabularx}
		\caption{retrospettiva del 18/12/2023.}
	\end{table}

	\subsection{Pianificazione}
	\subsubsection{Rischi}
	I rischi previsti sono i seguenti:
	\begin{itemize}
		\item RT: vista la mancanza di alcune fondamentali librerie per lo sviluppo del server con JMAP, si rende necessario estendere una delle librerie fornite per il client, questo comporta una una ridistribuzione delle risorse, come da piano di mitigazione
	\end{itemize}
	
    
    \section{Azioni da intraprendere}
    
        \begin{todo}
            VI-1970-01-01-01
            &
            -
            &
            -
            \\\midrule
            VI-1970-01-01-02
            &
            -
            &
            -
            \\\midrule
            VI-1970-01-01-03
            &
            -
            &
            -
            \\
    	\end{todo}
    
    \section{Ordine del giorno per la prossima riunione interna}
        \begin{itemize}
        		\item 
    	\end{itemize}
\end{document}
% LTeX: language=it
% DOC TYPE E DEF QB SOFTWARE %%%%%%%%%%%%%%%%%%%%%%%%%
\documentclass[12pt]{article}
%%%%%%%%%%%%%%%%%%%% PACKAGES %%%%%%%%%%%%%%%%%%%%
\usepackage[english,italian]{babel}
\usepackage[a4paper, margin=3cm]{geometry}
\usepackage[T1]{fontenc}
\usepackage[utf8]{inputenc}
\usepackage[table]{xcolor}
\usepackage{amsmath}
\usepackage{graphicx}
\usepackage{float}
\usepackage{tabularx}
\usepackage{booktabs}
\usepackage{hyperref}
\usepackage{xcolor}
\usepackage{multicol}
\usepackage{multirow}
\usepackage{soul}
\usepackage{enumitem}
\usepackage{textcomp}
\usepackage{eurosym}
\usepackage{lastpage}
\usepackage{fancyhdr}
\usepackage{adjustbox}
\usepackage{subfiles}

%%%%%%%%%%%%%%%%%%%% COMMAND -> New commands %%%%%%%%%%%%%%%%%%%%
\newcommand{\mailtoQBS}
{
	\href{mailto:qbsoftware.swe@gmail.com}{qbsoftware.swe@gmail.com}
}

\let\oldpar\paragraph
\renewcommand{\paragraph}[1]{\oldpar{#1}\mbox{}\\}

%%%%%%%%%%%%%%%%%%%% COMMAND -> Redefine LaTeX Commands %%%%%%%%%%%%%%%%%%%%
\def\title#1{\gdef\THETITLE{#1}}
\def\date#1{\gdef\THEDATE{#1}}
\def\footname#1{\gdef\THEFOOTNAME{#1}}

\let\oldtexteuro\texteuro
\renewcommand{\texteuro}{\euro}

%%%%%%%%%%%%%%%%%%%% STYLES -> Commands %%%%%%%%%%%%%%%%%%%%
\newcommand{\makefirstpage}
{
	\begin{titlepage}
		
		% Defines a new command for the horizontal lines, change thickness here
		\newcommand{\HRule}{\rule{\linewidth}{0.2mm}} 
		
		\center % Center everything on the page
		
		
		%	Heading Sections
		
		\textsc{\LARGE QB Software}\\[.1cm] 
		\includegraphics[scale=.15]{imgs/qb-software-logo.png}\\[-.1 cm] 
		\includegraphics[scale=.025]{imgs/x.png}\\[0.5cm]
		\includegraphics[scale=.3]{imgs/unipd_logo.png}\\[.5cm]
		\textsc{\Large Università degli studi di Padova}\\[0.5cm] 
		\textsc{\large corso di ingegneria del software }\\[0.5cm]
		\textsc{\large anno accademico 2023/2024 }\\[0.5cm]
		
		
		%	Title section and date section
		
		\ifdefined\THEDATE
		\HRule \\[0.4cm]
		{ \huge{ \bfseries {\THETITLE}} \\ [.5cm]
			\THEDATE}\\[0.4cm] 
		\HRule \\[0.4cm]
		\else
		\HRule \\[0.4cm]
		{ \huge{ \bfseries {\THETITLE}} } \\
		\HRule \\[0.4cm]
		\fi
		%
		\textsc{Contatti:} \mailtoQBS\\[0.3cm]
		
		\vfill 
		
	\end{titlepage}
}

%%%%%%%%%%%%%%%%%%%% ACCESSIBILITY %%%%%%%%%%%%%%%%%%%%
\let\oldhref\href
\renewcommand{\href}[2]{\oldhref{#1}{\textcolor{blue}{\ul{#2}}}}

\hypersetup{
	colorlinks = true,
	linkcolor = cyan,
}

%%%%%%%%%%%%%%%%%%%% STYLES -> Settings %%%%%%%%%%%%%%%%%%%%
% Enable header and footer style
\pagestyle{fancy}
\thispagestyle{empty}
\thispagestyle{fancy}
\pagestyle{fancy}

% Header
\setlength{\topmargin}{-40pt}
\setlength{\headsep}{60pt}
\fancyhf{}
\lhead{QB Software}
\setlength{\headheight}{15pt}
\rhead{\includegraphics[width=1cm]{imgs/qb-software-logo.png}}

% Footer 
\fancyfoot{}
\fancyfoot[L]{\THEFOOTNAME}
\fancyfoot[R]{Pagina \thepage~di~\pageref{LastPage}}
\futurelet\TMPfootrule\def\footrule{\TMPfootrule}
\setcounter{page}{0}
\pagenumbering{arabic}
\renewcommand{\footrulewidth}{0.3pt}

%%%%%%%%%%%%%%%%%%%% ENVIRONMENT %%%%%%%%%%%%%%%%%%%%
% Remove colorless padding in booktabs
\setlength{\aboverulesep}{0cm}
\setlength{\belowrulesep}{0cm}
\setlength{\extrarowheight}{.75ex}

\newenvironment{todo}{
	\rowcolors{2}{cyan!80!black!30!}{cyan!80!black!20!}
	\begin{tabular}{p{3.48cm}>{\raggedright\arraybackslash}p{4cm}>{\raggedright\arraybackslash}p{6.5cm}}
		\toprule
		\rowcolor{gray!20} \textbf{ID}	& \textbf{Interessato} & \textbf{Task} 
		\\\midrule
	}{
		\bottomrule
	\end{tabular}
}

\newenvironment{changelog}{
	\noindent
	{\Large \textbf{Registro delle modifiche}}
	\noindent
	\begin{table}[h]
		\rowcolors{2}{cyan!80!black!30!}{cyan!80!black!20!}
		\begin{adjustbox}{width=\textwidth}
			\begin{tabular}{|c|c|p{2.35cm}|c|p{3.3cm}|}
				\hline
				\rowcolor{gray!20}
				\textbf{V.} & \textbf{Data} & \textbf{Membro} & \textbf{Ruolo} & \textbf{Descrizione} \\
				
				\hline
			}{
				\hline
			\end{tabular}
		\end{adjustbox}
	\end{table}
	
	\clearpage
}

% 1   2    3        4             5            6            7          
% Ver Data Relatore RuoloRelatore DataVerifica Verificatore Descrizione
\newcommand{\newlog}[7]{
	#1 & #5 & #6 & Verificatore & Controllo qualità \\
	   & #2 & #3 & #4           & #7 \\\hline
}

\newcommand{\milestone}[3]{
	#1 & #2 & #3 & Responsabile & Approvazione \par documento
	\\\hline
}
\usepackage[italian]{babel}

% INFORMAZIONI DOCUMENTO %%%%%%%%%%%%%%%%%%%%%%%%%
\title{Verbale interno}
\date{del 06/11/2023}
\footname{Verbale interno del 06/11/2023}

\begin{document}
	% PRIME PAGINE %%%%%%%%%%%%%%%%%%%%%%%%%
	\makefirstpage
	
	% 1   2    3        4             5            6            7          
% Ver Data Relatore RuoloRelatore DataVerifica Verificatore Descrizione
\begin{changelog}
	\newlog{0.1.0}{12/12/2023}{A. Feltrin}{Responsabile}{16/12/2023}{A. Domuta}{Stesura verbale}
\end{changelog}
	\clearpage
	
	\tableofcontents
	\clearpage

    \section{Informazioni generali}
	
	\subsection{Luogo e data dell'incontro}
	
    	\begin{itemize}
    		\item \textbf{Luogo}: meeting su Discord
    		\item \textbf{Data}: 06/11/2023
    		\item \textbf{Ora di inizio}: 16:00
    		\item \textbf{Ora di fine}: 18:30
    	\end{itemize}
	
	\subsection{Presenze}
	
    	\begin{itemize}
    		\item \textbf{Totale presenze}:
    		\begin{itemize}
    			\item Bustreo Alessandro
    			\item Destro Stefano
    			\item Domuta Alessia 
    			\item Feltrin Alessandro 
    			\item Fontana Raffaele Paolo 
    			\item Giurisato Andrea 
    			\item Rovea Silvia
    		\end{itemize}
    		
    		\item \textbf{Totale assenze}: 0
    		
    		\item \textbf{Partecipanti esterni}:
    		\begin{itemize}
    			\item nessuno
    		\end{itemize}
    	\end{itemize}

    \section{Ordine del giorno}
        Questioni fissate nella riunione interna precedente:
    	\begin{itemize}
    		\item continuare la discussione del nostro way of working in seguito all'aggiudicazione degli appalti fissata per il 06/11/2023.
    	\end{itemize}
    	%
    	Nuove questioni:
    	\begin{itemize}
			\item retrospettiva basata sul feedback ricevuto durante l'aggiudicazione degli appalti;
    		\item discutere la struttura e i contenuti del documento Norme di Progetto.
    	\end{itemize}
    
    \section{Verbale}
		Basandosi sul feedback fornito durante l'aggiudicazione degli appalti del 06/11/2023, e in seguito a una discussione sul resoconto del periodo precedente, il gruppo ha sviluppato la seguente retrospettiva:
		\begin{table}[H]
			\begin{tabularx}{\textwidth}{X|X}
				\hline
				\multicolumn{1}{|>{\centering\arraybackslash}X|}{\textbf{Keep doing}}
				&
				\multicolumn{1}{>{\centering\arraybackslash}X|}{\textbf{Improvments}}
				\\\hline\hline
				Organizzazione repository
				&
				Sito web non semplifica veramente la navigazione della repository
				\\\arrayrulecolor{gray}\hline
				La struttura dei verbali e contenuto
				&
				Maggiore chiarezza delle scelte nei documenti per la candidatura
				\\\arrayrulecolor{gray}\hline
				Strumenti scelti finora (\LaTeX, Notion, Discord, eccetera). Tranne Overleaf
				&
				Gestione del registro delle modifiche e delle versioni errato a livello concettuale
				\\\arrayrulecolor{gray}\hline
				&
				Verbali esterni non approvati
				\\\arrayrulecolor{gray}\hline
				&
				Nome del file di ogni documento privo di versione 
				\\\arrayrulecolor{gray}\hline
				&
				Overleaf non permette l'automazione della workflow e una gestione sensata dei CI
				\\\arrayrulecolor{gray}\hline
				&
				Scelta dell'e-mail come sistema di comunicazione asincrona con il proponente
				\\
			\end{tabularx}
			\caption{retrospettiva del 06/11/2023.}
		\end{table}
		\noindent\\
		Il gruppo in risposta alle criticità individuate durante la retrospettiva ha deciso di:
		\begin{itemize}
			\item permettere nel sito:
			\begin{itemize}
				\item l'apertura di un documento PDF direttamente su una nuova scheda;
				\item di vedere direttamente vicino al link del documento la versione caricata;
			\end{itemize}
			\item ristrutturare i documenti della baseline della candidatura, riportare le motivazione della divisione delle ore stimate nel documento: preventivo dei costi e degli impegni. È stato deciso di sistemare tutti i documenti precedentemente prodotti con il nuovo registro delle modifiche e con il nuovo sistema di numerazione delle versioni, entrambi riportati nel documento Norme di Progetto (NdP);
			\item chiedere al prossimo incontro con il proponente di accordare un sistema di comunicazione asincrono;
			\item aggiungere a ogni file nel nome la versione attuale;
			\item passare allo sviluppo in locale dei documenti utilizzando TexLive e Visual Studio Code.
		\end{itemize}
		\noindent
		Si è deciso, per quanto riguarda i rapporti esterni con il proponente, di discutere nel prossimo incontro con lo stakeholder riguardo ai seguenti temi:
		\begin{itemize}
			\item instaurare un nuovo canale di comunicazione asincrona;
			\item accordare un metodo per dimostrare l'approvazione da parte del proponente del verbale esterno.
		\end{itemize}
		\noindent L'incontro con il proponente verrà fatto nella terza settimana di novembre 2023, per permettere al team di QB Software d'iniziare a preparare una bozza dell'analisi dei requisiti da presentare e discutere con il proponente.
		\\\noindent
		Il team ha discusso della struttura e dei contenuti del documento Norme di Progetto, il documento dovrà iniziare con una introduzione che riporti: gli scopi del prodotto e una breve spiegazione della sua struttura. Il contenuto del documento dovrà seguire quello dello standard ISO/IEC 12207:1997, questo perché il documento è rivolto al team di QB Software, e utilizzando una struttura già famigliare al team, quella dell'ISO, ci permette di ottenere o inserire le informazioni nelle Norme di Progetto più velocemente.
		Ogni attività verrà normata con le procedure definite dal team, queste procedure dovranno rispettare le indicazioni fornite dall'ISO.
		
    \section{Azioni da intraprendere}
        \begin{todo}
			\hline
            \href{https://github.com/QB-Software-swe/docs/issues/14}{VI-2023-11-06-\#14}
            &
            A. Bustreo,
			S. Destro
            &
            Creazione e strutturazione del documento Norme di Progetto
            \\\hline
            \href{https://github.com/QB-Software-swe/docs/issues/15}{VI-2023-11-06-\#15}
            &
            A. Domuta,
			A. Feltrin
            &
            Modificare la struttura del registro delle modifiche e aggiornare tutti i documenti precedenti
            \\\hline
            \href{https://github.com/QB-Software-swe/docs/issues/16}{VI-2023-11-06-\#16}
            &
            R. Fontana,
			A. Giurisato
            &
            Manutezione del sito
            \\\hline
    	\end{todo}
    
    \section{Ordine del giorno per la prossima riunione interna}
        \begin{itemize}
        		\item Iniziare la pianificazione delle prossime attività per la Requirements Technology Baseline.
    	\end{itemize}
\end{document}
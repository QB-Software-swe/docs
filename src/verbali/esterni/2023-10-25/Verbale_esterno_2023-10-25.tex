% DOC TYPE E DEF QB SOFTWARE %%%%%%%%%%%%%%%%%%%%%%%%%
\documentclass[12pt]{article}
%%%%%%%%%%%%%%%%%%%% PACKAGES %%%%%%%%%%%%%%%%%%%%
\usepackage[english,italian]{babel}
\usepackage[a4paper, margin=3cm]{geometry}
\usepackage[T1]{fontenc}
\usepackage[utf8]{inputenc}
\usepackage[table]{xcolor}
\usepackage{amsmath}
\usepackage{graphicx}
\usepackage{float}
\usepackage{tabularx}
\usepackage{booktabs}
\usepackage{hyperref}
\usepackage{xcolor}
\usepackage{multicol}
\usepackage{multirow}
\usepackage{soul}
\usepackage{enumitem}
\usepackage{textcomp}
\usepackage{eurosym}
\usepackage{lastpage}
\usepackage{fancyhdr}
\usepackage{adjustbox}
\usepackage{subfiles}

%%%%%%%%%%%%%%%%%%%% COMMAND -> New commands %%%%%%%%%%%%%%%%%%%%
\newcommand{\mailtoQBS}
{
	\href{mailto:qbsoftware.swe@gmail.com}{qbsoftware.swe@gmail.com}
}

\let\oldpar\paragraph
\renewcommand{\paragraph}[1]{\oldpar{#1}\mbox{}\\}

%%%%%%%%%%%%%%%%%%%% COMMAND -> Redefine LaTeX Commands %%%%%%%%%%%%%%%%%%%%
\def\title#1{\gdef\THETITLE{#1}}
\def\date#1{\gdef\THEDATE{#1}}
\def\footname#1{\gdef\THEFOOTNAME{#1}}

\let\oldtexteuro\texteuro
\renewcommand{\texteuro}{\euro}

%%%%%%%%%%%%%%%%%%%% STYLES -> Commands %%%%%%%%%%%%%%%%%%%%
\newcommand{\makefirstpage}
{
	\begin{titlepage}
		
		% Defines a new command for the horizontal lines, change thickness here
		\newcommand{\HRule}{\rule{\linewidth}{0.2mm}} 
		
		\center % Center everything on the page
		
		
		%	Heading Sections
		
		\textsc{\LARGE QB Software}\\[.1cm] 
		\includegraphics[scale=.15]{imgs/qb-software-logo.png}\\[-.1 cm] 
		\includegraphics[scale=.025]{imgs/x.png}\\[0.5cm]
		\includegraphics[scale=.3]{imgs/unipd_logo.png}\\[.5cm]
		\textsc{\Large Università degli studi di Padova}\\[0.5cm] 
		\textsc{\large corso di ingegneria del software }\\[0.5cm]
		\textsc{\large anno accademico 2023/2024 }\\[0.5cm]
		
		
		%	Title section and date section
		
		\ifdefined\THEDATE
		\HRule \\[0.4cm]
		{ \huge{ \bfseries {\THETITLE}} \\ [.5cm]
			\THEDATE}\\[0.4cm] 
		\HRule \\[0.4cm]
		\else
		\HRule \\[0.4cm]
		{ \huge{ \bfseries {\THETITLE}} } \\
		\HRule \\[0.4cm]
		\fi
		%
		\textsc{Contatti:} \mailtoQBS\\[0.3cm]
		
		\vfill 
		
	\end{titlepage}
}

%%%%%%%%%%%%%%%%%%%% ACCESSIBILITY %%%%%%%%%%%%%%%%%%%%
\let\oldhref\href
\renewcommand{\href}[2]{\oldhref{#1}{\textcolor{blue}{\ul{#2}}}}

\hypersetup{
	colorlinks = true,
	linkcolor = cyan,
}

%%%%%%%%%%%%%%%%%%%% STYLES -> Settings %%%%%%%%%%%%%%%%%%%%
% Enable header and footer style
\pagestyle{fancy}
\thispagestyle{empty}
\thispagestyle{fancy}
\pagestyle{fancy}

% Header
\setlength{\topmargin}{-40pt}
\setlength{\headsep}{60pt}
\fancyhf{}
\lhead{QB Software}
\setlength{\headheight}{15pt}
\rhead{\includegraphics[width=1cm]{imgs/qb-software-logo.png}}

% Footer 
\fancyfoot{}
\fancyfoot[L]{\THEFOOTNAME}
\fancyfoot[R]{Pagina \thepage~di~\pageref{LastPage}}
\futurelet\TMPfootrule\def\footrule{\TMPfootrule}
\setcounter{page}{0}
\pagenumbering{arabic}
\renewcommand{\footrulewidth}{0.3pt}

%%%%%%%%%%%%%%%%%%%% ENVIRONMENT %%%%%%%%%%%%%%%%%%%%
% Remove colorless padding in booktabs
\setlength{\aboverulesep}{0cm}
\setlength{\belowrulesep}{0cm}
\setlength{\extrarowheight}{.75ex}

\newenvironment{todo}{
	\rowcolors{2}{cyan!80!black!30!}{cyan!80!black!20!}
	\begin{tabular}{p{3.48cm}>{\raggedright\arraybackslash}p{4cm}>{\raggedright\arraybackslash}p{6.5cm}}
		\toprule
		\rowcolor{gray!20} \textbf{ID}	& \textbf{Interessato} & \textbf{Task} 
		\\\midrule
	}{
		\bottomrule
	\end{tabular}
}

\newenvironment{changelog}{
	\noindent
	{\Large \textbf{Registro delle modifiche}}
	\noindent
	\begin{table}[h]
		\rowcolors{2}{cyan!80!black!30!}{cyan!80!black!20!}
		\begin{adjustbox}{width=\textwidth}
			\begin{tabular}{|c|c|p{2.35cm}|c|p{3.3cm}|}
				\hline
				\rowcolor{gray!20}
				\textbf{V.} & \textbf{Data} & \textbf{Membro} & \textbf{Ruolo} & \textbf{Descrizione} \\
				
				\hline
			}{
				\hline
			\end{tabular}
		\end{adjustbox}
	\end{table}
	
	\clearpage
}

% 1   2    3        4             5            6            7          
% Ver Data Relatore RuoloRelatore DataVerifica Verificatore Descrizione
\newcommand{\newlog}[7]{
	#1 & #5 & #6 & Verificatore & Controllo qualità \\
	   & #2 & #3 & #4           & #7 \\\hline
}

\newcommand{\milestone}[3]{
	#1 & #2 & #3 & Responsabile & Approvazione \par documento
	\\\hline
}

% INFORMAZIONI DOCUMENTO %%%%%%%%%%%%%%%%%%%%%%%%%
\title{Verbale esterno}
\date{del 25 Ottobre 2023}
\footname{Verbale esterno del 25/10/2023}

\begin{document}
	% PRIME PAGINE %%%%%%%%%%%%%%%%%%%%%%%%%
	\makefirstpage
	
	% 1   2    3        4             5            6            7          
% Ver Data Relatore RuoloRelatore DataVerifica Verificatore Descrizione
\begin{changelog}
	\newlog{0.1.0}{12/12/2023}{A. Feltrin}{Responsabile}{16/12/2023}{A. Domuta}{Stesura verbale}
\end{changelog}
	\clearpage
	
	\tableofcontents
	\clearpage

    \section{Informazioni generali}
    
    \subsection{Luogo e data dell'incontro}
    
    \begin{itemize}
    	\item \textbf{Luogo}: meeting su Zextras Chat
    	\item \textbf{Data}: 25/10/2023
    	\item \textbf{Ora di inizio}: 17:00
    	\item \textbf{Ora di fine}: 17:30
    \end{itemize}
    
    \subsection{Presenze}
    
    \begin{itemize}
    	\item \textbf{Totale presenze}:
    	\begin{itemize}
    		\item Bustreo Alessandro
    		\item Destro Stefano
    		\item Domuta Alessia 
    		\item Feltrin Alessandro 
    		\item Fontana Raffaele Paolo 
    		\item Giurisato Andrea 
    		\item Rovea Silvia
    	\end{itemize}
    	
    	\item \textbf{Totale assenze}: 0
    	
    	\item \textbf{Partecipanti esterni}:
    	
    	\begin{itemize}
    		\item Alessio Crestani (Zextras)
    	\end{itemize}
    \end{itemize}
    
    \section{Ordine del giorno}
    \begin{itemize}
    	\item domande da porre all'azienda.
    \end{itemize}
    \clearpage
    
    \section{Verbale}
    Le domande poste da QB Software a Zextras durante il meeting:
    
    % Risposta
    \newcommand{\answer}{\item[\textbf{A:}]}
    
    \begin{enumerate}[label=\textbf{Q\arabic*:}]
    	\item Qual è la disponibilità dell'azienda per seguirci durante il progetto?
    	\answer Non c'è nessuna frequenza prefissata, siamo disponibili a seconda delle vostre necessità.
    	
    	\item Per comunicazioni brevi è possibile contattarvi via e-mail? Inoltre, è preferibile mettere come destinatario entrambi o solo uno?
    	\answer È possibile contattarci via e-mail, come destinatario potete mettere entrambi.
    	
    	\item Per il repository con il sorgente del software preferite che sia pubblico o privato?
    	\answer Non imponiamo restrizioni riguardo alla tipologia del repository, che può essere sia privato, e sia pubblico. È possibile aderire alla filosofia aziendale e rendere il vostro prodotto open source, se lo ritenete opportuno.
    	
    	\item Nella pratica, è necessario sviluppare un server che implementi il protocollo JMAP. Vorrei chiedere se avete delle raccomandazioni o consigli riguardo a un framework specifico che potrebbe essere appropriato per questo scopo?
    	\answer Non consigliamo di usare framework per lo sviluppo in quanto potrebbero influire sui risultati degli stress test da eseguire, è dunque preferibile usare Java "puro" per implementare il servizio. Inoltre, utilizzare un framework rendere meno scalabile il prodotto. Al posto di un framework possiamo aiutarvi a scegliere delle librerie utili a dialogare con il database. 
    	
    	\item Che tecnologie consigliate per lo sviluppo del database?
    	\answer Consigliamo di usare Postgre, e di utilizzare un database NoSQL. In ogni caso non c'è nessun vincolo sulla tipologia e che tipo di database utilizzare.
    	
    	\item Cosa comporta il significato di stateless a livello di programmazione? %TODO
    	\answer Stateless, a livello di programmazione, significa che ogni nodo del sistema deve essere capace di rispondere ad una qualunque richiesta, e quindi non deve essere vincolato dallo stato della conversazione. L'approccio che bisogna tenere è quello atomico alle operazioni. %FIXME: Quello che intende affermare è che ogni operazione è indipendete
    	
    	\item Cosa consigliate per creare e testare gli stess test?
    	\answer Chi di competenza non era disponibile, la domanda è stata rinviata per il prossimo meeting.
    \end{enumerate}
    
    \section{Azioni da intraprendere}
    Nessuna.
\end{document}
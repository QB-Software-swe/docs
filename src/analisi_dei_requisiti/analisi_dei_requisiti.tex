% DOC TYPE E DEF QB SOFTWARE %%%%%%%%%%%%%%%%%%%%%%%%%
\documentclass[12pt]{article}
%%%%%%%%%%%%%%%%%%%% PACKAGES %%%%%%%%%%%%%%%%%%%%
\usepackage[english,italian]{babel}
\usepackage[a4paper, margin=3cm]{geometry}
\usepackage[T1]{fontenc}
\usepackage[utf8]{inputenc}
\usepackage[table]{xcolor}
\usepackage{amsmath}
\usepackage{graphicx}
\usepackage{float}
\usepackage{tabularx}
\usepackage{booktabs}
\usepackage{hyperref}
\usepackage{xcolor}
\usepackage{multicol}
\usepackage{multirow}
\usepackage{soul}
\usepackage{enumitem}
\usepackage{textcomp}
\usepackage{eurosym}
\usepackage{lastpage}
\usepackage{fancyhdr}
\usepackage{adjustbox}
\usepackage{subfiles}

%%%%%%%%%%%%%%%%%%%% COMMAND -> New commands %%%%%%%%%%%%%%%%%%%%
\newcommand{\mailtoQBS}
{
	\href{mailto:qbsoftware.swe@gmail.com}{qbsoftware.swe@gmail.com}
}

\let\oldpar\paragraph
\renewcommand{\paragraph}[1]{\oldpar{#1}\mbox{}\\}

%%%%%%%%%%%%%%%%%%%% COMMAND -> Redefine LaTeX Commands %%%%%%%%%%%%%%%%%%%%
\def\title#1{\gdef\THETITLE{#1}}
\def\date#1{\gdef\THEDATE{#1}}
\def\footname#1{\gdef\THEFOOTNAME{#1}}

\let\oldtexteuro\texteuro
\renewcommand{\texteuro}{\euro}

%%%%%%%%%%%%%%%%%%%% STYLES -> Commands %%%%%%%%%%%%%%%%%%%%
\newcommand{\makefirstpage}
{
	\begin{titlepage}
		
		% Defines a new command for the horizontal lines, change thickness here
		\newcommand{\HRule}{\rule{\linewidth}{0.2mm}} 
		
		\center % Center everything on the page
		
		
		%	Heading Sections
		
		\textsc{\LARGE QB Software}\\[.1cm] 
		\includegraphics[scale=.15]{imgs/qb-software-logo.png}\\[-.1 cm] 
		\includegraphics[scale=.025]{imgs/x.png}\\[0.5cm]
		\includegraphics[scale=.3]{imgs/unipd_logo.png}\\[.5cm]
		\textsc{\Large Università degli studi di Padova}\\[0.5cm] 
		\textsc{\large corso di ingegneria del software }\\[0.5cm]
		\textsc{\large anno accademico 2023/2024 }\\[0.5cm]
		
		
		%	Title section and date section
		
		\ifdefined\THEDATE
		\HRule \\[0.4cm]
		{ \huge{ \bfseries {\THETITLE}} \\ [.5cm]
			\THEDATE}\\[0.4cm] 
		\HRule \\[0.4cm]
		\else
		\HRule \\[0.4cm]
		{ \huge{ \bfseries {\THETITLE}} } \\
		\HRule \\[0.4cm]
		\fi
		%
		\textsc{Contatti:} \mailtoQBS\\[0.3cm]
		
		\vfill 
		
	\end{titlepage}
}

%%%%%%%%%%%%%%%%%%%% ACCESSIBILITY %%%%%%%%%%%%%%%%%%%%
\let\oldhref\href
\renewcommand{\href}[2]{\oldhref{#1}{\textcolor{blue}{\ul{#2}}}}

\hypersetup{
	colorlinks = true,
	linkcolor = cyan,
}

%%%%%%%%%%%%%%%%%%%% STYLES -> Settings %%%%%%%%%%%%%%%%%%%%
% Enable header and footer style
\pagestyle{fancy}
\thispagestyle{empty}
\thispagestyle{fancy}
\pagestyle{fancy}

% Header
\setlength{\topmargin}{-40pt}
\setlength{\headsep}{60pt}
\fancyhf{}
\lhead{QB Software}
\setlength{\headheight}{15pt}
\rhead{\includegraphics[width=1cm]{imgs/qb-software-logo.png}}

% Footer 
\fancyfoot{}
\fancyfoot[L]{\THEFOOTNAME}
\fancyfoot[R]{Pagina \thepage~di~\pageref{LastPage}}
\futurelet\TMPfootrule\def\footrule{\TMPfootrule}
\setcounter{page}{0}
\pagenumbering{arabic}
\renewcommand{\footrulewidth}{0.3pt}

%%%%%%%%%%%%%%%%%%%% ENVIRONMENT %%%%%%%%%%%%%%%%%%%%
% Remove colorless padding in booktabs
\setlength{\aboverulesep}{0cm}
\setlength{\belowrulesep}{0cm}
\setlength{\extrarowheight}{.75ex}

\newenvironment{todo}{
	\rowcolors{2}{cyan!80!black!30!}{cyan!80!black!20!}
	\begin{tabular}{p{3.48cm}>{\raggedright\arraybackslash}p{4cm}>{\raggedright\arraybackslash}p{6.5cm}}
		\toprule
		\rowcolor{gray!20} \textbf{ID}	& \textbf{Interessato} & \textbf{Task} 
		\\\midrule
	}{
		\bottomrule
	\end{tabular}
}

\newenvironment{changelog}{
	\noindent
	{\Large \textbf{Registro delle modifiche}}
	\noindent
	\begin{table}[h]
		\rowcolors{2}{cyan!80!black!30!}{cyan!80!black!20!}
		\begin{adjustbox}{width=\textwidth}
			\begin{tabular}{|c|c|p{2.35cm}|c|p{3.3cm}|}
				\hline
				\rowcolor{gray!20}
				\textbf{V.} & \textbf{Data} & \textbf{Membro} & \textbf{Ruolo} & \textbf{Descrizione} \\
				
				\hline
			}{
				\hline
			\end{tabular}
		\end{adjustbox}
	\end{table}
	
	\clearpage
}

% 1   2    3        4             5            6            7          
% Ver Data Relatore RuoloRelatore DataVerifica Verificatore Descrizione
\newcommand{\newlog}[7]{
	#1 & #5 & #6 & Verificatore & Controllo qualità \\
	   & #2 & #3 & #4           & #7 \\\hline
}

\newcommand{\milestone}[3]{
	#1 & #2 & #3 & Responsabile & Approvazione \par documento
	\\\hline
}

% INFORMAZIONI DOCUMENTO %%%%%%%%%%%%%%%%%%%%%%%%%
\title{Analisi dei requisiti}
%\date{EMPTY}
\footname{Analisi dei requisiti}

\begin{document}
	% PRIME PAGINE %%%%%%%%%%%%%%%%%%%%%%%%%
	\makefirstpage
	
	% 1   2    3        4             5            6            7          
% Ver Data Relatore RuoloRelatore DataVerifica Verificatore Descrizione
\begin{changelog}
	\newlog{0.1.0}{12/12/2023}{A. Feltrin}{Responsabile}{16/12/2023}{A. Domuta}{Stesura verbale}
\end{changelog}
	\clearpage
	
	\tableofcontents
	\clearpage
	
	% CONTENUTO QUI %%%%%%%%%%%%%%%%%%%%%%%%%

	\section{Introduzione}
    	\subsection{Scopo del documento}
    		L'obiettivo di questo documento è quello di raccogliere i casi d’uso ed i requisiti relativi al capitolato proposto da Zextras: JMAP: il nuovo protocollo per la posta elettronica.
  
    
    	\subsection{Scopo del progetto}
    		Il progetto ha come obiettivo lo sviluppo di un servizio di posta elettronica che utilizza il protocollo JMAP.
    
    	\subsection{Glossario}
    		Un glossario in questo documento si rende necessario poiché alcuni termini potrebbero generare dei dubbi sul loro significato.
    		Tutti i termini facenti parte del Glossario verranno contrassegnati con una G.

    	\subsection{Riferimenti}
    		\subsubsection{Riferimenti normativi}
        		\begin{itemize}
            		\item \href{https://www.math.unipd.it/~tullio/IS-1/2023/Progetto/C8.pdf}{Capitolato C8 - JMAP: il nuovo protocollo per la posta elettronica}
        		\end{itemize}

	\section{Descrizione del prodotto}
    	\subsection{Funzioni del prodotto}


\section{Casi d'uso}
    \subsection{Introduzione}
    In questa parte del documento sono delineati i casi d'uso sviluppati in seguito all'analisi del capitolato, ai colloqui con il proponente e alle discussioni tra gli analisti del gruppo.
    Ciascun caso d’uso possiede un codice, la cui struttura è descritta nel documento Norme di Progetto V xxxxxx
    \subsection{Attori}
    In seguito ad una discussione con il proponente, gli attori individuati sono:
    \begin{itemize}
        \item Utente: il quale, a seguito di un colloquio con l'azienda, si riferisce ad un utente autenticato generico, in quanto, il proponente
        non richiede la registrazione dell'account, presupponendo, quindi, che l'utente sia già registrato nel sistema;
        \item Developer Zextras: si riferisce agli sviluppatori dell'azienda.
    \end{itemize}

    \subsection{UC1 - Scrittura email}
    \begin{itemize}
        \item Attore: utente.
        \item Descrizione: l'utente deve poter scrivere una email.
        \item Scenario principale:
            \begin{enumerate}
            \item l’utente apre l’applicazione;
            \item l'utente inserisce il destinatario della email (\textbf{UC1.1});
            \item l'utente inserisce l'oggetto della email (\textbf{UC1.2});
            \item l’utente scrive il corpo della email (\textbf{UC1.3});
            \item l’utente invia l'email (\textbf{UC1.4}).
            \end{enumerate}
        \item Estensioni: l'utente inserisce dei valori non validi nei campi oppure non è possibile inviare l'email e viene mostrato un messaggio d'errore (\textbf{UC1.4.1}).
        \item Precondizioni: l'utente vuole scrivere una email.
        \item Postcondizioni: l'email è stata creata e inviata.
    \end{itemize}

    \subsubsection{UC1.1 - Inserimento destinatario email}
    \begin{itemize}
        \item Attore: utente.
        \item Descrizione: l'utente deve poter inserire il destinatario della email.
        \item Scenario:
        \begin{enumerate}
        \item l'utente seleziona il campo relativo al destinatario della email;
        \item l'utente digita l'email del destinatario.
        \end{enumerate}
        \item Precondizioni: l'utente sta svolgendo la scrittura di una email;
        \item Postcondizioni: l'utente ha inserito il destinatario della email.
    \end{itemize}

    \subsubsection{UC1.2 - Inserimento oggetto email}
    \begin{itemize}
        \item Attore: utente.
        \item Descrizione: l'utente deve poter inserire l'oggetto della email.
        \item Scenario:
        \begin{enumerate}
        \item l'utente seleziona il campo relativo all'oggetto della email;
        \item l'utente inserisce l'oggetto della email.
        \end{enumerate}
        \item Precondizioni: l'utente sta svolgendo la scrittura di una email;
        \item Postcondizioni: l'utente ha inserito l'oggetto della email.
    \end{itemize}
    
    \subsubsection{UC1.3 - Inserimento corpo email}
    \begin{itemize}
        \item Attore: utente.
        \item Descrizione: l'utente deve poter scrivere il corpo della email.
        \item Scenario:
        \begin{enumerate}
        \item l'utente seleziona il campo relativo al corpo della email;
        \item l'utente scrive il corpo della email.
        \end{enumerate}
        \item Precondizioni: l'utente sta svolgendo la scrittura di una email;
        \item Postcondizioni: l'utente ha scritto il corpo della email.
    \end{itemize}

    \subsubsection{UC1.4 - Invio email}
    \begin{itemize}
        \item Attore: utente.
        \item Descrizione: l'utente deve poter inviare la email.
        \item Scenario:
        \begin{enumerate}
        \item l'utente preme il pulsante dedicato all'invio della email.
        \end{enumerate}
        \item Precondizioni: l'utente sta svolgendo la scrittura di una email.
        \item Postcondizioni: l'utente ha inviato l'email.
    \end{itemize}

    \subsubsection{UC1.4.1 - Visualizzazione errore }
    \begin{itemize}
        \item Attore: utente.
        \item Descrizione: l'utente deve ricevere un errore quando inserisce dei valori non validi nei campi oppure non è possibile inviare l'email.
        \item Scenario:
        \begin{enumerate}
        \item l'utente visualizza il messaggio d'errore.
        \end{enumerate}  
        \item Precondizioni: l'utente sta svolgendo la scrittura di una email.
        \item Postcondizioni: l'utente ha ricevuto un messaggio d' errore.
    \end{itemize}

    \subsection{UC2 - Visualizzazione lista email}
    \begin{itemize}
        \item Attore: utente.
        \item Descrizione: l'utente deve poter visualizzare le email ricevute e inviate.
        \item Scenario principale:
            \begin{enumerate}
            \item l’utente apre la lista delle email;
            \item l'utente visualizza una email specifica (\textbf{UC2.1});
            \item l’utente risponde ad una email (\textbf{UC2.2}).
            \end{enumerate}
        \item Precondizioni: l'utente vuole visualizzare le email ricevute;
        \item Postcondizioni: l'utente ha visualizzato le email ricevute.
    \end{itemize}

    \subsubsection{UC2.1 - Visualizzazione email specifica}
    \begin{itemize}
        \item Attore: utente.
        \item Descrizione: l'utente deve poter visualizzare una email specifica.
        \item Scenario:
        \begin{enumerate}
        \item l'utente seleziona una email;
        \item l'utente visualizza l'email.
        \end{enumerate}
        \item Precondizioni: l'utente sta visualizzando le email ricevute.
        \item Postcondizioni: l'utente ha visualizzato una email specifica.
    \end{itemize}

    \subsubsection{UC2.2 - Rispondere ad una email}
    \begin{itemize}
        \item Attore: utente.
        \item Descrizione: l'utente deve poter rispondere ad una email.
        \item Scenario:
        \begin{enumerate}
        \item l'utente seleziona l'email a cui vuole rispondere (\textbf{UC2.2.1});
        \item l'utente scrive la risposta (\textbf{UC2.2.2});
        \item l'utente conferma l'invio della risposta (\textbf{UC2.2.3}).
        \end{enumerate}
        \item Estensioni: non è possibile inviare l'email e viene mostrato un messaggio d'errore (\textbf{UC2.2.3.1}).
        \item Precondizioni: l'utente vuole visualizaare e rispondere ad una email.
        \item Postcondizioni: l'utente ha risposto alla email.
    \end{itemize}

    \subsubsection{UC2.2.1 - Selezionare email}
    \begin{itemize}
        \item Attore: utente.
        \item Descrizione: l'utente deve poter selezionare una email specifica per rispondere.
        \item Scenario:
        \begin{enumerate}
        \item l'utente seleziona una email.
        \end{enumerate}
        \item Precondizioni: l'utente sta selezionando una email per rispondere.;
        \item Postcondizioni: l'utente ha selezionato una email specifica.
    \end{itemize}

    \subsubsection{UC2.2.2 - Scrivere risposta}
    \begin{itemize}
        \item Attore: utente.
        \item Descrizione: l'utente deve poter scrivere una risposta.
        \item Scenario:
        \begin{enumerate}
        \item l'utente scrive l'oggetto;
        \item l'utente scrive il corpo della email.
        \end{enumerate}
        \item Precondizioni: l'utente ha selezionando una email per rispondere.;
        \item Postcondizioni: l'utente ha scritto una risposta.
    \end{itemize}

    \subsubsection{UC2.2.3 - Confermare invio}
    \begin{itemize}
        \item Attore: utente.
        \item Descrizione: l'utente deve poter inviare la risposta.
        \item Scenario:
        \begin{enumerate}
        \item l'utente preme il pulsante dedicato all'invio della email.
        \end{enumerate}
        \item Precondizioni: l'utente ha scritto una risposta.
        \item Postcondizioni: l'utente ha inviato l'email.
    \end{itemize}

    \subsubsection{UC2.2.3.1 - Visualizzazione errore}
    \begin{itemize}
        \item Attore: utente.
        \item Descrizione: l'utente deve ricevere un errore quando non è possibile inviare l'email.
        \item Scenario:
        \begin{enumerate}
        \item l'utente visualizza il messaggio d'errore.
        \end{enumerate}  
        \item Precondizioni: l'utente conferma l'invio della risposta.
        \item Postcondizioni: l'utente ha ricevuto un messaggio d'errore.
    \end{itemize}

    \subsection{UC3 - Eliminazione email}
    \begin{itemize}
        \item Attore: utente.
        \item Descrizione: l'utente deve poter eliminare una o più email.
        \item Scenario principale:
            \begin{enumerate}
            \item l’utente seleziona una o più email (\textbf{UC3.1});
            \item l’utente conferma l'eliminazione di una o più email (\textbf{UC3.2}).
            \end{enumerate}
        \item Estensioni: non è possibile eliminare le email selezionate e viene mostrato un messaggio d'errore (\textbf{UC3.2.1}).
        \item Precondizioni: l'utente vuole eliminare una o più email.
        \item Postcondizioni: l'utente ha eliminato una o più email.
    \end{itemize}

    \subsubsection{UC3.1 - Selezionare email}
    \begin{itemize}
        \item Attore: utente.
        \item Descrizione: l'utente deve poter selezionare una o più email.
        \item Scenario:
            \begin{enumerate}
            \item l'utente seleziona una o più email.
            \item \end{enumerate}
        \item Precondizioni: l'utente sta selezionando una o più email.
        \item Postcondizioni: l'utente ha selezionato una o più email.
    \end{itemize}

    \subsubsection{UC3.2 - Conferma eliminazione}
    \begin{itemize}
        \item Attore: utente.
        \item Descrizione: l'utente deve poter eliminare le email selezionate.
        \item Scenario:
        \begin{enumerate}
        \item l'utente preme il pulsante dedicato ed elimina le email selezionate.
        \end{enumerate}
        \item Precondizioni: l'utente vuole eliminare le email selezionate.
        \item Postcondizioni: l'utente ha eliminato le email selezionate.
    \end{itemize}

    \subsection{UC3.2.1 - Visualizzazione errore}
    \begin{itemize}
        \item Attore: utente;
        \item Descrizione: l'utente deve ricevere un messaggio d'errore quando non è possibile eliminare le email selezionate.
        \item Scenario:
        \begin{enumerate}
        \item l'utente visualizza il messaggio d'errore.
        \end{enumerate}
        \item Precondizioni: l'utente sta eliminando le email selezionate.
        \item Postcondizioni: l'utente ha visualizzato un messaggio d'errore.
    \end{itemize}

    \subsection{UC4 - Refresh}
    \begin{itemize}
        \item Attore: utente.
        \item Descrizione: l'utente deve poter eseguire un refresh per visualizzare nuovi dati.
        \item Scenario principale:
            \begin{enumerate}
            \item l'utente preme il pulsante dedicato per eseguire il refresh della pagina;
            \item l'utente visualizza un pagina aggiornata (\textbf{UC4.1}).
            \end{enumerate}
        \item Precondizioni: l'utente vuole visualizzare una pagina aggiornata.
        \item Postcondizioni: l'utente ha visualizzato una pagina aggiornata.
    \end{itemize}

    \subsubsection{UC4.1 - Visualizzazione dati aggiornati}
    \begin{itemize}
        \item Attore: utente.
        \item Descrizione: l'utente visualizza i dati aggiornati.
        \item Scenario:
        \begin{enumerate}
        \item l'utente visualizza i dati aggiornati.
        \end{enumerate}
        \item Precondizioni: l'utente vuole visualizzare una pagina aggiornata.
        \item Postcondizioni: l'utente ha visualizzato una pagina aggiornata.
    \end{itemize}



%%%%%%%%%%%%%%% SCENARIO EVENTI %%%%%%%%%%%%%%%%%%%%%%%

\subsection{UC 12 - Creazione evento nel calendario}
\begin{itemize}
    \item Attore: utente;
    \item Descrizione: l'utente deve poter creare nuovi eventi nel calendario;
    \item Scenario principale:
        \begin{enumerate}
        \item l'utente inserisce il nome dell'evento (\hyperref[sec: UC 12.1]{UC 12.1});
        \item l'utente inserisce la data (\hyperref[sec: UC 12.2]{UC 12.2});
        \item l'utente inserisce una fascia oraria (\hyperref[sec: UC 12.3]{UC 12.3});
        \item l'utente conferma la creazione dell'evento (\hyperref[sec: UC 12.4]{UC 12.4}).
        \end{enumerate}
    \item Estensioni: l'utente inserisce dei valori non validi nei campi e viene mostrato un messaggio d'errore (\hyperref[sec: UC 12.4.1]{UC 12.4.1});
    \item Precondizioni: l'utente ha acceduto al calendario e vuole creare un nuovo evento;
    \item Postcondizioni: un nuovo evento è stato creato dall'utente.
\end{itemize}

\subsubsection{UC 12.1 - Inserimento nome evento} \label{sec: UC 12.1}
\begin{itemize}
    \item Attore: utente;
    \item Descrizione: l'utente deve poter inserire il nome dell'evento che vuole \par aggiungere al calendario;
    \item Scenario:
        \begin{enumerate}
        \item l'utente seleziona il campo relativo al nome dell'evento;
        \item l'utente digita il nome dell'evento che vuole creare.
        \end{enumerate}
    
    \item Precondizioni: l'utente vuole creare un nuovo evento;
    \item Postcondizioni: l'utente ha compilato il campo relativo al nome dell'evento.
\end{itemize}


\subsubsection{UC 12.2 - Inserimento data evento} \label{sec: UC 12.2}
\begin{itemize}
    \item Attore: utente;
    \item Descrizione: l'utente deve poter inserire la data dell'evento da creare;
    \item Scenario:
        \begin{enumerate}
        \item l'utente seleziona il campo relativo alla data dell'evento;
        \item l'utente inserisce la data per l'evento che vuole creare.
        \end{enumerate}
    
    \item Precondizioni: l'utente vuole creare un nuovo evento;
    \item Postcondizioni: l'utente ha compilato il campo relativo alla data dell'evento.
\end{itemize}


\subsubsection{UC 12.3 - Inserimento ora evento} \label{sec: UC 12.3}
\begin{itemize}
    \item Attore: utente;
    \item Descrizione: l'utente deve poter inserire l'orario dell'evento;
    \item Scenario:
        \begin{enumerate}
        \item l'utente seleziona il campo relativo all'orario dell'evento;
        \item l'utente inserisce la fascia oraria dell'evento che vuole creare.
        \end{enumerate}
    
    \item Precondizioni: l'utente vuole creare un nuovo evento;
    \item Postcondizioni: l'utente ha compilato il campo relativo all'orario dell'evento.
\end{itemize}

\subsubsection{UC 12.4 - Conferma creazione evento} \label{sec: UC 12.4}
\begin{itemize}
    \item Attore: utente;
    \item Descrizione: l'utente deve poter confermare la creazione dell'evento;
    \item Scenario:
        \begin{enumerate}
        \item l'utente conferma la creazione dell'evento.
        \end{enumerate}
    
    \item Precondizioni: l'utente vuole creare un nuovo evento;
    \item Postcondizioni: l'utente ha confermato la creazione dell'evento.
\end{itemize}

\subsubsection{UC 12.4.1 - Visualizzazione errore creazione evento} \label{sec: UC 12.4.1}
\begin{itemize}
    \item Attore: utente;
    \item Descrizione: l'utente deve ricevere un errore a seguito di dati non validi inseriti durante la procedura per creare un evento;
    \item Scenario:
        \begin{enumerate}
        \item l'utente visualizza il messaggio d'errore.
        \end{enumerate}
    
    \item Precondizioni: l'utente vuole creare un nuovo evento e inserisce dei dati non validi;
    \item Postcondizioni: l'utente ha ricevuto un messaggio d'errore.
\end{itemize}



\subsection{UC 13 - Condivisione di un evento}
\begin{itemize}
    \item Attore: utente;
    \item Descrizione: l'utente deve poter condividere degli eventi con altri utenti;
    \item Scenario principale:
        \begin{enumerate}
        \item l'utente sceglie l'evento da condividere (\hyperref[sec: UC 13.1]{UC 13.1});
        \item l'utente inserisce il nome del destinatario dell'evento (\hyperref[sec: UC 13.2]{UC 13.2});
        \item l'utente conferma la condivisione di un evento (\hyperref[sec: UC 13.3]{UC 13.3}).
        \end{enumerate}
    \item Estensioni: l'utente inserisce dei valori non validi nei campi e viene mostrato un messaggio d'errore (\hyperref[sec: UC 13.3.1]{UC 13.3.1});
    \item Precondizioni: l'utente ha acceduto al calendario e vuole condividere un evento;
    \item Postcondizioni: l'utente ha condiviso un evento.
\end{itemize}

\subsubsection{UC 13.1 - Scelta dell'evento da condividere} \label{sec: UC 13.1}
\begin{itemize}
    \item Attore: utente;
    \item Descrizione: l'utente deve scegliere l'evento del calendario che vuole \par condividere;
    \item Scenario:
        \begin{enumerate}
        \item l'utente seleziona l'evento.
        \end{enumerate}
    
    \item Precondizioni: l'utente vuole condividere un evento;
    \item Postcondizioni: l'utente ha scelto l'evento.
\end{itemize}


\subsubsection{UC 13.2 - Scelta del destinatario} \label{sec: UC 13.2}
\begin{itemize}
    \item Attore: utente;
    \item Descrizione: l'utente inserisce il destinatario con cui desidera condividere l'evento;
    \item Scenario:
        \begin{enumerate}
        \item l'utente digita il nome dell'utente destinatario.
        \end{enumerate}
    
    \item Precondizioni: l'utente vuole condividere un evento;
    \item Postcondizioni: l'utente ha scelto il destinatario.
\end{itemize}


\subsubsection{UC 13.3 - Conferma condivisione evento} \label{sec: UC 13.3}
\begin{itemize}
    \item Attore: utente;
    \item Descrizione: l'utente conferma la condivisione di un evento;
    \item Scenario:
        \begin{enumerate}
        \item l'utente deve confermare la condivisione dell'evento scelto con il destinatario.
        \end{enumerate}
    
    \item Precondizioni: l'utente vuole condividere un evento;
    \item Postcondizioni: l'utente confermato la condivisione dell'evento.
\end{itemize}

\subsubsection{UC 13.3.1 - Visualizzazione errore condivisione evento} \label{sec: UC 13.3.1}
\begin{itemize}
    \item Attore: utente;
    \item Descrizione: l'utente deve ricevere un errore a seguito di dati non validi inseriti durante la procedura per condividere un evento;
    \item Scenario:
        \begin{enumerate}
        \item l'utente visualizza il messaggio d'errore.
        \end{enumerate}
    
    \item Precondizioni:l'utente vuole condividere un evento e inserisce dei dati non validi;
    \item Postcondizioni: l'utente ha ricevuto un messaggio d'errore.
\end{itemize}


\subsection{UC 14 - Eliminazione evento nel calendario}
\begin{itemize}
    \item Attore: utente;
    \item Descrizione: l'utente deve poter eliminare eventi nel calendario;
    \item Scenario principale:
        \begin{enumerate}
        \item l'utente sceglie l'evento da eliminare (\hyperref[sec: UC 14.1]{UC 14.1});
        \item l'utente conferma l'eliminazione (\hyperref[sec: UC 14.2]{UC 14.2});
        \item l'utente riceve il messaggio di eliminazione di un evento (\hyperref[sec: UC 14.3]{UC 14.3}).
        \end{enumerate}
    \item Precondizioni: l'utente ha acceduto al calendario e vuole eliminare un evento;
    \item Postcondizioni: l'evento scelto è stato eliminato.
\end{itemize}

\subsubsection{UC 14.1 - Scelta dell'evento da eliminare} \label{sec: UC 14.1}
\begin{itemize}
    \item Attore: utente;
    \item Descrizione: l'utente deve poter scegliere l'evento che vuole eliminare dal calendario;
    \item Scenario:
        \begin{enumerate}
        \item l'utente seleziona l'evento.
        \end{enumerate}
    
    \item Precondizioni: l'utente vuole eliminare un evento;
    \item Postcondizioni: l'utente ha scelto l'evento da eliminare.
\end{itemize}


\subsubsection{UC 14.2 - Conferma eliminazione evento} \label{sec: UC 14.2}
\begin{itemize}
    \item Attore: utente;
    \item Descrizione: l'utente deve confermare l'eliminazione dell'evento scelto;
    \item Scenario:
        \begin{enumerate}
        \item l'utente conferma l'eliminazione.
        \end{enumerate}
    
    \item Precondizioni: l'utente vuole eliminare un evento;
    \item Postcondizioni: l'utente ha confermato l'eliminazione dell'evento.
\end{itemize}


\subsubsection{UC 14.3 - Ricezione messaggio eliminazione} \label{sec: UC 14.3}
\begin{itemize}
    \item Attore: utente;
    \item Descrizione: l'utente deve ricevere un messaggio che garantisce il buon esito dell'operazione;
    \item Scenario:
        \begin{enumerate}
        \item l'utente riceve il messaggio informativo.
        \end{enumerate}
    
    \item Precondizioni: l'utente vuole eliminare un evento;
    \item Postcondizioni: l'utente ha ricevuto il messaggio di corretta eliminazione.
\end{itemize}


\subsection{UC 15 - Eliminazione evento condiviso}
\begin{itemize}
    \item Attore: utente;
    \item Descrizione: l'utente deve poter eliminare eventi condivisi;
    \item Scenario principale:
        \begin{enumerate}
        \item l'utente sceglie l'evento condiviso da eliminare (\hyperref[sec: UC 15.1]{UC 15.1});
        \item l'utente conferma l'eliminazione (\hyperref[sec: UC 15.2]{UC 15.2});
        \item l'utente riceve il messaggio d'eliminazione dell'evento condiviso (\hyperref[sec: UC 15.3]{UC 15.3}).
        \end{enumerate}
    \item Precondizioni: l'utente ha acceduto al calendario e vuole eliminare un evento condiviso esistente;
    \item Postcondizioni: un evento condiviso è stato eliminato dall'utente.
\end{itemize}

\subsubsection{UC 15.1 - Scelta dell'evento condiviso da eliminare} \label{sec: UC 15.1}
\begin{itemize}
    \item Attore: utente;
    \item Descrizione: l'utente deve poter scegliere l'evento condiviso che vuole \par eliminare;
    \item Scenario:
        \begin{enumerate}
        \item l'utente seleziona l'evento.
        \end{enumerate}
    
    \item Precondizioni: l'utente vuole eliminare un evento condiviso;
    \item Postcondizioni: l'utente ha scelto l'evento condiviso da eliminare.
\end{itemize}


\subsubsection{UC 15.2 - Conferma eliminazione evento condiviso} \label{sec: UC 15.2}
\begin{itemize}
    \item Attore: utente;
    \item Descrizione: l'utente deve confermare l'eliminazione dell'evento condiviso \par scelto;
    \item Scenario:
        \begin{enumerate}
        \item l'utente conferma l'eliminazione.
        \end{enumerate}
    
    \item Precondizioni: l'utente vuole eliminare un evento condiviso;
    \item Postcondizioni: l'utente ha confermato l'eliminazione dell'evento.
\end{itemize}


\subsubsection{UC 15.3 - Ricezione messaggio eliminazione} \label{sec: UC 15.3}
\begin{itemize}
    \item Attore: utente;
    \item Descrizione: l'utente deve ricevere un messaggio che garantisce il buon esito dell'operazione;
    \item Scenario:
        \begin{enumerate}
        \item l'utente riceve il messaggio informativo.
        \end{enumerate}
    
    \item Precondizioni: l'utente vuole eliminare un evento condiviso;
    \item Postcondizioni: l'utente ha ricevuto il messaggio di corretta eliminazione.
\end{itemize}



%%%%%%%%%%%%%%% SCENARIO CONTATTI %%%%%%%%%%%%%%%%%%%%%%%

\subsection{UC 16 - Creazione nuovo contatto}
\begin{itemize}
    \item Attore: utente;
    \item Descrizione: l'utente deve poter creare nuovi contatti;
    \item Scenario principale:
        \begin{enumerate}
        \item l'utente inserisce il nome del contatto (\hyperref[sec: UC 16.1]{UC 16.1});
        \item l'utente inserisce l'email del nuovo contatto (\hyperref[sec: UC 16.2]{UC 16.2});
        \item l'utente conferma la creazione del contatto (\hyperref[sec: UC 16.3]{UC 16.3}).
        \end{enumerate}
    \item Estensioni: l'utente inserisce dei valori non validi nei campi e viene mostrato un messaggio d'errore (\hyperref[sec: UC 16.3.1]{UC 16.3.1});
    \item Precondizioni: l'utente ha acceduto alla rubrica e vuole creare un nuovo contatto;
    \item Postcondizioni: un nuovo contatto è stato creato dall'utente.
\end{itemize}


\subsubsection{UC 16.1 - Inserimento nome nuovo contatto} \label{sec: UC 16.1}
\begin{itemize}
    \item Attore: utente;
    \item Descrizione: l'utente deve poter inserire il nome del nuovo contatto che vuole aggiungere alla rubrica;
    \item Scenario:
        \begin{enumerate}
        \item l'utente seleziona il campo relativo al nome del contatto;
        \item l'utente digita il nome.
        \end{enumerate}
    
    \item Precondizioni: l'utente vuole creare un nuovo contatto;
    \item Postcondizioni: l'utente ha compilato il campo relativo al nome del contatto.
\end{itemize}


\subsubsection{UC 16.2 - Inserimento email contatto} \label{sec: UC 16.2}
\begin{itemize}
    \item Attore: utente;
    \item Descrizione: l'utente deve poter inserire la mail del nuovo contatto da creare;
    \item Scenario:
        \begin{enumerate}
        \item l'utente seleziona il campo relativo alla mail;
        \item l'utente inserisce la mail del contatto.
        \end{enumerate}
    
    \item Precondizioni: l'utente vuole creare un nuovo contatto;
    \item Postcondizioni: l'utente ha compilato il campo relativo alla mail del contatto.
\end{itemize}


\subsubsection{UC 16.3 - Conferma creazione contatto} \label{sec: UC 16.3}
\begin{itemize}
    \item Attore: utente;
    \item Descrizione: l'utente deve poter confermare la creazione del nuovo contatto;
    \item Scenario:
        \begin{enumerate}
        \item l'utente conferma la creazione.
        \end{enumerate}
    
    \item Precondizioni: l'utente vuole creare un nuovo contatto;
    \item Postcondizioni: l'utente ha confermato la creazione di un nuovo contatto.
\end{itemize}


\subsubsection{UC 16.3.1 - Visualizzazione errore creazione contatto} \label{sec: UC 16.3.1}
\begin{itemize}
    \item Attore: utente;
    \item Descrizione: l'utente deve ricevere un errore a seguito di dati non validi inseriti durante la procedura per creare un nuovo contatto;
    \item Scenario:
        \begin{enumerate}
        \item l'utente visualizza il messaggio d'errore.
        \end{enumerate}
    
    \item Precondizioni: l'utente sta svolgendo la creazione di un contatto e inserisce dei dati non validi;
    \item Postcondizioni: l'utente ha ricevuto un messaggio d'errore.
\end{itemize}


\subsection{UC17 - Condivisione di un contatto}
\begin{itemize}
    \item Attore: utente;
    \item Descrizione: l'utente deve poter condividere un contatto con altri utenti;
    \item Scenario principale:
        \begin{enumerate}
        \item l'utente sceglie il contatto da condividere (\hyperref[sec: UC 17.1]{UC 17.1});
        \item l'utente inserisce il nome del destinatario del utente (\hyperref[sec: UC 17.2]{UC 17.2});
        \item l'utente conferma la condivisione del contatto (\hyperref[sec: UC 17.3]{UC 17.3}).
        \end{enumerate}
    \item Estensioni: l'utente inserisce dei valori non validi nei campi e viene mostrato un messaggio d'errore (\hyperref[sec: UC 17.3.1]{UC 17.3.1});
    \item Precondizioni: l'utente ha acceduto alla rubrica e vuole condividere un contatto;
    \item Postcondizioni: l'utente ha condiviso un contatto.
\end{itemize}

\subsubsection{UC 17.1 - Scelta del contatto da condividere} \label{sec: UC 17.1}
\begin{itemize}
    \item Attore: utente;
    \item Descrizione: l'utente deve scegliere il contatto della rubrica che vuole \par condividere;
    \item Scenario:
        \begin{enumerate}
        \item l'utente seleziona il contatto.
        \end{enumerate}
    
    \item Precondizioni: l'utente vuole condividere un contatto;
    \item Postcondizioni: l'utente ha scelto il contatto.
\end{itemize}


\subsubsection{UC 17.2 - Scelta del destinatario} \label{sec: UC 17.2}
\begin{itemize}
    \item Attore: utente;
    \item Descrizione: l'utente inserisce il destinatario con cui desidera condividere il contatto;
    \item Scenario:
        \begin{enumerate}
        \item l'utente digita il nome dell'utente destinatario.
        \end{enumerate}
    
    \item Precondizioni: l'utente vuole condividere un contatto;
    \item Postcondizioni: l'utente ha scelto il destinatario.
\end{itemize}


\subsubsection{UC 17.3 - Conferma condivisione contatto} \label{sec: UC 17.3}
\begin{itemize}
    \item Attore: utente;
    \item Descrizione: l'utente conferma la condivisione del contatto;
    \item Scenario:
        \begin{enumerate}
        \item l'utente deve confermare la condivisione del contatto scelto con l'utente scelto.
        \end{enumerate}
    
    \item Precondizioni: l'utente vuole condividere un contatto;
    \item Postcondizioni: l'utente ha confermato la condivisione del contatto.
\end{itemize}

\subsubsection{UC 17.3.1 - Visualizzazione errore condivisione contatto} \label{sec: UC 17.3.1}
\begin{itemize}
    \item Attore: utente;
    \item Descrizione: l'utente deve ricevere un errore a seguito di dati non validi inseriti durante la procedura per condividere un contatto;
    \item Scenario:
        \begin{enumerate}
        \item l'utente visualizza il messaggio d'errore.
        \end{enumerate}
    
    \item Precondizioni: l'utente vuole condividere un contatto e inserisce dei dati non validi;
    \item Postcondizioni: l'utente ha ricevuto un messaggio d'errore.
\end{itemize}



\subsection{UC 18 - Eliminazione contatto}
\begin{itemize}
    \item Attore: utente;
    \item Descrizione: l'utente deve poter eliminare un contatto dalla rubrica;
    \item Scenario principale:
        \begin{enumerate}
        \item l'utente sceglie il contatto da eliminare (\hyperref[sec: UC 18.1]{UC 18.1});
        \item l'utente conferma l'eliminazione (\hyperref[sec: UC 18.2]{UC 18.2});
        \item l'utente riceve il messaggio di eliminazione di un contatto (\hyperref[sec: UC 18.3]{UC 18.3}).
        \end{enumerate}
    \item Precondizioni: l'utente ha acceduto alla rubrica e vuole eliminare un contatto;
    \item Postcondizioni: il contatto scelto è stato eliminato.
\end{itemize}

\subsubsection{UC 18.1 - Scelta del contatto da eliminare} \label{sec: UC 18.1}
\begin{itemize}
    \item Attore: utente;
    \item Descrizione: l'utente deve poter scegliere il contatto che vuole eliminare dalla rubrica;
    \item Scenario:
        \begin{enumerate}
        \item l'utente seleziona il contatto.
        \end{enumerate}
    
    \item Precondizioni: l'utente vuole eliminare un contatto;
    \item Postcondizioni: l'utente ha scelto l'evento da eliminare.
\end{itemize}


\subsubsection{UC 18.2 - Conferma eliminazione evento} \label{sec: UC 18.2}
\begin{itemize}
    \item Attore: utente;
    \item Descrizione: l'utente deve confermare l'eliminazione del contatto scelto;
    \item Scenario:
        \begin{enumerate}
        \item l'utente conferma l'eliminazione.
        \end{enumerate}
    
    \item Precondizioni: l'utente vuole eliminare un contatto;
    \item Postcondizioni: l'utente ha confermato l'eliminazione del contatto.
\end{itemize}


\subsubsection{UC 18.3 - Ricezione messaggio eliminazione} \label{sec: UC 18.3}
\begin{itemize}
    \item Attore: utente;
    \item Descrizione: l'utente deve ricevere un messaggio che garantisce il buon esito dell'operazione;
    \item Scenario:
        \begin{enumerate}
        \item l'utente riceve il messaggio informativo.
        \end{enumerate}
    
    \item Precondizioni: l'utente vuole eliminare un contatto;
    \item Postcondizioni: l'utente ha ricevuto il messaggio di corretta eliminazione.
\end{itemize}

%%%%%%%%%%%%%%% TESTING %%%%%%%%%%%%%%%%%%%%%%

\subsection{UC 112 - Eliminazione contatto}
\begin{itemize}
    \item Attore: developer Zextras;
    \item Descrizione: l'utente deve poter fare degli stess testper misurare le performance del prodotto;
    \item Scenario principale:
        \begin{enumerate}
        \item l'utente avvia uno o una serie di stress test. 
        \end{enumerate}
    \item Precondizioni: l'utente vuole avviare degli stess test;
    \item Postcondizioni: l'utente visualizza i risultati dei test.
\end{itemize}

\end{document}
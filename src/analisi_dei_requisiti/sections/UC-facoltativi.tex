
%%%%%%%%%%%%%%% SCENARIO EVENTI %%%%%%%%%%%%%%%%%%%%%%%

\subsection{UC9 - Creazione evento nel calendario}
\begin{itemize}
    \item Attore: utente;
    \item Descrizione: l'utente deve poter creare nuovi eventi nel calendario;
    \item Scenario principale:
        \begin{enumerate}
        \item l'utente inserisce il nome dell'evento (\hyperref[sec: UC9.1]{UC9.1});
        \item l'utente inserisce la data (\hyperref[sec: UC9.2]{UC9.2});
        \item l'utente inserisce una fascia oraria (\hyperref[sec: UC9.3]{UC9.3});
        \item l'utente conferma la creazione dell'evento (\hyperref[sec: UC9.4]{UC9.4}).
        \end{enumerate}
    \item Estensioni: l'utente inserisce dei valori non validi nei campi e viene mostrato un messaggio d'errore (\hyperref[sec: UC9.4.1]{UC9.4.1});
    \item Precondizioni: l'utente ha acceduto al calendario e vuole creare un nuovo evento;
    \item Postcondizioni: un nuovo evento è stato creato dall'utente.
\end{itemize}

\subsubsection{UC9.1 - Inserimento nome evento} \label{sec: UC9.1}
\begin{itemize}
    \item Attore: utente;
    \item Descrizione: l'utente deve poter inserire il nome dell'evento che vuole \par aggiungere al calendario;
    \item Scenario:
        \begin{enumerate}
        \item l'utente seleziona il campo relativo al nome dell'evento;
        \item l'utente digita il nome dell'evento che vuole creare.
        \end{enumerate}
    
    \item Precondizioni: l'utente vuole creare un nuovo evento;
    \item Postcondizioni: l'utente ha compilato il campo relativo al nome dell'evento.
\end{itemize}


\subsubsection{UC9.2 - Inserimento data evento} \label{sec: UC9.2}
\begin{itemize}
    \item Attore: utente;
    \item Descrizione: l'utente deve poter inserire la data dell'evento da creare;
    \item Scenario:
        \begin{enumerate}
        \item l'utente seleziona il campo relativo alla data dell'evento;
        \item l'utente inserisce la data per l'evento che vuole creare.
        \end{enumerate}
    
    \item Precondizioni: l'utente vuole creare un nuovo evento;
    \item Postcondizioni: l'utente ha compilato il campo relativo alla data dell'evento.
\end{itemize}


\subsubsection{UC9.3 - Inserimento ora evento} \label{sec: UC9.3}
\begin{itemize}
    \item Attore: utente;
    \item Descrizione: l'utente deve poter inserire l'orario dell'evento;
    \item Scenario:
        \begin{enumerate}
        \item l'utente seleziona il campo relativo alla ora dell'evento;
        \item l'utente inserisce la fascai oraria dell'evento che vuole creare.
        \end{enumerate}
    
    \item Precondizioni: l'utente vuole creare un nuovo evento;
    \item Postcondizioni: l'utente ha compilato il campo relativo all'orario dell'evento.
\end{itemize}

\subsubsection{UC9.4 - Conferma creazione evento} \label{sec: UC9.4}
\begin{itemize}
    \item Attore: utente;
    \item Descrizione: l'utente deve poter confermare la creazione dell'evento;
    \item Scenario:
        \begin{enumerate}
        \item l'utente conferma la creazione dell'evento.
        \end{enumerate}
    
    \item Precondizioni: l'utente vuole creare un nuovo evento;
    \item Postcondizioni: l'utente ha confermato la creazione dell'evento.
\end{itemize}

\subsubsection{UC9.4.1 - Visualizzazione errore creazione evento} \label{sec: UC9.4.1}
\begin{itemize}
    \item Attore: utente;
    \item Descrizione: l'utente deve ricevere un errore a seguito di dati non validi inseriti durante la procedura per creare un evento;
    \item Scenario:
        \begin{enumerate}
        \item l'utente visualizza il messaggio d'errore.
        \end{enumerate}
    
    \item Precondizioni: l'utente vuole creare un nuovo evento e inserisce dei dati non validi;
    \item Postcondizioni: l'utente ha ricevuto un messaggio d'errore.
\end{itemize}



\subsection{UC10 - Condivisione di un evento}
\begin{itemize}
    \item Attore: utente;
    \item Descrizione: l'utente deve poter condividere degli eventi con altri utenti;
    \item Scenario principale:
        \begin{enumerate}
        \item l'utente sceglie l'evento da condividere (\hyperref[sec: UC10.1]{UC10.1});
        \item l'utente inserisce il nome del destinatario dell'evento (\hyperref[sec: UC10.2]{UC10.2});
        \item l'utente conferma la condivisione di un evento (\hyperref[sec: UC10.3]{UC10.3}).
        \end{enumerate}
    \item Estensioni: l'utente inserisce dei valori non validi nei campi e viene mostrato un messaggio d'errore (\hyperref[sec: UC10.3.1]{UC10.3.1});
    \item Precondizioni: l'utente ha acceduto al calendario e vuole condividere un evento;
    \item Postcondizioni: l'utente ha condiviso un evento.
\end{itemize}

\subsubsection{UC10.1 - Scelta dell'evento da condividere} \label{sec: UC10.1}
\begin{itemize}
    \item Attore: utente;
    \item Descrizione: l'utente deve scegliere l'evento del calendario che vuole \par condividere;
    \item Scenario:
        \begin{enumerate}
        \item l'utente seleziona l'evento.
        \end{enumerate}
    
    \item Precondizioni: l'utente vuole condividere un evento;
    \item Postcondizioni: l'utente ha scelto l'evento.
\end{itemize}


\subsubsection{UC10.2 - Scelta del destinatario} \label{sec: UC10.2}
\begin{itemize}
    \item Attore: utente;
    \item Descrizione: l'utente inserisce il destinatario con cui desidera condividere l'evento;
    \item Scenario:
        \begin{enumerate}
        \item l'utente digita il nome dell'utente destinatario.
        \end{enumerate}
    
    \item Precondizioni: l'utente vuole condividere un evento;
    \item Postcondizioni: l'utente ha scelto il destinatario.
\end{itemize}


\subsubsection{UC10.3 - Conferma condivisione evento} \label{sec: UC10.3}
\begin{itemize}
    \item Attore: utente;
    \item Descrizione: l'utente conferma la condivisione di un evento;
    \item Scenario:
        \begin{enumerate}
        \item l'utente deve confermare la condivisione dell'evento scelto con l'utente scelto.
        \end{enumerate}
    
    \item Precondizioni: l'utente vuole condividere un evento;
    \item Postcondizioni: l'utente confermato la condivisione dell'evento.
\end{itemize}

\subsubsection{UC10.3.1 - Visualizzazione errore condivisione evento} \label{sec: UC10.3.1}
\begin{itemize}
    \item Attore: utente;
    \item Descrizione: l'utente deve ricevere un errore a seguito di dati non validi inseriti durante la procedura per condividere un evento;
    \item Scenario:
        \begin{enumerate}
        \item l'utente visualizza il messaggio d'errore.
        \end{enumerate}
    
    \item Precondizioni:l'utente vuole condividere un evento e inserisce dei dati non validi;
    \item Postcondizioni: l'utente ha ricevuto un messaggio d'errore.
\end{itemize}


\subsection{UC11 - Eliminazione evento nel calendario}
\begin{itemize}
    \item Attore: utente;
    \item Descrizione: l'utente deve poter eliminare eventi nel calendario;
    \item Scenario principale:
        \begin{enumerate}
        \item l'utente sceglie l'evento da eliminare (\hyperref[sec: UC11.1]{UC11.1});
        \item l'utente conferma l'eliminazione (\hyperref[sec: UC11.2]{UC11.2});
        \item l'utente riceve il messaggio di eliminazione di un evento (\hyperref[sec: UC11.3]{UC11.3}).
        \end{enumerate}
    \item Precondizioni: l'utente ha acceduto al calendario e vuole eliminare un evento;
    \item Postcondizioni: l'evento scelto è stato eliminato.
\end{itemize}

\subsubsection{UC11.1 - Scelta dell'evento da eliminare} \label{sec: UC11.1}
\begin{itemize}
    \item Attore: utente;
    \item Descrizione: l'utente deve poter scegliere l'evento che vuole eliminare dal \par calendario, che sia normale o condiviso;
    \item Scenario:
        \begin{enumerate}
        \item l'utente seleziona l'evento.
        \end{enumerate}
    
    \item Precondizioni: l'utente vuole eliminare un evento;
    \item Postcondizioni: l'utente ha scelto l'evento da eliminare.
\end{itemize}


\subsubsection{UC11.2 - Conferma eliminazione evento} \label{sec: UC11.2}
\begin{itemize}
    \item Attore: utente;
    \item Descrizione: l'utente deve confermare l'eliminazione dell'evento scelto;
    \item Scenario:
        \begin{enumerate}
        \item l'utente conferma l'eliminazione.
        \end{enumerate}
    
    \item Precondizioni: l'utente vuole eliminare un evento;
    \item Postcondizioni: l'utente ha confermato l'eliminazione dell'evento.
\end{itemize}


\subsubsection{UC11.3 - Ricezione messaggio eliminazione} \label{sec: UC11.3}
\begin{itemize}
    \item Attore: utente;
    \item Descrizione: l'utente deve ricevere un messaggio che garantisce il buon esito dell'operazione;
    \item Scenario:
        \begin{enumerate}
        \item l'utente riceve il messaggio informativo.
        \end{enumerate}
    
    \item Precondizioni: l'utente vuole eliminare un evento;
    \item Postcondizioni: l'utente ha ricevuto il messaggio di corretta eliminazione.
\end{itemize}


\subsection{UC12 - Eliminazione evento condiviso}
\begin{itemize}
    \item Attore: utente;
    \item Descrizione: l'utente deve poter eliminare eventi condivisi;
    \item Scenario principale:
        \begin{enumerate}
        \item l'utente sceglie l'evento condiviso da eliminare (\hyperref[sec: UC12.1]{UC12.1});
        \item l'utente conferma l'eliminazione (\hyperref[sec: UC12.2]{UC12.2});
        \item l'utente riceve il messaggio di eliminazione dell'evento condiviso (\hyperref[sec: UC12.3]{UC12.3}).
        \end{enumerate}
    \item Precondizioni: l'utente ha acceduto al calendario e vuole eliminare un evento condiviso esistente;
    \item Postcondizioni: un evento condiviso è stato eliminato dall'utente.
\end{itemize}

\subsubsection{UC12.1 - Scelta dell'evento condiviso da eliminare} \label{sec: UC12.1}
\begin{itemize}
    \item Attore: utente;
    \item Descrizione: l'utente deve poter scegliere l'evento condiviso che vuole \par eliminare;
    \item Scenario:
        \begin{enumerate}
        \item l'utente seleziona l'evento.
        \end{enumerate}
    
    \item Precondizioni: l'utente vuole eliminare un evento condiviso;
    \item Postcondizioni: l'utente ha scelto l'evento condiviso da eliminare.
\end{itemize}


\subsubsection{UC12.2 - Conferma eliminazione evento condiviso} \label{sec: UC12.2}
\begin{itemize}
    \item Attore: utente;
    \item Descrizione: l'utente deve confermare l'eliminazione dell'evento condiviso \par scelto;
    \item Scenario:
        \begin{enumerate}
        \item l'utente conferma l'eliminazione.
        \end{enumerate}
    
    \item Precondizioni: l'utente vuole eliminare un evento condiviso;
    \item Postcondizioni: l'utente ha confermato l'eliminazione dell'evento.
\end{itemize}


\subsubsection{UC12.3 - Ricezione messaggio eliminazione} \label{sec: UC12.3}
\begin{itemize}
    \item Attore: utente;
    \item Descrizione: l'utente deve ricevere un messaggio che garantisce il buon esito dell'operazione;
    \item Scenario:
        \begin{enumerate}
        \item l'utente riceve il messaggio informativo.
        \end{enumerate}
    
    \item Precondizioni: l'utente vuole eliminare un evento condiviso;
    \item Postcondizioni: l'utente ha ricevuto il messaggio di corretta eliminazione.
\end{itemize}



%%%%%%%%%%%%%%% SCENARIO CONTATTI %%%%%%%%%%%%%%%%%%%%%%%

\subsection{UC13 - Creazione nuovo contatto}
\begin{itemize}
    \item Attore: utente;
    \item Descrizione: l'utente deve poter creare nuovi contatti;
    \item Scenario principale:
        \begin{enumerate}
        \item l'utente inserisce il nome del contatto (\hyperref[sec: UC13.1]{UC13.1});
        \item l'utente inserisce l'email del nuovo contatto (\hyperref[sec: UC13.2]{UC13.2});
        \item l'utente conferma la creazione del contatto (\hyperref[sec: UC13.3]{UC13.3}).
        \end{enumerate}
    \item Estensioni: l'utente inserisce dei valori non validi nei campi e viene mostrato un messaggio d'errore (\hyperref[sec: UC13.3.1]{13.3.1});
    \item Precondizioni: l'utente ha acceduto alla rubrica e vuole creare un nuovo contatto;
    \item Postcondizioni: un nuovo contatto è stato creato dall'utente.
\end{itemize}


\subsubsection{UC13.1 - Inserimento nome nuovo contatto} \label{sec: UC13.1}
\begin{itemize}
    \item Attore: utente;
    \item Descrizione: l'utente deve poter inserire il nome del nuovo contatto che vuole aggiungere alla rubrica;
    \item Scenario:
        \begin{enumerate}
        \item l'utente seleziona il campo relativo al nome del contatto;
        \item l'utente digita il nome.
        \end{enumerate}
    
    \item Precondizioni: l'utente vuole creare un nuovo contatto;
    \item Postcondizioni: l'utente ha compilato il campo relativo al nome del contatto.
\end{itemize}


\subsubsection{UC13.2 - Inserimento email contatto} \label{sec: UC13.2}
\begin{itemize}
    \item Attore: utente;
    \item Descrizione: l'utente deve poter inserire la mail del nuovo contatto da creare;
    \item Scenario:
        \begin{enumerate}
        \item l'utente seleziona il campo relativo alla mail;
        \item l'utente inserisce la mail del contatto.
        \end{enumerate}
    
    \item Precondizioni: l'utente vuole creare un nuovo contatto;
    \item Postcondizioni: l'utente ha compilato il campo relativo alla mail del contatto.
\end{itemize}


\subsubsection{UC13.3 - Conferma creazione contatto} \label{sec: UC13.3}
\begin{itemize}
    \item Attore: utente;
    \item Descrizione: l'utente deve poter confermare la creazione del nuovo contatto;
    \item Scenario:
        \begin{enumerate}
        \item l'utente conferma la creazione.
        \end{enumerate}
    
    \item Precondizioni: l'utente vuole creare un nuovo contatto;
    \item Postcondizioni: l'utente ha confermato la creaione di un nuovo contatto.
\end{itemize}


\subsubsection{UC13.3.1 - Visualizzazione errore creazione contatto} \label{sec: UC13.3.1}
\begin{itemize}
    \item Attore: utente;
    \item Descrizione: l'utente deve ricevere un errore a seguito di dati non validi inseriti durante la procedura per creare un nuovo contatto;
    \item Scenario:
        \begin{enumerate}
        \item l'utente visualizza il messaggio d'errore.
        \end{enumerate}
    
    \item Precondizioni: l'utente sta svolgendo la creazione di un contatto e inserisce dei dati non validi;
    \item Postcondizioni: l'utente ha ricevuto un messaggio d'errore.
\end{itemize}


\subsection{UC14 - Condivisione di un contatto}
\begin{itemize}
    \item Attore: utente;
    \item Descrizione: l'utente deve poter condividere un contatto con altri utenti;
    \item Scenario principale:
        \begin{enumerate}
        \item l'utente sceglie il contatto da condividere (\hyperref[sec: UC14.1]{UC14.1});
        \item l'utente inserisce il nome del destinatario del utente (\hyperref[sec: UC14.2]{UC14.2});
        \item l'utente conferma la condivisione del contatto (\hyperref[sec: UC14.3]{UC14.3}).
        \end{enumerate}
    \item Estensioni: l'utente inserisce dei valori non validi nei campi e viene mostrato un messaggio d'errore (\hyperref[sec: UC14.3.1]{UC14.3.1});
    \item Precondizioni: l'utente ha acceduto alla rubrica e vuole condividere un contatto;
    \item Postcondizioni: l'utente ha condiviso un contatto.
\end{itemize}

\subsubsection{UC14.1 - Scelta del contatto da condividere} \label{sec: UC14.1}
\begin{itemize}
    \item Attore: utente;
    \item Descrizione: l'utente deve scegliere il contatto della rubrica che vuole \par condividere;
    \item Scenario:
        \begin{enumerate}
        \item l'utente seleziona il contatto.
        \end{enumerate}
    
    \item Precondizioni: l'utente vuole condividere un contatto;
    \item Postcondizioni: l'utente ha scelto il contatto.
\end{itemize}


\subsubsection{UC14.2 - Scelta del destinatario} \label{sec: UC14.2}
\begin{itemize}
    \item Attore: utente;
    \item Descrizione: l'utente inserisce il destinatario con cui desidera condividere il contatto;
    \item Scenario:
        \begin{enumerate}
        \item l'utente digita il nome dell'utente destinatario.
        \end{enumerate}
    
    \item Precondizioni: l'utente vuole condividere un contatto;
    \item Postcondizioni: l'utente ha scelto il destinatario.
\end{itemize}


\subsubsection{UC14.3 - Conferma condivisione contatto} \label{sec: UC14.3}
\begin{itemize}
    \item Attore: utente;
    \item Descrizione: l'utente conferma la condivisione del contatto;
    \item Scenario:
        \begin{enumerate}
        \item l'utente deve confermare la condivisione del contatto scelto con l'utente scelto.
        \end{enumerate}
    
    \item Precondizioni: l'utente vuole condividere un contatto;
    \item Postcondizioni: l'utente ha confermato la condivisione del contatto.
\end{itemize}

\subsubsection{UC14.3.1 - Visualizzazione errore condivisione evento} \label{sec: UC14.3.1}
\begin{itemize}
    \item Attore: utente;
    \item Descrizione: l'utente deve ricevere un errore a seguito di dati non validi inseriti durante la procedura per condividere un contatto;
    \item Scenario:
        \begin{enumerate}
        \item l'utente visualizza il messaggio d'errore.
        \end{enumerate}
    
    \item Precondizioni: l'utente vuole condividere un contatto e inserisce dei dati non validi;
    \item Postcondizioni: l'utente ha ricevuto un messaggio d'errore.
\end{itemize}



\subsection{UC15 - Eliminazione contatto}
\begin{itemize}
    \item Attore: utente;
    \item Descrizione: l'utente deve poter eliminare un contatto dalla rubrica;
    \item Scenario principale:
        \begin{enumerate}
        \item l'utente sceglie il contatto da eliminare (\hyperref[sec: UC15.1]{UC15.1});
        \item l'utente conferma l'eliminazione (\hyperref[sec: UC15.2]{UC15.2});
        \item l'utente riceve il messaggio di eliminazione di un contatto (\hyperref[sec: UC15.3]{UC15.3}).
        \end{enumerate}
    \item Precondizioni: l'utente ha acceduto alla rubrica e vuole eliminare un contatto;
    \item Postcondizioni: il contatto scelto è stato eliminato.
\end{itemize}

\subsubsection{UC15.1 - Scelta del contatto da eliminare} \label{sec: UC15.1}
\begin{itemize}
    \item Attore: utente;
    \item Descrizione: l'utente deve poter scegliere il contatto che vuole eliminare dalla rubrica;
    \item Scenario:
        \begin{enumerate}
        \item l'utente seleziona il contatto.
        \end{enumerate}
    
    \item Precondizioni: l'utente vuole eliminare un contatto;
    \item Postcondizioni: l'utente ha scelto l'evento da eliminare.
\end{itemize}


\subsubsection{UC15.2 - Conferma eliminazione evento} \label{sec: UC15.2}
\begin{itemize}
    \item Attore: utente;
    \item Descrizione: l'utente deve confermare l'eliminazione del contatto scelto;
    \item Scenario:
        \begin{enumerate}
        \item l'utente conferma l'eliminazione.
        \end{enumerate}
    
    \item Precondizioni: l'utente vuole eliminare un contatto;
    \item Postcondizioni: l'utente ha confermato l'eliminazione del contatto.
\end{itemize}


\subsubsection{UC15.3 - Ricezione messaggio eliminazione} \label{sec: UC15.3}
\begin{itemize}
    \item Attore: utente;
    \item Descrizione: l'utente deve ricevere un messaggio che garantisce il buon esito dell'operazione;
    \item Scenario:
        \begin{enumerate}
        \item l'utente riceve il messaggio informativo.
        \end{enumerate}
    
    \item Precondizioni: l'utente vuole eliminare un contatto;
    \item Postcondizioni: l'utente ha ricevuto il messaggio di corretta eliminazione.
\end{itemize}
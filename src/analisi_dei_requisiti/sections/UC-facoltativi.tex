\subsection{UC9 - Creazione evento nel calendario}
\begin{itemize}
    \item Attore: utente;
    \item Descrizione: l'utente deve poter creare nuovi eventi nel calendario;
    \item Scenario principale:
        \begin{enumerate}
        \item l'utente accede al calendario;
        \item l'utente inserisce il nome dell'evento (\textbf{UC9.1});
        \item l'utente inserisce la data (\textbf{UC9.2});
        \item l'utente inserisce una fascia oraria (\textbf{UC9.3});
        \item l'utente conferma la creazione dell'evento (\textbf{UC9.4}).
        \end{enumerate}
    \item Estensioni: l'utente inserisce dei valori non validi nei campi e viene mostrato un messaggio d'errore (\textbf{UC9.4.1});
    \item Precondizioni: l'utente sta creando un nuovo evento;
    \item Postcondizioni: un nuovo evento è stato creato dall'utente.
\end{itemize}

\subsubsection{UC9.1 - Inserimento nome evento}
\begin{itemize}
    \item Attore: utente;
    \item Descrizione: l'utente deve poter inserire il nome dell'evento che vuole \par aggiungere al calendario;
    \item Scenario:
        \begin{enumerate}
        \item l'utente seleziona il campo relativo al nome dell'evento;
        \item l'utente digita il nome dell'evento che vuole creare.
        \end{enumerate}
    
    \item Precondizioni: l'utente sta svolgendo la creazione di un evento;
    \item Postcondizioni: l'utente ha compilato il campo relativo al nome dell'evento.
\end{itemize}


\subsubsection{UC9.2 - Inserimento data evento}
\begin{itemize}
    \item Attore: utente;
    \item Descrizione: l'utente deve poter inserire la data dell'evento da creare;
    \item Scenario:
        \begin{enumerate}
        \item l'utente seleziona il campo relativo alla data dell'evento;
        \item l'utente inserisce la data per l'evento che vuole creare.
        \end{enumerate}
    
    \item Precondizioni: l'utente sta svolgendo la creazione di un evento;
    \item Postcondizioni: l'utente ha compilato il campo relativo alla data dell'evento.
\end{itemize}


\subsubsection{UC9.3 - Inserimento ora evento}
\begin{itemize}
    \item Attore: utente;
    \item Descrizione: l'utente deve poter inserire l'orario dell'evento;
    \item Scenario:
        \begin{enumerate}
        \item l'utente seleziona il campo relativo alla ora dell'evento;
        \item l'utente inserisce la fascai oraria dell'evento che vuole creare.
        \end{enumerate}
    
    \item Precondizioni: l'utente sta svolgendo la creazione di un evento;
    \item Postcondizioni: l'utente ha compilato il campo relativo all'orario dell'evento.
\end{itemize}

\subsubsection{UC9.4 - Conferma creazione evento}
\begin{itemize}
    \item Attore: utente;
    \item Descrizione: l'utente deve poter confermare la creazione dell'evento;
    \item Scenario:
        \begin{enumerate}
        \item l'utente conferma la creazione dell'evento.
        \end{enumerate}
    
    \item Precondizioni: l'utente sta svolgendo la creazione di un evento;
    \item Postcondizioni: l'utente ha confermato la creazione dell'evento.
\end{itemize}

\subsubsection{UC9.4.1 - Visualizzazione errore creazione evento}
\begin{itemize}
    \item Attore: utente;
    \item Descrizione: l'utente deve ricevere un errore a seguito di dati non validi inseriti durante la procedura per creare un evento;
    \item Scenario:
        \begin{enumerate}
        \item l'utente visualizza il messaggio d'errore.
        \end{enumerate}
    
    \item Precondizioni: l'utente sta svolgendo la creazione di un evento e inserisce dei dati non validi;
    \item Postcondizioni: l'utente ha ricevuto un messaggio d'errore.
\end{itemize}
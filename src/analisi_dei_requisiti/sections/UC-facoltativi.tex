
%%%%%%%%%%%%%%% SCENARIO EVENTI %%%%%%%%%%%%%%%%%%%%%%%

\subsection{UC 12 - Creazione evento nel calendario}
\begin{itemize}
    \item Attore: utente;
    \item Descrizione: l'utente deve poter creare nuovi eventi nel calendario;
    \item Scenario principale:
        \begin{enumerate}
        \item l'utente inserisce il nome dell'evento (\hyperref[sec: UC 12.1]{UC 12.1});
        \item l'utente inserisce la data (\hyperref[sec: UC 12.2]{UC 12.2});
        \item l'utente inserisce una fascia oraria (\hyperref[sec: UC 12.3]{UC 12.3});
        \item l'utente conferma la creazione dell'evento (\hyperref[sec: UC 12.4]{UC 12.4}).
        \end{enumerate}
    \item Estensioni: l'utente inserisce dei valori non validi nei campi e viene mostrato un messaggio d'errore (\hyperref[sec: UC 12.4.1]{UC 12.4.1});
    \item Precondizioni: l'utente ha acceduto al calendario e vuole creare un nuovo evento;
    \item Postcondizioni: un nuovo evento è stato creato dall'utente.
\end{itemize}

\subsubsection{UC 12.1 - Inserimento nome evento} \label{sec: UC 12.1}
\begin{itemize}
    \item Attore: utente;
    \item Descrizione: l'utente deve poter inserire il nome dell'evento che vuole \par aggiungere al calendario;
    \item Scenario:
        \begin{enumerate}
        \item l'utente seleziona il campo relativo al nome dell'evento;
        \item l'utente digita il nome dell'evento che vuole creare.
        \end{enumerate}
    
    \item Precondizioni: l'utente vuole creare un nuovo evento;
    \item Postcondizioni: l'utente ha compilato il campo relativo al nome dell'evento.
\end{itemize}


\subsubsection{UC 12.2 - Inserimento data evento} \label{sec: UC 12.2}
\begin{itemize}
    \item Attore: utente;
    \item Descrizione: l'utente deve poter inserire la data dell'evento da creare;
    \item Scenario:
        \begin{enumerate}
        \item l'utente seleziona il campo relativo alla data dell'evento;
        \item l'utente inserisce la data per l'evento che vuole creare.
        \end{enumerate}
    
    \item Precondizioni: l'utente vuole creare un nuovo evento;
    \item Postcondizioni: l'utente ha compilato il campo relativo alla data dell'evento.
\end{itemize}


\subsubsection{UC 12.3 - Inserimento ora evento} \label{sec: UC 12.3}
\begin{itemize}
    \item Attore: utente;
    \item Descrizione: l'utente deve poter inserire l'orario dell'evento;
    \item Scenario:
        \begin{enumerate}
        \item l'utente seleziona il campo relativo all'orario dell'evento;
        \item l'utente inserisce la fascia oraria dell'evento che vuole creare.
        \end{enumerate}
    
    \item Precondizioni: l'utente vuole creare un nuovo evento;
    \item Postcondizioni: l'utente ha compilato il campo relativo all'orario dell'evento.
\end{itemize}

\subsubsection{UC 12.4 - Conferma creazione evento} \label{sec: UC 12.4}
\begin{itemize}
    \item Attore: utente;
    \item Descrizione: l'utente deve poter confermare la creazione dell'evento;
    \item Scenario:
        \begin{enumerate}
        \item l'utente conferma la creazione dell'evento.
        \end{enumerate}
    
    \item Precondizioni: l'utente vuole creare un nuovo evento;
    \item Postcondizioni: l'utente ha confermato la creazione dell'evento.
\end{itemize}

\subsubsection{UC 12.4.1 - Visualizzazione errore creazione evento} \label{sec: UC 12.4.1}
\begin{itemize}
    \item Attore: utente;
    \item Descrizione: l'utente deve ricevere un errore a seguito di dati non validi inseriti durante la procedura per creare un evento;
    \item Scenario:
        \begin{enumerate}
        \item l'utente visualizza il messaggio d'errore.
        \end{enumerate}
    
    \item Precondizioni: l'utente vuole creare un nuovo evento e inserisce dei dati non validi;
    \item Postcondizioni: l'utente ha ricevuto un messaggio d'errore.
\end{itemize}



\subsection{UC 13 - Condivisione di un evento}
\begin{itemize}
    \item Attore: utente;
    \item Descrizione: l'utente deve poter condividere degli eventi con altri utenti;
    \item Scenario principale:
        \begin{enumerate}
        \item l'utente sceglie l'evento da condividere (\hyperref[sec: UC 13.1]{UC 13.1});
        \item l'utente inserisce il nome del destinatario dell'evento (\hyperref[sec: UC 13.2]{UC 13.2});
        \item l'utente conferma la condivisione di un evento (\hyperref[sec: UC 13.3]{UC 13.3}).
        \end{enumerate}
    \item Estensioni: l'utente inserisce dei valori non validi nei campi e viene mostrato un messaggio d'errore (\hyperref[sec: UC 13.3.1]{UC 13.3.1});
    \item Precondizioni: l'utente ha acceduto al calendario e vuole condividere un evento;
    \item Postcondizioni: l'utente ha condiviso un evento.
\end{itemize}

\subsubsection{UC 13.1 - Scelta dell'evento da condividere} \label{sec: UC 13.1}
\begin{itemize}
    \item Attore: utente;
    \item Descrizione: l'utente deve scegliere l'evento del calendario che vuole \par condividere;
    \item Scenario:
        \begin{enumerate}
        \item l'utente seleziona l'evento.
        \end{enumerate}
    
    \item Precondizioni: l'utente vuole condividere un evento;
    \item Postcondizioni: l'utente ha scelto l'evento.
\end{itemize}


\subsubsection{UC 13.2 - Scelta del destinatario} \label{sec: UC 13.2}
\begin{itemize}
    \item Attore: utente;
    \item Descrizione: l'utente inserisce il destinatario con cui desidera condividere l'evento;
    \item Scenario:
        \begin{enumerate}
        \item l'utente digita il nome dell'utente destinatario.
        \end{enumerate}
    
    \item Precondizioni: l'utente vuole condividere un evento;
    \item Postcondizioni: l'utente ha scelto il destinatario.
\end{itemize}


\subsubsection{UC 13.3 - Conferma condivisione evento} \label{sec: UC 13.3}
\begin{itemize}
    \item Attore: utente;
    \item Descrizione: l'utente conferma la condivisione di un evento;
    \item Scenario:
        \begin{enumerate}
        \item l'utente deve confermare la condivisione dell'evento scelto con il destinatario.
        \end{enumerate}
    
    \item Precondizioni: l'utente vuole condividere un evento;
    \item Postcondizioni: l'utente confermato la condivisione dell'evento.
\end{itemize}

\subsubsection{UC 13.3.1 - Visualizzazione errore condivisione evento} \label{sec: UC 13.3.1}
\begin{itemize}
    \item Attore: utente;
    \item Descrizione: l'utente deve ricevere un errore a seguito di dati non validi inseriti durante la procedura per condividere un evento;
    \item Scenario:
        \begin{enumerate}
        \item l'utente visualizza il messaggio d'errore.
        \end{enumerate}
    
    \item Precondizioni:l'utente vuole condividere un evento e inserisce dei dati non validi;
    \item Postcondizioni: l'utente ha ricevuto un messaggio d'errore.
\end{itemize}


\subsection{UC 14 - Eliminazione evento nel calendario}
\begin{itemize}
    \item Attore: utente;
    \item Descrizione: l'utente deve poter eliminare eventi nel calendario;
    \item Scenario principale:
        \begin{enumerate}
        \item l'utente sceglie l'evento da eliminare (\hyperref[sec: UC 14.1]{UC 14.1});
        \item l'utente conferma l'eliminazione (\hyperref[sec: UC 14.2]{UC 14.2});
        \item l'utente riceve il messaggio di eliminazione di un evento (\hyperref[sec: UC 14.3]{UC 14.3}).
        \end{enumerate}
    \item Precondizioni: l'utente ha acceduto al calendario e vuole eliminare un evento;
    \item Postcondizioni: l'evento scelto è stato eliminato.
\end{itemize}

\subsubsection{UC 14.1 - Scelta dell'evento da eliminare} \label{sec: UC 14.1}
\begin{itemize}
    \item Attore: utente;
    \item Descrizione: l'utente deve poter scegliere l'evento che vuole eliminare dal calendario;
    \item Scenario:
        \begin{enumerate}
        \item l'utente seleziona l'evento.
        \end{enumerate}
    
    \item Precondizioni: l'utente vuole eliminare un evento;
    \item Postcondizioni: l'utente ha scelto l'evento da eliminare.
\end{itemize}


\subsubsection{UC 14.2 - Conferma eliminazione evento} \label{sec: UC 14.2}
\begin{itemize}
    \item Attore: utente;
    \item Descrizione: l'utente deve confermare l'eliminazione dell'evento scelto;
    \item Scenario:
        \begin{enumerate}
        \item l'utente conferma l'eliminazione.
        \end{enumerate}
    
    \item Precondizioni: l'utente vuole eliminare un evento;
    \item Postcondizioni: l'utente ha confermato l'eliminazione dell'evento.
\end{itemize}


\subsubsection{UC 14.3 - Ricezione messaggio eliminazione} \label{sec: UC 14.3}
\begin{itemize}
    \item Attore: utente;
    \item Descrizione: l'utente deve ricevere un messaggio che garantisce il buon esito dell'operazione;
    \item Scenario:
        \begin{enumerate}
        \item l'utente riceve il messaggio informativo.
        \end{enumerate}
    
    \item Precondizioni: l'utente vuole eliminare un evento;
    \item Postcondizioni: l'utente ha ricevuto il messaggio di corretta eliminazione.
\end{itemize}


\subsection{UC 15 - Eliminazione evento condiviso}
\begin{itemize}
    \item Attore: utente;
    \item Descrizione: l'utente deve poter eliminare eventi condivisi;
    \item Scenario principale:
        \begin{enumerate}
        \item l'utente sceglie l'evento condiviso da eliminare (\hyperref[sec: UC 15.1]{UC 15.1});
        \item l'utente conferma l'eliminazione (\hyperref[sec: UC 15.2]{UC 15.2});
        \item l'utente riceve il messaggio d'eliminazione dell'evento condiviso (\hyperref[sec: UC 15.3]{UC 15.3}).
        \end{enumerate}
    \item Precondizioni: l'utente ha acceduto al calendario e vuole eliminare un evento condiviso esistente;
    \item Postcondizioni: un evento condiviso è stato eliminato dall'utente.
\end{itemize}

\subsubsection{UC 15.1 - Scelta dell'evento condiviso da eliminare} \label{sec: UC 15.1}
\begin{itemize}
    \item Attore: utente;
    \item Descrizione: l'utente deve poter scegliere l'evento condiviso che vuole \par eliminare;
    \item Scenario:
        \begin{enumerate}
        \item l'utente seleziona l'evento.
        \end{enumerate}
    
    \item Precondizioni: l'utente vuole eliminare un evento condiviso;
    \item Postcondizioni: l'utente ha scelto l'evento condiviso da eliminare.
\end{itemize}


\subsubsection{UC 15.2 - Conferma eliminazione evento condiviso} \label{sec: UC 15.2}
\begin{itemize}
    \item Attore: utente;
    \item Descrizione: l'utente deve confermare l'eliminazione dell'evento condiviso \par scelto;
    \item Scenario:
        \begin{enumerate}
        \item l'utente conferma l'eliminazione.
        \end{enumerate}
    
    \item Precondizioni: l'utente vuole eliminare un evento condiviso;
    \item Postcondizioni: l'utente ha confermato l'eliminazione dell'evento.
\end{itemize}


\subsubsection{UC 15.3 - Ricezione messaggio eliminazione} \label{sec: UC 15.3}
\begin{itemize}
    \item Attore: utente;
    \item Descrizione: l'utente deve ricevere un messaggio che garantisce il buon esito dell'operazione;
    \item Scenario:
        \begin{enumerate}
        \item l'utente riceve il messaggio informativo.
        \end{enumerate}
    
    \item Precondizioni: l'utente vuole eliminare un evento condiviso;
    \item Postcondizioni: l'utente ha ricevuto il messaggio di corretta eliminazione.
\end{itemize}



%%%%%%%%%%%%%%% SCENARIO CONTATTI %%%%%%%%%%%%%%%%%%%%%%%

\subsection{UC 16 - Creazione nuovo contatto}
\begin{itemize}
    \item Attore: utente;
    \item Descrizione: l'utente deve poter creare nuovi contatti;
    \item Scenario principale:
        \begin{enumerate}
        \item l'utente inserisce il nome del contatto (\hyperref[sec: UC 16.1]{UC 16.1});
        \item l'utente inserisce l'email del nuovo contatto (\hyperref[sec: UC 16.2]{UC 16.2});
        \item l'utente conferma la creazione del contatto (\hyperref[sec: UC 16.3]{UC 16.3}).
        \end{enumerate}
    \item Estensioni: l'utente inserisce dei valori non validi nei campi e viene mostrato un messaggio d'errore (\hyperref[sec: UC 16.3.1]{UC 16.3.1});
    \item Precondizioni: l'utente ha acceduto alla rubrica e vuole creare un nuovo contatto;
    \item Postcondizioni: un nuovo contatto è stato creato dall'utente.
\end{itemize}


\subsubsection{UC 16.1 - Inserimento nome nuovo contatto} \label{sec: UC 16.1}
\begin{itemize}
    \item Attore: utente;
    \item Descrizione: l'utente deve poter inserire il nome del nuovo contatto che vuole aggiungere alla rubrica;
    \item Scenario:
        \begin{enumerate}
        \item l'utente seleziona il campo relativo al nome del contatto;
        \item l'utente digita il nome.
        \end{enumerate}
    
    \item Precondizioni: l'utente vuole creare un nuovo contatto;
    \item Postcondizioni: l'utente ha compilato il campo relativo al nome del contatto.
\end{itemize}


\subsubsection{UC 16.2 - Inserimento email contatto} \label{sec: UC 16.2}
\begin{itemize}
    \item Attore: utente;
    \item Descrizione: l'utente deve poter inserire la mail del nuovo contatto da creare;
    \item Scenario:
        \begin{enumerate}
        \item l'utente seleziona il campo relativo alla mail;
        \item l'utente inserisce la mail del contatto.
        \end{enumerate}
    
    \item Precondizioni: l'utente vuole creare un nuovo contatto;
    \item Postcondizioni: l'utente ha compilato il campo relativo alla mail del contatto.
\end{itemize}


\subsubsection{UC 16.3 - Conferma creazione contatto} \label{sec: UC 16.3}
\begin{itemize}
    \item Attore: utente;
    \item Descrizione: l'utente deve poter confermare la creazione del nuovo contatto;
    \item Scenario:
        \begin{enumerate}
        \item l'utente conferma la creazione.
        \end{enumerate}
    
    \item Precondizioni: l'utente vuole creare un nuovo contatto;
    \item Postcondizioni: l'utente ha confermato la creazione di un nuovo contatto.
\end{itemize}


\subsubsection{UC 16.3.1 - Visualizzazione errore creazione contatto} \label{sec: UC 16.3.1}
\begin{itemize}
    \item Attore: utente;
    \item Descrizione: l'utente deve ricevere un errore a seguito di dati non validi inseriti durante la procedura per creare un nuovo contatto;
    \item Scenario:
        \begin{enumerate}
        \item l'utente visualizza il messaggio d'errore.
        \end{enumerate}
    
    \item Precondizioni: l'utente sta svolgendo la creazione di un contatto e inserisce dei dati non validi;
    \item Postcondizioni: l'utente ha ricevuto un messaggio d'errore.
\end{itemize}


\subsection{UC17 - Condivisione di un contatto}
\begin{itemize}
    \item Attore: utente;
    \item Descrizione: l'utente deve poter condividere un contatto con altri utenti;
    \item Scenario principale:
        \begin{enumerate}
        \item l'utente sceglie il contatto da condividere (\hyperref[sec: UC 17.1]{UC 17.1});
        \item l'utente inserisce il nome del destinatario del utente (\hyperref[sec: UC 17.2]{UC 17.2});
        \item l'utente conferma la condivisione del contatto (\hyperref[sec: UC 17.3]{UC 17.3}).
        \end{enumerate}
    \item Estensioni: l'utente inserisce dei valori non validi nei campi e viene mostrato un messaggio d'errore (\hyperref[sec: UC 17.3.1]{UC 17.3.1});
    \item Precondizioni: l'utente ha acceduto alla rubrica e vuole condividere un contatto;
    \item Postcondizioni: l'utente ha condiviso un contatto.
\end{itemize}

\subsubsection{UC 17.1 - Scelta del contatto da condividere} \label{sec: UC 17.1}
\begin{itemize}
    \item Attore: utente;
    \item Descrizione: l'utente deve scegliere il contatto della rubrica che vuole \par condividere;
    \item Scenario:
        \begin{enumerate}
        \item l'utente seleziona il contatto.
        \end{enumerate}
    
    \item Precondizioni: l'utente vuole condividere un contatto;
    \item Postcondizioni: l'utente ha scelto il contatto.
\end{itemize}


\subsubsection{UC 17.2 - Scelta del destinatario} \label{sec: UC 17.2}
\begin{itemize}
    \item Attore: utente;
    \item Descrizione: l'utente inserisce il destinatario con cui desidera condividere il contatto;
    \item Scenario:
        \begin{enumerate}
        \item l'utente digita il nome dell'utente destinatario.
        \end{enumerate}
    
    \item Precondizioni: l'utente vuole condividere un contatto;
    \item Postcondizioni: l'utente ha scelto il destinatario.
\end{itemize}


\subsubsection{UC 17.3 - Conferma condivisione contatto} \label{sec: UC 17.3}
\begin{itemize}
    \item Attore: utente;
    \item Descrizione: l'utente conferma la condivisione del contatto;
    \item Scenario:
        \begin{enumerate}
        \item l'utente deve confermare la condivisione del contatto scelto con l'utente scelto.
        \end{enumerate}
    
    \item Precondizioni: l'utente vuole condividere un contatto;
    \item Postcondizioni: l'utente ha confermato la condivisione del contatto.
\end{itemize}

\subsubsection{UC 17.3.1 - Visualizzazione errore condivisione contatto} \label{sec: UC 17.3.1}
\begin{itemize}
    \item Attore: utente;
    \item Descrizione: l'utente deve ricevere un errore a seguito di dati non validi inseriti durante la procedura per condividere un contatto;
    \item Scenario:
        \begin{enumerate}
        \item l'utente visualizza il messaggio d'errore.
        \end{enumerate}
    
    \item Precondizioni: l'utente vuole condividere un contatto e inserisce dei dati non validi;
    \item Postcondizioni: l'utente ha ricevuto un messaggio d'errore.
\end{itemize}



\subsection{UC 18 - Eliminazione contatto}
\begin{itemize}
    \item Attore: utente;
    \item Descrizione: l'utente deve poter eliminare un contatto dalla rubrica;
    \item Scenario principale:
        \begin{enumerate}
        \item l'utente sceglie il contatto da eliminare (\hyperref[sec: UC 18.1]{UC 18.1});
        \item l'utente conferma l'eliminazione (\hyperref[sec: UC 18.2]{UC 18.2});
        \item l'utente riceve il messaggio di eliminazione di un contatto (\hyperref[sec: UC 18.3]{UC 18.3}).
        \end{enumerate}
    \item Precondizioni: l'utente ha acceduto alla rubrica e vuole eliminare un contatto;
    \item Postcondizioni: il contatto scelto è stato eliminato.
\end{itemize}

\subsubsection{UC 18.1 - Scelta del contatto da eliminare} \label{sec: UC 18.1}
\begin{itemize}
    \item Attore: utente;
    \item Descrizione: l'utente deve poter scegliere il contatto che vuole eliminare dalla rubrica;
    \item Scenario:
        \begin{enumerate}
        \item l'utente seleziona il contatto.
        \end{enumerate}
    
    \item Precondizioni: l'utente vuole eliminare un contatto;
    \item Postcondizioni: l'utente ha scelto l'evento da eliminare.
\end{itemize}


\subsubsection{UC 18.2 - Conferma eliminazione evento} \label{sec: UC 18.2}
\begin{itemize}
    \item Attore: utente;
    \item Descrizione: l'utente deve confermare l'eliminazione del contatto scelto;
    \item Scenario:
        \begin{enumerate}
        \item l'utente conferma l'eliminazione.
        \end{enumerate}
    
    \item Precondizioni: l'utente vuole eliminare un contatto;
    \item Postcondizioni: l'utente ha confermato l'eliminazione del contatto.
\end{itemize}


\subsubsection{UC 18.3 - Ricezione messaggio eliminazione} \label{sec: UC 18.3}
\begin{itemize}
    \item Attore: utente;
    \item Descrizione: l'utente deve ricevere un messaggio che garantisce il buon esito dell'operazione;
    \item Scenario:
        \begin{enumerate}
        \item l'utente riceve il messaggio informativo.
        \end{enumerate}
    
    \item Precondizioni: l'utente vuole eliminare un contatto;
    \item Postcondizioni: l'utente ha ricevuto il messaggio di corretta eliminazione.
\end{itemize}

%%%%%%%%%%%%%%% TESTING %%%%%%%%%%%%%%%%%%%%%%

\subsection{UC 19 - Stress test}
\begin{itemize}
    \item Attore: developer Zextras;
    \item Descrizione: l'utente deve poter fare degli stess test per misurare le performance del prodotto;
    \item Scenario principale:
        \begin{enumerate}
        \item l'utente avvia uno o una serie di stress test. 
        \end{enumerate}
    \item Precondizioni: l'utente vuole avviare degli stess test;
    \item Postcondizioni: l'utente visualizza i risultati dei test.
\end{itemize}
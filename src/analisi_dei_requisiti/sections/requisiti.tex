\section{Requisiti}

\subsection{Requisiti funzionali}

\begin{table}[H]
    \centering
    \rowcolors{2}{cyan!80!black!30!}{cyan!80!black!20!}
    \begin{tabular}{*{1}{>{\centering\arraybackslash}p{2cm}}*{1}{>{\centering\arraybackslash}p{3cm}}p{5cm}*{1}{>{\centering\arraybackslash}p{3cm}}}
    \toprule
    \rowcolor{gray!20} \textbf{Codice} & \textbf{Importanza} & \textbf{Descrizione} & \textbf{Fonte}
    \\\midrule 
    RF & Obbligatorio & Invio di un'email & Capitolato
    \\\midrule
    RF & Obbligatorio & Ricezione di un'email & Capitolato
    \\\midrule
    RF & Obbligatorio & Creazione di una cartella & Capitolato
    \\\midrule
    RF & Obbligatorio & Eliminazione di una cartella & Capitolato
    \\\midrule
    RF & Obbligatorio & Modifica di una cartella & Capitolato
    \\\midrule %is that obbligatorio???
    RF & Obbligatorio & Creazione condivisione di una cartella & Capitolato
    \\\midrule 
    RF & Obbligatorio & Eliminazione della condivisione di una cartella & Capitolato
    \\\midrule
    RF & Obbligatorio & Modifica condivisione di una cartella & Capitolato
    \\\midrule %is that obbligatorio???
    RF & Obbligatorio &  ??? & Capitolato
    \\\midrule
    RF & Opzionale & Creazione di un contatto & Capitolato
    \\\midrule
    RF & Opzionale & Eliminazione di un contatto & Capitolato
    \\\midrule
    RF & Opzionale & Modifica di un contatto & Capitolato
    \\\midrule %is that obbligatorio???
    RF & Opzionale & Creazione di un calendario & Capitolato
    \\\midrule
    RF & Opzionale & Eliminazione di un calendario & Capitolato
    \\\midrule
    RF & Opzionale & Modifica di un calendario & Capitolato
    \\\midrule %is that obbligatorio???
    RF & Opzionale & Creazione di un evento nel calendario & Capitolato
    \\\midrule
    RF & Opzionale & Eliminazione di un evento nel calendario & Capitolato
    \\\midrule
    RF & Opzionale & Modifica di un evento nel calendario & Capitolato 
    \\\midrule %is that obbligatorio???
    \\\bottomrule
    \end{tabular}
\caption{requisiti funzionali}
\label{tab:req-fun}
\end{table}


\subsection{Requisiti di vincolo}
\begin{table}[H]
    \centering
    \rowcolors{2}{cyan!80!black!30!}{cyan!80!black!20!}
    \begin{tabular}{*{1}{>{\centering\arraybackslash}p{2cm}}*{1}{>{\centering\arraybackslash}p{3cm}}p{5cm}*{1}{>{\centering\arraybackslash}p{3cm}}}
    \toprule
    \rowcolor{gray!20} \textbf{Codice} & \textbf{Importanza} & \textbf{Descrizione} & \textbf{Fonte}
    \\\midrule 
    RV & Obbligatorio & usare JMAP & Capitolato
    \\\midrule
    RV & Obbligatorio & libreria al link https://github.com/iNPUTmice/jmap oppure una tra quelle presenti al link https://jmap.io/software.html & Capitolato
    \\\midrule
    RV & Obbligatorio & Il servizio deve essere sviluppato in un sistema container & Capitolato
    \\\midrule
    RV & Obbligatorio & Il servizio sviluppato deve essere scalabile mediante l’inizializzazione di più nodi stateless & Capitolato
    \\\midrule
    RV & Obbligatorio & Il database deve essere unico & Verbale .....
    \\\midrule
    RV & Opzionale & Stress test & Capitolato
    \\\midrule
    RV & Opzionale & - & Capitolato
    \\\bottomrule
    \end{tabular}

\caption{requisiti di vincolo}
\label{tab:req-vin}
\end{table}
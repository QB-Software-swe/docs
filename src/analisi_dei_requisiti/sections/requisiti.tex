%LTeX
\section{Requisiti}
    \subsection{Requisiti funzionali}


    \rowcolors{2}{cyan!80!black!30!}{cyan!80!black!20!}
    \begin{longtable}{*{1}{>{\centering\arraybackslash}p{1.5cm}}*{1}{>{\centering\arraybackslash}p{2.5cm}}p{6cm}*{1}{>{\centering\arraybackslash}p{3cm}}}
    \caption{Requisiti funzionali}
    \label{tab:req-fun}
    \\\hline
    \rowcolor{gray!20} \textbf{Codice} & \textbf{Importanza} & \textbf{Descrizione} & \textbf{Fonte}
    \\\hline 
    RF & Obbligatorio & È necessario che il MUA abbia la capacità di inviare un'email & UC 1
    \\\hline 
    RF & Obbligatorio & È necessario che il MUA abbia la capacità di trasmettere l'id dell'account durante l'attività di invio e-mail & UC 1.1
    \\\hline
    RF & Obbligatorio & È necessario che il MUA abbia la capacità di trasmettere l'id dell'e-mail durante l'attività di invio e-mail & UC 1.2
    \\\hline
    RF & Obbligatorio & È necessario che il MUA visualizzi un messaggio d'errore se l'id e-mail non è valido durante l'attività di invio e-mail & UC 1.5
    \\\hline
    RF & Obbligatorio & È necessario che il MUA abbia la capacità di trasmettere il destinatario dell'e-mail durante l'attività di invio e-mail & UC 1.3
    \\\hline
    RF & Obbligatorio & È necessario che il MUA visualizzi un messaggio d'errore se il destinatario non è valido durante l'attività di invio e-mail & UC 1.6
    \\\hline
    RF & Obbligatorio & È necessario che il MUA visualizzi un messaggio d'errore se il destinatario non è esiste durante l'attività di invio e-mail & UC 1.7
    \\\hline
    RF & Obbligatorio & È necessario che il MUA visualizzi un messaggio d'errore se sono presenti troppi destinatari durante l'attività di invio e-mail & UC 1.8
    \\\hline
    RF & Obbligatorio & È necessario che il MUA abbia la capacità di trasmettere il mittente dell'e-mail durante l'attività di invio e-mail & UC 1.4
    \\\hline
    RF & Obbligatorio & È necessario che il MUA visualizzi un messaggio d'errore se il mittente non è valido durante l'attività di invio e-mail & UC 1.9
    \\\hline
    RF & Obbligatorio & È necessario che il MUA abbia la capacità di creare un'e-mail & UC 2
    \\\hline
    RF & Obbligatorio & È necessario che il MUA abbia la capacità di trasmettere il destinatario dell'e-mail durante l'attività di creazione e-mail & UC 2.1
    \\\hline
    RF & Obbligatorio & È necessario che il MUA visualizzi un messaggio d'errore se il destinatario non è valido durante l'attività di creazione e-mail & UC 2.6
    \\\hline
    RF & Obbligatorio & È necessario che il MUA abbia la capacità di trasmettere il mittente dell'e-mail durante l'attività di creazione e-mail & UC 2.2
    \\\hline
    RF & Obbligatorio & È necessario che il MUA visualizzi un messaggio d'errore se il mittente non è valido durante l'attività di creazione e-mail & UC 2.7
    \\\hline
    RF & Obbligatorio & È necessario che il MUA abbia la capacità di trasmettere l'oggetto dell'e-mail durante l'attività di creazione e-mail & UC 2.3
    \\\hline
    RF & Obbligatorio & È necessario che il MUA abbia la capacità di trasmettere il corpo dell'e-mail durante l'attività di creazione e-mail & UC 2.4
    \\\hline
    RF & Obbligatorio & È necessario che il MUA abbia la capacità di trasmettere la cartella dell'e-mail durante l'attività di creazione e-mail & UC 2.5
    \\\hline
    RF & Obbligatorio & È necessario che il MUA abbia la capacità di creare una cartella & UC 3
    \\\hline
    RF & Obbligatorio & È necessario che il MUA abbia la capacità di trasmettere il nome della cartella durante l'attività di creazione cartella & UC 3.1
    \\\hline
    RF & Obbligatorio & È necessario che il MUA visualizzi un messaggio d'errore se si crea una cartella duplicata & UC 3.3
    \\\hline
    RF & Obbligatorio & È necessario che il MUA visualizzi un messaggio d'errore se si crea una cartella con un nome non valido & UC 3.4
    \\\hline
    RF & Obbligatorio & È necessario che il MUA abbia la capacità di trasmettere l'id della cartella genitore durante l'attività di creazione cartella & UC 3.2
    \\\hline
    RF & Opzionale & È necessario che il MUA abbia la capacità di creare un contatto & UC 4
    \\\hline
    RF & Opzionale & È necessario che il MUA abbia la capacità di trasmettere il nome del contatto durante l'attività di creazione contatto & UC 4.1
    \\\hline
    RF & Opzionale & È necessario che il MUA visualizzi un messaggio d'errore se i dati trasmessi durante l'attività di creazione contatto non sono validi & UC 4.3
    \\\hline
    RF & Opzionale & È necessario che il MUA abbia la capacità di trasmettere l'indirizzo e-mail del contatto durante l'attività di creazione contatto & UC 4.2
    \\\hline
    \end{longtable}


\subsection{Requisiti di vincolo}
\begin{table}[H]
    \centering
    \rowcolors{2}{cyan!80!black!30!}{cyan!80!black!20!}
    \begin{tabular}{*{1}{>{\centering\arraybackslash}p{2cm}}*{1}{>{\centering\arraybackslash}p{3cm}}p{5cm}*{1}{>{\centering\arraybackslash}p{3cm}}}
    \toprule
    \rowcolor{gray!20} \textbf{Codice} & \textbf{Importanza} & \textbf{Descrizione} & \textbf{Fonte}
    \\\midrule 
    RV & Obbligatorio & È necessario usare JMAP come protocollo per accedere al server & Capitolato
    \\\midrule
    RV & Obbligatorio & L'implementazione di JMAP deve essere fatta utilizzando una delle librerie al link https://github.com/iNPUTmice/jmap oppure una tra quelle presenti al link https://jmap.io/software.html & Capitolato
    \\\midrule
    RV & Obbligatorio & Il servizio deve essere sviluppato in un sistema container & Capitolato
    \\\midrule
    RV & Obbligatorio & Il servizio sviluppato deve essere scalabile mediante l’inizializzazione di più nodi stateless & Capitolato
    \\\midrule
    RV & Obbligatorio & Il client deve essere uno tra quelli presenti al link https://jmap.io/software.html  & Capitolato
    \\\midrule
    RV & Obbligatorio & Il database deve essere unico & Verbale .....
    \\\midrule
    RV & Desiderabile & Stress test per testare il protocollo JAMP & Capitolato
    \\\midrule
    RV & Desiderabile & Java come linguaggio di programmazione & Capitolato
    \\\midrule
    RV & Opzionale & - & Capitolato
    \\\bottomrule
    \end{tabular}

\caption{requisiti di vincolo}
\label{tab:req-vin}
\end{table}

\subsection{Requisiti di qualità}
\begin{table}[H]
    \centering
    \rowcolors{2}{cyan!80!black!30!}{cyan!80!black!20!}
    \begin{tabular}{*{1}{>{\centering\arraybackslash}p{2cm}}*{1}{>{\centering\arraybackslash}p{3cm}}p{5cm}*{1}{>{\centering\arraybackslash}p{3cm}}}
    \toprule
    \rowcolor{gray!20} \textbf{Codice} & \textbf{Importanza} & \textbf{Descrizione} & \textbf{Fonte}
    \\\midrule 
    RQ & Obbligatorio & Tutte le norme delineate in Norme di Progetto v1.0.0 devono essere osservate & Interno
    \\\midrule
    RQ & Obbligatorio & Tutti i vincoli e le metriche indicati nel Piano di Qualifica v1.0.0 devono essere rispettati & Interno
    \\\midrule
    RQ & Obbligatorio & Deve essere consegnato il Glossario & Interno

    \\\bottomrule
    \end{tabular}

\caption{Requisiti di qualità}
\label{tab:req-vin}
\end{table}
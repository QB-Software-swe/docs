\subsection{Gestione organizzativa}\label{sec:organizationalP}
    In questa sezione vengono esposte le norme relative l'organizzazione e il coordinamento delle attività interne ed esterne del gruppo, l'assegnazione dei ruoli e relativi compiti. \\
    Come stabilito dallo standard ISO/IEC 12207:1997 \cite{bib:ISO12207_1997} il processo di gestione identifica le seguenti attività:
    \begin{enumerate}
        \item Inizializzazione e definizione dello scopo \ref{sec:scope};
        \item pianificazione \ref{sec:planning};
        \item esecuzione e controllo \ref{sec:execution};
        \item revisione e valutazione \ref{sec:review};
        \item chiusura \ref{sec:closure}.
    \end{enumerate}
    \subsubsection{Inizializzazione e definizione dello scopo}\label{sec:scope}
        Il responsabile di progetto si impegna a:
        \begin{itemize}
            \item Assegnare i ruoli e i relativi compiti ai membri del gruppo;
            \item amministrare gli strumenti di coordinamento concordati;
            \item stimare i rischi;
            \item gestire gli imprevisti;
            \item gestire le comunicazioni interne ed esterne;
            \item calcolare i preventivi:
            \begin{enumerate}
                \item relativo le ore e i costi divisi per ruolo;
                \item a finire, relativo il consuntivo di ogni periodo;
            \end{enumerate}
            \item determinare quando un processo, attività o task è giunto al termine e approvarlo. 
        \end{itemize}
    \subsubsection{Pianificazione}\label{sec:planning}
        Il processo ha l'obiettivo di garantire:
        \begin{itemize}
            \item Pianificazione delle attività in modo preciso e definizione delle loro scadenze;
            \item monitoraggio dei progressi del team rispetto il prodotto finale;
            \item assegnazione dinamica ed efficace di ruoli e compiti perseguendo economicità;
            \item facilitazione delle comunicazioni interne ed esterne;
            \item stima della quantità e del tipo di risorse necessarie per ultimare il progetto nel rispetto del budget rimanente.
        \end{itemize}
        Ogni membro del team è tenuto a coprire almeno una volta ognuno dei ruoli elencati qui di seguito, come previsto dal \emph{Piano dei costi}.\\
        I ruoli predisposti per il progetto QB Software prevedono:
        
        \paragraph{Responsabile}
            Svolge la funzione di coordinatore del gruppo e di rappresentante del team nei confronti di committente e proponente per tutta la durata del progetto. Ha molteplici responsabilità quali:
            \begin{itemize}
                \item Predisposizione e gestione delle risorse;
                \item coordinamento dei membri del team;
                \item verifica dello stato di attività e processi;
                \item relazioni con le figure esterne al team;
                \item approvazione dei documenti di progetto.
            \end{itemize}
        \paragraph{Amministratore}
            Definisce, controlla e amministra l'ambiente di lavoro e le risorse messe a disposizione per tutto il periodo di sviluppo del progetto.
            Deve accertarsi che i mezzi messi a disposizione perseguano produttività, garantendo allo stesso tempo qualità ed economicità. \\
            Redige le \emph{Norme di progetto} ed sua responsabilità verificare che i membri del team le seguano. \\
            Ed \`e sua responsabilità:
            \begin{itemize}
                \item Amministrare infrastrutture, strumenti e documentazione;
                \item gestione degli imprevisti legati alla gestione dei processi;
                \item redigere e mantenere aggiornata la documentazione e il versionamento della stessa;
                \item gestire la configurazione del prodotto.
            \end{itemize}
        \paragraph{Analista}
            \`E una figura molto competente, a cui ci si affida per scomporre e approfondire le esigenze del committente. Il suo ruolo è fondamentale nelle prime fasi del progetto, in particolare per la stesura del file \emph{Analisi dei requisiti} che sarà una serie di specifiche tecniche che saranno fondamentali per le fasi successive di sviluppo. \\
            Si occupa di:
            \begin{itemize}
                \item Mediare fra proponenti/committenti e sviluppatori;
                \item studia le necessità dei proponenti definendo problemi, obiettivi e requisiti soluzione;
                \item etichetta i requisiti in: impliciti/espliciti e opzionali/obbligatori;
                \item redige i file \emph{Studio di fattibilità} e \emph{Analisi dei requisiti}.
            \end{itemize}
        \paragraph{Progettista}
            Ha l'onere di sviluppare una soluzione che soddisfi in maniera accettabile i vincoli dati dall'Analista.
            Effettua scelte tecniche e tecnologiche, segue lo sviluppo, non la manutenzione. \\
            Si occupa di:
            \begin{itemize}
                \item Sviluppare un'architettura robusta seguendo le best practises perseguendo coerenza e consistenza;
                \item ricercare soluzioni efficienti ed efficaci che soddisfino i requisiti nel rispetto dei vincoli dati;
                \item decomporre il sistema in componenti e organizzarne le interazioni fra di essi;
                \item decomporre ruoli e responsabilità in favore di modularizzazione e riutilizzo;
                \item usare soluzioni ottimizzate.
            \end{itemize}
        \paragraph{Programmatore}
            Ha competenze tecniche e appartiene alla categoria pi\`u popolosa del gruppo.
            Si occupa di:
            \begin{itemize}
                \item Codificare la soluzione in modo mantenibile;
                \item partecipare a implementazione e manutenzione del prodotto;
                \item creare test ad hoc per la verifica e valutazione del codice;
                \item redige il manuale utente.
            \end{itemize}
        \paragraph{Verificatore}
            Il suo compito \`e quello di controllare il lavoro svolto dagli altri membri del gruppo. \\
            Inoltre ha l'onere di redigere il \emph{Piano di qualifica}. \\
            Deve assicurarsi che:
            \begin{itemize}
                \item I compiti vengano svolti come da programma, altrimenti offre spunti per le correzioni;
                \item l'esecuzione delle attivit\`a di processo non causi errori.
            \end{itemize}
    \subsubsection{Esecuzione e controllo}\label{sec:execution}
        Prevede che il responsabile:
        \begin{itemize}
            \item Inizi l'implementazione del piano per soddisfare gli obiettivi secondo i vincoli predisposti;
            \item controlli e monitori il processo fornendo report interni ed esterni come contrattato col committente;
            \item investighi, analizzi, riporti e risolva i problemi riscontrati durante l'esecuzione del processo. Questo può portare a cambi di programma ed \`e sua responsabilità assicurarsi che ogni cambiamento sia monitorato e controllato;
            \item riporti secondo i vincoli concordati l'avanzamento o il mancato avanzamento del processo, sia internamente che esternamente.
        \end{itemize}
    \subsubsection{Revisione e valutazione}\label{sec:review}
        Prevede che il responsabile:
        \begin{itemize}
            \item Valuti prodotti e piani d'azione ai fini di soddisfare i vincoli contrattuali;
            \item valuti prodotti software, attivit\`a e task completate durante l'esecuzione, del processo per il raggiungimento dell'obiettivo e il completamento del piano.
        \end{itemize}
    \subsubsection{Chiusura}\label{sec:closure}
        Al termine dello sviluppo dei prodotti, attivit\`a e task, il responsabile determina se il processo si possa definire ultimato verificando che rispetti i vincoli contrattuali. Infine archivia documentazione e prodotti software in un ambiente predefinito da contratto.
        
        
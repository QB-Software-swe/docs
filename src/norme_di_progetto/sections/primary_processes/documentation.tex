\subsection{Processo di fornitura}\label{sec:organizationalP}
Il processo di fornitura comprende le attività e i compiti del fornitore. Il processo ha come obiettivo determinare le procedure e le risorse necessarie per gestire e garantire il prodotto software all'acquirente.
Il fornitore gestisce il processo di fornitura seguendo il processo di gestione organizzativa e il processo di formazione. \\
    Come stabilito dallo standard ISO/IEC 12207:1997 \cite{bib:ISO12207_1997} il processo di fornitura identifica le seguenti attività:
    \begin{enumerate}
        \item avvio \ref{sec:avvio};
        \item preparazione della risposta \ref{sec:preparazione};
        \item contrattazione \ref{sec:contrattazione};
        \item pianificazione \ref{sec:pianificazione};
        \item esecuzione e controllo \ref{sec:esecuzione e controllo};
        \item revisione e valutazione \ref{sec:revisione e valutazione};
        \item consegna e completamento \ref{sec:consegna e completamento}.
    \end{enumerate}

    \subsubsection{Avvio}\label{sec:avvio}
        Questa attività consiste nei seguenti compiti:
        \begin{itemize}
            \item il fornitore conduce una revisione dei requisiti nella richiesta di proposta;
            \item il fornitore prende una decisione per fare un'offerta o accettare il contratto.
        \end{itemize}

    \subsubsection{Preparazione alla risposta}\label{sec:preparazione}
        Questa attività consiste nel seguente compito:
        \begin{itemize}
            \item il fornitore dovrebbe definire e preparare una proposta in risposta alla richiesta di proposta.
        \end{itemize}

    \subsubsection{Contrattazione}\label{sec:contrattazione}
        Questa attività consiste nei seguenti compiti:
        \begin{itemize}
            \item il fornitore deve negoziare e stipulare un contratto con l'organizzazione acquirente per fornire il prodotto o servizio software;
            \item il fornitore può richiedere un modifica al contratto.
        \end{itemize}

    \subsubsection{Pianificazione}\label{sec:pianificazione}
        Questa attività consiste nei seguenti compiti:
        \begin{itemize}
            \item il fornitore deve condurre una revisione dei requisiti di acquisizione per definire il quadro per la gestione e l'assicurazione del progetto e per assicurare la qualità del prodotto o servizio software consegnabile;
            \item se non è stipulato nel contratto, il fornitore deve definire o selezionare un modello di ciclo di vita del software appropriato all'ambito, all'entità e alla complessità del progetto. I processi, le attività e i compiti di questo Standard Internazionale devono essere selezionati e mappati sul modello di ciclo di vita.
            \item il fornitore deve stabilire i requisiti per i piani di gestione e assicurazione del progetto e per assicurare la qualità del prodotto o servizio software consegnabile. I requisiti per i piani dovrebbero includere le esigenze di risorse e il coinvolgimento dell’acquirente;
            \item una volta stabiliti i requisiti di pianificazione, il fornitore deve considerare le opzioni per lo sviluppo del prodotto software o la fornitura del servizio software, rispetto a un’analisi dei rischi associati a ciascuna opzione;
            \item le opzioni includono: 
            \begin{enumerate}
                \item sviluppare il prodotto software o fornire il servizio software utilizzando risorse interne;
                \item sviluppare il prodotto software o fornire il servizio software mediante subappalto;
                \item ottenere prodotti software pronti all’uso da fonti interne o esterne;
                \item una combinazione di 1, 2 e 3.
            \end{enumerate}
            \item Il fornitore deve sviluppare e documentare il piano di gestione del progetto sulla base dei requisiti di pianificazione e delle opzioni selezionate in precedenza;
            \item gli elementi da considerare nel piano includono, ma non sono limitati ai seguenti:
            \begin{enumerate}
                \item Struttura organizzativa del progetto e autorità e responsabilità di ciascuna unità organizzativa, comprese le organizzazioni esterne;
                \item Ambiente di ingegneria (per lo sviluppo, l’operazione o la manutenzione, se applicabile), incluso l’ambiente di test, la biblioteca, l’attrezzatura, le strutture, gli standard, le procedure e gli strumenti;
                \item Struttura di suddivisione del lavoro dei processi e delle attività del ciclo di vita, compresi i prodotti software, i servizi software e gli articoli non consegnabili, da eseguire insieme ai budget, al personale, alle risorse fisiche, alla dimensione del software e ai programmi associati ai compiti;
                \item Gestione delle caratteristiche di qualità dei prodotti o servizi software. Possono essere sviluppati piani separati per la qualità;
                \item Gestione della sicurezza, della sicurezza e di altri requisiti critici dei prodotti o servizi software. Possono essere sviluppati piani separati per la sicurezza e la sicurezza;
                \item Gestione del subappaltatore, compresa la selezione del subappaltatore e il coinvolgimento tra il subappaltatore e l’acquirente, se presente;
                \item Assicurazione della qualità (vedi 6.3);
                \item Verifica (vedi 6.4) e convalida (vedi 6.5); compreso l’approccio per interfacciarsi con l’agente di verifica e convalida, se specificato;
                \item Coinvolgimento dell’acquirente; cioè, per mezzo di revisioni congiunte (vedi 6.6), audit (vedi 6.7), incontri informali, rapporti, modifiche e cambiamenti; implementazione, approvazione, accettazione e accesso alle strutture;
                \item Coinvolgimento dell’utente; per mezzo di esercizi di impostazione dei requisiti, dimostrazioni e valutazioni del prototipo;
                \item Gestione del rischio; cioè, la gestione delle aree del progetto che comportano potenziali rischi tecnici, di costo e di programma;
                \item Politica di sicurezza; cioè, le regole per il bisogno di conoscere e l’accesso alle informazioni a ciascun livello dell’organizzazione del progetto;
                \item Approvazione richiesta da mezzi come regolamenti, certificazioni richieste, proprietario, utilizzo, diritti di proprietà, garanzia e licenza;
                \item Mezzi per la programmazione, il monitoraggio e la segnalazione;
                \item Formazione del personale (vedi 7.4).
            \end{enumerate}
        \end{itemize}

    \subsubsection{Esecuzione e controllo}\label{sec:esecuzione e controllo}
        Questa attività consiste nei seguenti compiti:
        \begin{itemize}
            \item il fornitore deve implementare ed eseguire il(i) piano(i) di gestione del progetto sviluppato in 5.2.4;
            \item il fornitore deve:
            \begin{enumerate}
                \item sviluppare il prodotto software in conformità con il Processo di Sviluppo (5.3);
                \item operare il prodotto software in conformità con il Processo di Operazione (5.4);
                \item mantenere il prodotto software in conformità con il Processo di Manutenzione (5.5).
            \end{enumerate}
            \item Il fornitore deve monitorare e controllare il progresso e la qualità dei prodotti o servizi software del progetto durante tutto il ciclo di vita contrattuale. Questo sarà un compito continuo e iterativo, che prevede:
            \begin{enumerate}
                \item monitoraggio del progresso delle prestazioni tecniche, dei costi e dei programmi e segnalazione dello stato del progetto; 
                \item identificazione, registrazione, analisi e risoluzione dei problemi.
            \end{enumerate} 
            \item Il fornitore deve gestire e controllare i subappaltatori in conformità con il Processo di Acquisizione (5.1). Il fornitore deve trasmettere tutti i requisiti contrattuali necessari per garantire che il prodotto o servizio software consegnato all’acquirente sia sviluppato o eseguito in conformità con i requisiti del contratto principale.
            \item il fornitore deve interfacciarsi con l'agente di verifica, validazione o test indipendente come specificato nel contratto e nei piani di progetto;
            \item il fornitore deve interfacciarsi con altre parti come specificato nel contratto e nei piani di progetto.
        \end{itemize}

        \subsubsection{Revisione e valutazione}\label{sec:revisione e valutazione}
            Questa attività consiste nei seguenti compiti:
            \begin{itemize}
                \item il fornitore dovrebbe coordinare le attività di revisione del contratto, le interfacce e la comunicazione con l'organizzazione dell'acquirente;
                \item il fornitore deve condurre o supportare gli incontri informali, la revisione di accettazione, i test di accettazione, le revisioni congiunte e gli audit con l'acquirente come specificato nel contratto e nei piani di progetto. Le revisioni congiunte devono essere condotte in conformità con 6.6, gli audit in conformità con 6.7;
                \item il fornitore deve eseguire la verifica e la validazione in conformità con 6.4 e 6.5 rispettivamente per dimostrare che i prodotti o servizi software e i processi soddisfano pienamente i rispettivi requisiti;
                \item il fornitore deve mettere a disposizione dell'acquirente i rapporti di valutazione, revisioni, audit, test e risoluzioni dei problemi come specificato nel contratto;
                \item il fornitore deve fornire all'acquirente l'accesso alle strutture del fornitore e dei subappaltatori per la revisione dei prodotti o servizi software come specificato nel contratto e nei piani di progetto;
                \item il fornitore deve eseguire le attività di assicurazione della qualità in conformità con 6.3.
            \end{itemize}
        
        \subsubsection{Consegna e completamento}\label{sec:consegna e completamento}
            Questa attività consiste nei seguenti compiti:
            \begin{itemize}
                \item il fornitore deve consegnare il prodotto o servizio software come specificato nel contratto;
                \item il fornitore deve fornire assistenza all'acquirente a supporto del prodotto software consegnato o del servizio come specificato nel contratto.
            \end{itemize}



    
        
        
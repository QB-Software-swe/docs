\subsection{Processo di fornitura}\label{sec:organizationalP}
Il processo di fornitura comprende le attività e i compiti del fornitore. Il processo ha come obiettivo determinare le procedure e le risorse necessarie per gestire e garantire il prodotto software all'acquirente.
Il fornitore gestisce il processo di fornitura seguendo il processo di gestione organizzativa e il processo di formazione. \\
    Come stabilito dallo standard ISO/IEC 12207:1997 \cite{bib:ISO12207_1997} il processo di fornitura identifica le seguenti attività:
    \begin{enumerate}
        \item avvio \ref{sec:avvio};
        \item preparazione della risposta \ref{sec:preparazione};
        \item contrattazione;
        \item pianificazione \ref{sec:pianificazione};
        \item esecuzione e controllo \ref{sec:esecuzione e controllo};
        \item revisione e valutazione \ref{sec:revisione e valutazione};
        \item consegna e completamento \ref{sec:consegna e completamento}.
    \end{enumerate}

    \subsubsection{Avvio}\label{sec:avvio}
        Questa attività consiste nei seguenti compiti:
        \begin{itemize}
            \item il fornitore conduce una revisione dei vari capitolati d'appalto;
            \item il fornitore produce il documento Valutazione dei capitolati, che comprende per ogni capitolato valutato:
            \begin{enumerate}
                \item una breve descrizione;
                \item il dominio applicativo;
                \item il dominio tecnologico;
                \item gli aspetti positivi;
                \item i fattori critici;
                \item conclusioni.
            \end{enumerate}
        \end{itemize}

    \subsubsection{Preparazione alla risposta}\label{sec:preparazione}
        Il fornitore, dopo aver scelto il capitolato, deve produrre i seguenti documenti:
        \begin{itemize}
            \item la Lettera di presentazione, che comprende: 
            \item \begin{enumerate}
                \item il capitolato scelto;
                \item una lista dei documenti allegati;
                \item una lista dei membri del gruppo.
            \end{enumerate}
            \item Il Preventivo dei costi e degli impegni, che comprende:  
            \begin{enumerate}
                \item gli impegni orari per ogni ruolo;
                \item una breve considerazione per ogni ruolo;
                \item preventivo dei costi, basato sulle tariffe orarie dei ruoli;
                \item scadenza di consegna. 
            \end{enumerate}
        \end{itemize}

    \subsubsection{Pianificazione}\label{sec:pianificazione}
        Questa attività consiste nella produzione dei seguenti documenti:
        \begin{itemize}
            \item il Piano di Progetto, che comprende:
            \begin{enumerate}
                \item analisi dei rischi: analizza i rischi che possono influenzare la pianificazione del progetto o la qualità del software;
                \item modello di sviluppo: analizza il modello scelto dal fornitore;
                \item pianificazione temporale: analizza la pianificazione temporale del progetto, e in particolare la suddivisione in fasi e la loro durata;
                \item preventivi di periodo: viene calcolata una stima del tempo di lavoro necessario e dei costi per ogni fase;
                \item consuntivi di periodo: riporta le attività effettivamente completate e i costi reali, sia in termini economici che temporali. 
            \end{enumerate}
            \item Il Piano di Qualifica;
            \item l'Analisi dei requisiti, che comprende:
            \begin{enumerate}
                \item introduzione: viene descritto lo scopo del documento e lo scopo del progetto. Inoltre vengono forniti i riferimenti normativi e un glossario;  
                \item descrizione del prodotto;
                \item analisi dei casi d'uso oblligatori;
                \item analisi dei casi d'uso facoltativi.
            \end{enumerate}
        \end{itemize}

    \subsubsection{Esecuzione e controllo}\label{sec:esecuzione e controllo}
        Questa attività consiste nei seguenti compiti:
        \begin{itemize}
            \item il fornitore deve implementare ed eseguire il Piano di Progetto;
            \item il fornitore deve sviluppare il prodotto software in conformità con il processo di sviluppo;
        \end{itemize}

        \subsubsection{Revisione e valutazione}\label{sec:revisione e valutazione}
            Questa attività consiste nell'eseguire la verifica e la validazione del prodotto software in conformità con il processo di verifica e di validazione. 
        
        \subsubsection{Consegna e completamento}\label{sec:consegna e completamento}
            Questa attività consiste nei seguenti compiti:
            \begin{itemize}
                \item il fornitore deve consegnare il prodotto software;
                \item il fornitore deve fornire la documentazione necessaria.
            \end{itemize}



    
        
        
\section{Prefazione} \label{sec:prefazione}
    Con questo documento QB Software intende normare i propri processi per lo sviluppo di un progetto software. Tali processi sono stati creati a partire dallo standard ISO 12207 del 1997 \cite{bib:ISO12207_1997} proposto durante il corso di ingegneria del software.

    La struttura di questo documento segue lo standard \cite{bib:ISO12207_1997}, dove ogni processo è indicato da un numero (a)%
    \footnote{Eccezione per la sezione: \ref{sec:prefazione} e \ref{sec:bib}.}%
    , ogni attività è indicata dal numero (a.b), e ogni task è indicata dalla numerazione (a.b.c). La scelta di avere un documento per struttura simile allo standard ci permette di mantenere il più fedelmente possibile le linee guida dettate dallo standard stesso. Le norme presentate in questo documento hanno lo scopo di essere il più prescrittive possibile, al fine di definire in modo "algoritmico" le procedure di lavoro, e limitare fortemente lo spazio alle scelte di libero arbitrio che rischiano di portare a situazioni meno controllate rispetto a quelle previste durante la stesura del way of working. 

    Prima di leggere un qualunque documento prodotto da QB Software è necessario conoscere il significato dei termini riportati nel documento: \verb|glossario|; presente nel \href{https://github.com/QB-Software-swe/docs}{repository GitHub con la documentazione di QB Software}.

    \begin{center}
        \begin{tabularx}{0.85\textwidth}{>{\centering\arraybackslash}X}
            \toprule
            Ogni membro del gruppo si impegna a leggere, a comprendere, e a mettere in pratica in pieno le norme presenti in questo documento.
            \\\bottomrule
        \end{tabularx}
    \end{center}
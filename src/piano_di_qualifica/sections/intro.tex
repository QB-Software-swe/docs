\section{Introduzione}

\subsection{Scopo del documento}
Lo scopo del \textit{Piano di Qualifica} è raccogliere e documentare le strategie che il gruppo QB Software
adotta per assicurare sia la qualità di processo che di prodotto.

\subsection{Premessa}


\subsection{Scopo del Prodotto}
Il fine è quello di implementare un servizio di posta elettronica che utilizza il protocollo \textit{JMAP\textsubscript{G}}.
Il servizio deve essere testabile per permettere all'azienda di valutare le prestazioni, la manuntenibilità e 
la completezza del protocollo JMAP, paragonandolo agli attuali protocolli già implementati in 
\textit{Carbonio\textsubscript{G}}, prodotto open source dell'azienda Zexstras ideato per la collaborazione 
e la gestione dell'e-mail.


\subsection{Glossario}
Per una maggiore chiarezza di contenuto dei documenti redatti, viene fornito in allegato un il \textit{Glossario V1.0.0}, 
dove vengono definiti tutti i termini con un significato particolare o di rilievo nell'ambito del progetto. 
Un termine presente nel \textit{Glossario} viene contrasseganto dal testo formattato in corsivo, seguito da una "G"
a pedice.


\subsection{Riferimenti}
    \subsubsection{Normativi}
        \begin{itemize}
            \item \textit{Norme di Progetto V1.0.0};
            \item Capitolato d'appalto C8: 
                \begin{itemize}
                    \item   \href{https://www.math.unipd.it/~tullio/IS-1/2023/Progetto/C8.pdf}{https://www.math.unipd.it/~tullio/IS-1/2023/Progetto/C8.pdf};
                    \item   \href{https://www.math.unipd.it/~tullio/IS-1/2023/Progetto/C8p.pdf}{https://www.math.unipd.it/~tullio/IS-1/2023/Progetto/C8p.pdf}.
                \end{itemize}
            \item verbali?
        \end{itemize}

    \subsubsection{Informativi}
        \begin{itemize}
            \item \textit{Analisi dei Requisiti V1.0.0};
            \item \textit{Piano di Progetto V1.0.0};
            \item Standard ISO/IEC 12207:1995:
            
                \href{https://www.math.unipd.it/~tullio/IS-1/2009/Approfondimenti/ISO_12207-1995.pdf}{https://www.math.unipd.it/~tullio/IS-1/2009/Approfondimenti/ISO\_12207-1995.pdf};

            \item Standard ISO/IEC 9126:
            
                 \href{https://it.wikipedia.org/wiki/ISO/IEC_9126}{https://it.wikipedia.org/wiki/ISO/IEC\_9126};
            \clearpage
            \item Dispense dell'insegnamento di Ingegnieria del Software:
                \begin{itemize}
                    \item Progettazione software:
                    
                        \href{https://www.math.unipd.it/~tullio/IS-1/2023/Dispense/T6.pdf}{https://www.math.unipd.it/~tullio/IS-1/2023/Dispense/T6.pdf}

                    \item Qualità del software:
                    
                        \href{https://www.math.unipd.it/~tullio/IS-1/2023/Dispense/T7.pdf}{https://www.math.unipd.it/~tullio/IS-1/2023/Dispense/T7.pdf};

                    \item Qualità di processo:
                    
                    \href{https://www.math.unipd.it/~tullio/IS-1/2023/Dispense/T8.pdf}{https://www.math.unipd.it/~tullio/IS-1/2023/Dispense/T8.pdf};


                    \item Verifica e validazione:
                    
                    \href{https://www.math.unipd.it/~tullio/IS-1/2023/Dispense/T9.pdf}{https://www.math.unipd.it/~tullio/IS-1/2023/Dispense/T9.pdf}.
                \end{itemize}
            \item Modello a V:
            
                \href{https://en.wikipedia.org/wiki/V-model_(software_development)}{https://en.wikipedia.org/wiki/V-model\_(software\_development)}
            \item Metriche per testing di qualità:
            
                \href{https://www.tricentis.com/blog/64-essential-testing-metrics-for-measuring-quality-assurance-success}{tehttps://www.tricentis.com/blog/64-essential-testing-metrics-for-measuring-quality-assurance-successxt}
        \end{itemize}
        \clearpage


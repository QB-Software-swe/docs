\section{Valutazioni per il miglioramento} \label{valutazioni}
In questa sezione viene analizzato il processo di automiglioramento che il gruppo
\textit{QB Software} intraprende per gestire i problemi riscontrati lungo il corso del progetto.
Viene fornito un registro cronologico, comprendente di tutti i maggiori problemi incontrati
e le rispettive soluzioni che sono state adottate.

\noindent Riprendendo la categorizzazione dei rischi nel \textit{Piano di Progetto V1.0.0},
si possono suddividere in:

\begin{itemize}
    \item \textbf{rischi tecnologici}: rappresentano l'insieme dei problemi dovuto all'utilizzo di 
                nuove tecnologie e all'inesperienze del gruppo, che possono rallentare il processo;
    \item \textbf{rischi legati alle persone}: rappresentano l'insieme dei problemi che nascono dalla diversità di
                carattere, idee e personalità all'interno del gruppo, le quali possono generare conflitti o schrezi interni;
    \item  \textbf{rischi organizzativi}: rappresentano l'insieme dei problemi di carattere organizzativo
                e gestionale, dovute alla complessità del progetto;
    \item \textbf{rischi sulle stime}: rappresentano l'insieme dei problemi nati da una stima troppo ottimistica e poco reale di un problema,
                che porta un effetto a cascata dove si accumulano sempre più ritardi sulla tabella di marcia.
    \item \textbf{rischi sui requisiti}: rappresentano l'insieme dei problemi legati all'intereptazione 
    e allo sviluppo dell'analisi dei requisiti.
\end{itemize}


\subsection{Valutazione tecnologica}

    \begin{table}[h]
        \rowcolors{2}{gray}{white}
        \begin{adjustbox}{width=\textwidth}
            \centering
            \renewcommand{\arraystretch}{1.1}
            \begin{tabular}{>{\centering\arraybackslash} m{5cm} >{\centering\arraybackslash} m{1.5cm} >{\centering\arraybackslash} m{4cm}}
                \rowcolor[HTML]{bfbfbf} 
                \textbf{Problema emerso} & \textbf{Rishio} & \textbf{Reazione} \\
                Durante le prime fasi del progetto, alcuni membri avevano inesperienza con \LaTeX & RT1 & È stato applicato il piano di contingenza previsto \\
                Durante le prime fasi del progetto, alcuni membri avevano inesperienza con \textit{Git}\textsubscript{G} e  \textit{GitHub}\textsubscript{G} & RT1 & È stato applicato il piano di contingenza previsto \\
            \end{tabular}
        \end{adjustbox}
        \caption{Valutazione tecnologica}
    \end{table}
    \clearpage

    
    \begin{table}[h]
        \rowcolors{2}{gray}{white}
        \begin{adjustbox}{width=\textwidth}
            \centering
            \renewcommand{\arraystretch}{1.1}
            \begin{tabular}{>{\centering\arraybackslash} m{5cm} >{\centering\arraybackslash} m{1.5cm} >{\centering\arraybackslash} m{4cm}}
                \rowcolor[HTML]{bfbfbf} 
                \textbf{Problema emerso} & \textbf{Rishio} & \textbf{Reazione} \\
                I membri del gruppo addetti alla creazione del \textit{PoC}\textsubscript{G} hanno incontrato difficoltà con lo studio e l'implementazione delle librerie & RT1 & È stato applicato il piano di contingenza previsto \\
            \end{tabular}
        \end{adjustbox}
        \caption*{Tabella 8 (continuazione): valutazione tecnologica}
    \end{table}


\vfil    
\subsection{Valutazione relazioni interpersonali}

    \begin{table}[h]
        \rowcolors{2}{gray}{white}
        \begin{adjustbox}{width=\textwidth}
            \centering
            \renewcommand{\arraystretch}{1.1}
            \begin{tabular}{>{\centering\arraybackslash} m{5cm} >{\centering\arraybackslash} m{1.5cm} >{\centering\arraybackslash} m{4cm}}
                \rowcolor[HTML]{bfbfbf} 
                \textbf{Problema emerso} & \textbf{Rishio} & \textbf{Reazione} \\
                Per via dei diversi caratteri dei membri del gruppo, si è generato un clima di tensione & RP1 &  È stato applicato il piano di contingenza e di controllo \\
            \end{tabular}
        \end{adjustbox}
        \caption{Valutazione relazioni interpersonali}
    \end{table}
    \clearpage

\subsection{Rischi organizzativi}
    \begin{table}[h]
        \rowcolors{2}{gray}{white}
        \begin{adjustbox}{width=\textwidth}
            \centering
            \renewcommand{\arraystretch}{1.1}
            \begin{tabular}{>{\centering\arraybackslash} m{5cm} >{\centering\arraybackslash} m{1.5cm} >{\centering\arraybackslash} m{4cm}}
                \rowcolor[HTML]{bfbfbf} 
                \textbf{Problema emerso} & \textbf{Rishio} & \textbf{Reazione} \\
                A causa dell'inesperienza del gruppo nel gestire i progetti, ci sono stati problemi di comunicazione & RO1 & Il gruppo si impegna ad esporsi in maniera più chiara e trasparente \\
                A causa del licenziamento del project manager affidatoci da Zextras, il gruppo ha avuto difficoltà a fissare un incontro con l'azienda & RO5 & È stato applicato il piano di contingenza previsto \\
                A seguito di un intervento medico, un membro del gruppo non ha potuto essere presente per una settimana & RO2 & È stato applicato il piano di controllo previsto\\
                In concomitanza alle festività invernali, alcuni membri hanno avuto impegni personali inderogabili & RO2 & È stato applicato il piano di contingenza previsto \\
                Con l'inizio della sessione invernale degli esami, il gruppo ha rallentato l'avanzamento del progetto & RO3 & Il gruppo si impegna a recuperare il tempo perso
            \end{tabular}
        \end{adjustbox}
        \caption{Valutazione organizzativa}
    \end{table}

\clearpage
\subsection{Rischi sulle stime}
    \begin{table}[h]
        \rowcolors{2}{gray}{white}
        \begin{adjustbox}{width=\textwidth}
            \centering
            \renewcommand{\arraystretch}{1.1}
                \begin{tabular}{>{\centering\arraybackslash} m{5cm} >{\centering\arraybackslash} m{1.5cm} >{\centering\arraybackslash} m{4cm}}
                    \rowcolor[HTML]{bfbfbf} 
                    \textbf{Problema emerso} & \textbf{Rishio} & \textbf{Reazione} \\
                    A causa di complicanze dovute all'analisi dei requisiti e dal PoC, il gruppo non riesce presentarsi al colloquio dell'RTB entro i tempi previsti & RS2 & Il gruppo ha deciso di posticipare la data per presentarsi in maniera adeguata\\
                \end{tabular}
            \end{adjustbox}
        \caption{Valutazione sulle stime}
    \end{table}


\subsection{Rischi sui requisiti}
    \begin{table}[h]
        \rowcolors{2}{gray}{white}
        \begin{adjustbox}{width=\textwidth}
            \centering
            \renewcommand{\arraystretch}{1.1}
                \begin{tabular}{>{\centering\arraybackslash} m{5cm} >{\centering\arraybackslash} m{1.5cm} >{\centering\arraybackslash} m{4cm}}
                    \rowcolor[HTML]{bfbfbf} 
                    \textbf{Problema emerso} & \textbf{Rishio} & \textbf{Reazione} \\
                    A seguito di un colloquio con il professor Cardin, il gruppo si è accorto si aver sbagliato l'analisi dei requisiti & RR1 & È stato applicato il piano di contingenza previsto \\
                \end{tabular}
            \end{adjustbox}
        \caption{Valutazione sui requisiti}
    \end{table}
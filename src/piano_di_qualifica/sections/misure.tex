\section{Misure per la qualità} \label{misure}
Per misurare ogni processo sono state usate delle metriche la cui definizione
 è nelle \textit{Norme di Progetto V1.0.0}, che si rifanno allo standard 
ISO/IEC 9126. In questa sezione sono riportati i valori che le metriche 
devono assumere per essere ritenute accettabili o pienamente soddisfatte.

\subsection{Qualità di processo}



\begin{table}[h]
    \rowcolors{2}{gray}{white}
    \begin{adjustbox}{width=\textwidth}
    \centering
    \renewcommand{\arraystretch}{1.5}
    \begin{tabular}{>{\centering\arraybackslash} m{2cm} >{\centering\arraybackslash} m{6cm} >{\centering\arraybackslash} m{3cm} >{\centering\arraybackslash} m{3cm}}
    \rowcolor[HTML]{bfbfbf} 
    \textbf{Metrica} & \textbf{Nome} & \textbf{Valore accettabile} & \textbf{Valore preferibile} \\
    M1-PMS & Percentuale di  Metriche soddisfatte & $\ge 85\%$ & $100\%$\\
    M2-VP & Variazione di Piano & $\ge -7$ & $\ge 0$\\
    M3-VC & Variazione di Costo & $0$ & $\le 0$\\
    M4-VR & Variazione dei Requisiti & $\le 4$ & $0$\\
    M5-IF & Implementazione delle funzionalità & $100\%$ & $100\%$\\
    M6-CCM & Complessità Ciclomatica per Metodo & $\le 5$ & $\le 3$\\
    M7-CC  & Code Coverage & $\ge 85\%$ & $100\%$               
    \end{tabular}
    \end{adjustbox}
    \caption{Metriche per la qualità dei processi}
\end{table}

\clearpage


\subsection{Qualità di prodotto}

\begin{table}[h]
    \rowcolors{2}{gray}{white}
    \begin{adjustbox}{width=\textwidth}
    \centering
    \renewcommand{\arraystretch}{1.5}
    \begin{tabular}{>{\centering\arraybackslash} m{2cm} >{\centering\arraybackslash} m{6cm} >{\centering\arraybackslash} m{3cm} >{\centering\arraybackslash} m{3cm}}
    \rowcolor[HTML]{bfbfbf} 
    \textbf{Metrica} & \textbf{Nome} & \textbf{Valore accettabile} & \textbf{Valore preferibile} \\
    M8-PROS & Percentuale dei Requisiti Obbligatori Soddisfatti &  $100\%$ & $100\%$\\
    M9-PRDS & Percentuale dei Requisiti Desiderabili Soddisfatti & $\ge 0$ & $100\%$\\
    M10-PRFS & Percentuale dei Requisiti Facoltativi Soddisfatti &   $\ge 0$ & $100\%$\\
    M11-AC & Accoppiamento tra Classi & $\le 4$ & $\le 2$ \\
    M12-PG & Profondità delle Gerarchie & $\le 5$ & $\le 2$\\
    M13-NAC & Numero di Attributi per Classe & $\le 8$ & $\le 6$ \\
    M14-NPM & Numero di Parametri per Metodo & $\le 6$ & $\le 4$ \\
    M15-LCM & Linee di Codice per Metodo & $\le 40$ & $\le 25$ \\
    M16-LCC & Linee di Commenti per linee di Codice & $\le 0.30$ & $\le 0.35$\\
    M17-PIC & Profondità di Innestamento Condizionale & $\le 5$ & $\le 3$\\
    M18-DE & Densità di Errori & $\le 10\%$ & $0\%$\\
    M19-FU & Facilità di Utilizzo & $\le 15$ & $\le 10$ \\ 
    \end{tabular}
    \end{adjustbox}
    \caption{Metriche per la qualità del prodotto}
\end{table}

\clearpage

\subsection{Qualità per Obiettivo}
Le metriche precedentemente elencate, verranno divise in base alle standard ISO/
IEC 12207:1995, ovvero tra:
\begin{itemize}
    \item Processi primari (\ref{processi primari});
    \item Processi di supporto (\ref{processi di supporto});
    \item Processi organizzativi (\ref{processi organizzativi}).
\end{itemize}

\noindent Tuttavia, lo standard è stato adattato e semplificato per il progetto.


\subsubsection{Processi primari} \label{processi primari}

\paragraph{Analisi dei Requisiti}

\begin{table}[h]
    \rowcolors{2}{gray}{white}
    \begin{adjustbox}{width=\textwidth}
    \centering
    \renewcommand{\arraystretch}{1.5}
    \begin{tabular}{>{\centering\arraybackslash} m{2cm} >{\centering\arraybackslash} m{6cm} >{\centering\arraybackslash} m{3cm} >{\centering\arraybackslash} m{3cm}}
    \rowcolor[HTML]{bfbfbf} 
    \textbf{Metrica} & \textbf{Nome} & \textbf{Valore accettabile} & \textbf{Valore preferibile} \\
    M8-PROS & Percentuale dei Requisiti Obbligatori Soddisfatti &  $100\%$ & $100\%$\\
    M9-PRDS & Percentuale dei Requisiti Desiderabili Soddisfatti & $\ge 0$ & $100\%$\\
    M10-PRFS & Percentuale dei Requisiti Facoltativi Soddisfatti &   $\ge 0$ & $100\%$\\
    \end{tabular}
    \end{adjustbox}
    \caption{Metriche per l'analisi dei requisiti}
\end{table}

\clearpage


\paragraph{Progettazione}

\begin{table}[h]
    \rowcolors{2}{gray}{white}
    \begin{adjustbox}{width=\textwidth}
    \centering
    \renewcommand{\arraystretch}{1.5}
    \begin{tabular}{>{\centering\arraybackslash} m{2cm} >{\centering\arraybackslash} m{6cm} >{\centering\arraybackslash} m{3cm} >{\centering\arraybackslash} m{3cm}}
    \rowcolor[HTML]{bfbfbf} 
    \textbf{Metrica} & \textbf{Nome} & \textbf{Valore accettabile} & \textbf{Valore preferibile} \\
    M11-AC & Accoppiamento tra Classi & $\le 4$ & $\le 2$ \\
    M12-PG & Profondità delle Gerarchie & $\le 5$ & $\le 2$\\
    M19-FU & Facilità di Utilizzo & $\le 15$ & $\le 10$ \\ 
    \end{tabular}
    \end{adjustbox}
    \caption{Metriche per la progettazione}
\end{table}


\paragraph{Codifica}

\begin{table}[h]
    \rowcolors{2}{gray}{white}
    \begin{adjustbox}{width=\textwidth}
    \centering
    \renewcommand{\arraystretch}{1.5}
    \begin{tabular}{>{\centering\arraybackslash} m{2cm} >{\centering\arraybackslash} m{6cm} >{\centering\arraybackslash} m{3cm} >{\centering\arraybackslash} m{3cm}}
    \rowcolor[HTML]{bfbfbf} 
    \textbf{Metrica} & \textbf{Nome} & \textbf{Valore accettabile} & \textbf{Valore preferibile} \\
    M6-CCM & Complessità Ciclomatica per Metodo & $\le 5$ & $\le 3$\\
    M7-CC  & Code Coverage & $\ge 85\%$ & $100\%$ \\
    M13-NAC & Numero di Attributi per Classe & $\le 8$ & $\le 6$ \\
    M14-NPM & Numero di Parametri per Metodo & $\le 6$ & $\le 4$ \\
    M15-LCM & Linee di Codice per Metodo & $\le 40$ & $\le 25$ \\
    M16-LCC & Linee di Commenti per linee di Codice & $\le 0.30$ & $\le 0.35$\\
    M17-PIC & Profondità di Innestamento Condizionale & $\le 5$ & $\le 3$\\
    \end{tabular}
    \end{adjustbox}
    \caption{Metriche per la codifica}
\end{table}

\clearpage


\subsubsection{Processi di supporto} \label{processi di supporto}

\paragraph{Miglioramento}

\begin{table}[h]
    \rowcolors{2}{gray}{white}
    \begin{adjustbox}{width=\textwidth}
    \centering
    \renewcommand{\arraystretch}{1.5}
    \begin{tabular}{>{\centering\arraybackslash} m{2cm} >{\centering\arraybackslash} m{6cm} >{\centering\arraybackslash} m{3cm} >{\centering\arraybackslash} m{3cm}}
    \rowcolor[HTML]{bfbfbf} 
    \textbf{Metrica} & \textbf{Nome} & \textbf{Valore accettabile} & \textbf{Valore preferibile} \\
    M1-PMS & Percentuale di  Metriche soddisfatte & $\ge 85\%$ & $100\%$\\
    M18-DE & Densità di Errori & $\le 10\%$ & $0\%$\\
    \end{tabular}
    \end{adjustbox}
    \caption{Metriche per il miglioramento}
\end{table}


\subsubsection{Processi organizzativi} \label{processi organizzativi}
\paragraph{Pianificazione}

\begin{table}[h]
    \rowcolors{2}{gray}{white}
    \begin{adjustbox}{width=\textwidth}
    \centering
    \renewcommand{\arraystretch}{1.5}
    \begin{tabular}{>{\centering\arraybackslash} m{2cm} >{\centering\arraybackslash} m{6cm} >{\centering\arraybackslash} m{3cm} >{\centering\arraybackslash} m{3cm}}
    \rowcolor[HTML]{bfbfbf} 
    \textbf{Metrica} & \textbf{Nome} & \textbf{Valore accettabile} & \textbf{Valore preferibile} \\
    M2-VP & Variazione di Piano & $\ge -7$ & $\ge 0$\\
    M3-VC & Variazione di Costo & $0$ & $\le 0$\\
    M4-VR & Variazione dei Requisiti & $\le 4$ & $0$\\
    M5-IF & Implementazione delle funzionalità & $100\%$ & $100\%$\\
    \end{tabular}
    \end{adjustbox}
    \caption{Metriche per la pianificazione}
\end{table}